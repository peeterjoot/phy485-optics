%
% Copyright � 2013 Peeter Joot.  All Rights Reserved.
% Licenced as described in the file LICENSE under the root directory of this GIT repository.
%
\makeoproblem{Spatial coherence, grating.}
{modernOptics:problemSet3:3}
{2012 Ps3, P3}
{
A look inside a grating spectrometer reveals that incident light is passed through a series of slits to increase the transverse spatial coherence. In this problem, we'll try to understand why. For all of the parts below, consider a grating of \(N\) slits, periodicity \(a\), and width \(b\).

For parts \ref{modernOptics:problemSet3:3c} and \ref{modernOptics:problemSet3:3d}, neglect the envelope due to a finite slit width \(b\), and consider only the sharp diffraction peaks.

\makesubproblem{Wavelength resolution}{modernOptics:problemSet3:3a}

For a collimated (\(k_{s,y}=0\)), monochromatic source illuminating \(N\) slits, diffraction peaks would have an angular width of \(\Delta \theta = \lambda / (N a \cos{\theta})\), for the first order of diffraction. {\bf re-derive this result} for yourself. Show that this gives a wavelength resolution is \(\Delta \lambda = \lambda/N m\)

\makesubproblem{Intensity}{modernOptics:problemSet3:3b}

Next, consider how the output intensity of the grating shifts if the input comes in at an angle \(\theta_s\). {\bf Write an expression for \(I(\theta_s, \theta)\).} You can also use the variables \(k_{s,y}=k \sin \theta_s\) and \(k_{y}=k \sin \theta\), as we did in class.

\makesubproblem{Resolution of spectrometer}{modernOptics:problemSet3:3c}
If the incident beam has an angular spread \(\Delta \theta_s\) around normal incidence, what is the {\bf resolution \(\Delta \lambda\) of the spectrometer?} Calculate this in the limit of large \(N\), or \(N \gg \lambda / a \Delta \theta_s\), where the angular width of the diffracted light is completely determined by the angular width of the incident light.

\makesubproblem{Decreased coherence length at the grating}{modernOptics:problemSet3:3d}

An alternate view of \partref{modernOptics:problemSet3:3c} is that by broadening the angular distribution of the source, we also decrease the transverse coherence length at the grating. The number of slits leading to coherent diffraction is reduced to some \(N_{\mathrm{eff}}\), which is the number of slits within one coherence length \(\ell_{\mathrm{tc}} = \lambda / \Delta \theta_s\). {\bf Sketch a diagram explaining this.} The frequency resolution of the spectrometer is then reduced from \(\lambda/N\) to \(\lambda/N_{\mathrm{eff}}\). (for order \(m=1\)) {\bf Compare this to the result you found in \partref{modernOptics:problemSet3:3c}.}

} % makeproblem

\makeanswer{modernOptics:problemSet3:3}{
\makeSubAnswer{Output intensity given input angle \(\theta_s\)}{modernOptics:problemSet3:3b}

Let's derive the \(N\) slit diffraction wave function and intensity given an off normal input.  We'll be able to use this in \partref{modernOptics:problemSet3:3a} once we do.  We'll use a Fraunhofer geometry as in \cref{fig:modernOpticsProblemSet3Problem3:modernOpticsProblemSet3Problem3Fig1}.
%
\imageFigure{../figures/phy485-optics/modernOpticsProblemSet3Problem3Fig1}{Fraunhofer geometry.}{fig:modernOpticsProblemSet3Problem3:modernOpticsProblemSet3Problem3Fig1}{0.2}
%
\begin{subequations}
\begin{dmath}\label{eqn:modernOptics:problemSet3:3:20}
\BR + \Br' = \Br.
\end{dmath}
\begin{dmath}\label{eqn:modernOptics:problemSet3:3:40}
\BR_s + \Br' = \Br_s.
\end{dmath}
\end{subequations}
%
for which our path length from \(\Br'\) to the observation point is
%
\begin{dmath}\label{eqn:modernOptics:problemSet3:3:60}
\Abs{\BR}
= r \left( 1 + \frac{{r'}^2}{r^2} - 2 \frac{\Br \cdot \Br'}{r^2} \right)^{1/2}
\sim r + \frac{{r'}^2}{2 r^2} - \rcap \cdot \Br',
\end{dmath}
%
and to first order
%
\begin{dmath}\label{eqn:modernOptics:problemSet3:3:80}
k\Abs{\BR}
\sim k r - k \rcap \cdot \Br'
= k r - k y' \sin\theta.
\end{dmath}
%
Similarly
\begin{dmath}\label{eqn:modernOptics:problemSet3:3:100}
k\Abs{\BR_s}
\sim k r_s - k \rcap \cdot \Br'
= k r_s + \Bk \cdot \Br'
= k r_s + k \sin\theta_s y'.
\end{dmath}
%
With
%
\begin{subequations}
\begin{dmath}\label{eqn:modernOptics:problemSet3:3:120}
k_y \equiv \frac{2 \pi}{\lambda} \sin\theta,
\end{dmath}
\begin{dmath}\label{eqn:modernOptics:problemSet3:3:140}
k_{y,s} \equiv \frac{2 \pi}{\lambda} \sin\theta_s.
\end{dmath}
\end{subequations}
%
Our diffraction integral
%
\begin{dmath}\label{eqn:modernOptics:problemSet3:3:160}
\Psi \sim \int \frac{e^{i k (R + R_s)}}{ R R_s},
\end{dmath}
%
after pulling out and dropping the \(r\) and \(r_s\) dependent terms, takes the one dimensional form
%
\begin{dmath}\label{eqn:modernOptics:problemSet3:3:180}
\Psi(\Br) \sim \int e^{i (k_{y,s} - k_{y}) y' } dy'
=
\int e^{i \frac{2 \pi}{\lambda} (\sin\theta_s - \sin\theta) y' } dy'.
\end{dmath}
%
Let's write
%
\begin{dmath}\label{eqn:modernOptics:problemSet3:3:200}
\Delta k = k_{y} - k_{y,s},
\end{dmath}
%
and evaluate this over intervals \([h + m a, h + m a + b]\), for \(m \in [0, N-1]\) as in \cref{fig:modernOpticsProblemSet3Problem3:modernOpticsProblemSet3Problem3Fig2}.
%
\imageFigure{../figures/phy485-optics/modernOpticsProblemSet3Problem3Fig2}{N slit geometry.}{fig:modernOpticsProblemSet3Problem3:modernOpticsProblemSet3Problem3Fig2}{0.2}
%
Integrating over the \(m\)th slit, we have
%
\begin{dmath}\label{eqn:modernOptics:problemSet3:3:220}
\int_{S_m} e^{-i \Delta k y' } dy'
=
\int_{h + m a }^{h + m a + b} e^{-i \Delta k y' } dy'
=
\evalrange{\frac{e^{-i \Delta k y' }}{-i \Delta k }}
{h + m a }{h + m a + b}
=
\frac{e^{-i \Delta k (h + m a) }}{-i \Delta k }
\left( e^{-i \Delta k b } - 1 \right)
=
\frac{e^{-i \Delta k (h + m a ) }}{\Delta k }
e^{i \Delta k b/2 } 2 \sin( \Delta k b/2 )
=
b e^{-i \Delta k (h + m a ) }
e^{i \Delta k b/2 } \frac{\sin( \Delta k b/2 )}{ \Delta k b/2}.
\end{dmath}
%
Adding all the slit contributions we have
%
\begin{dmath}\label{eqn:modernOptics:problemSet3:3:240}
\Psi =
b
e^{i \Delta k (b/2 -h) }
\frac{\sin( \Delta k b/2 )}{ \Delta k b/2}
\sum_{m = 0}^{N-1}
e^{-i \Delta k m a }
=
b
e^{i \Delta k (b/2 -h) }
\frac{\sin( \Delta k b/2 )}{ \Delta k b/2}
\frac{1 - e^{-i \Delta k a N }}
{1 - e^{-i \Delta k a N }}
=
b
e^{i \Delta k (b/2 -h) }
\frac{\sin( \Delta k b/2 )}{ \Delta k b/2}
\frac
{
e^{-i \Delta k a N/2 }
}
{
e^{-i \Delta k a /2 }
}
\frac
{
e^{i \Delta k a N/2 } - e^{-i \Delta k a N/2 }
}
{
e^{i \Delta k a /2 } - e^{-i \Delta k a /2 }
}.
\end{dmath}
%
\begin{dmath}\label{eqn:modernOptics:problemSet3:3:260}
\Psi \sim
\frac{\sin( \Delta k b/2 )}{ \Delta k b/2}
\frac
{\sin( \Delta k a N/2 )}
{N \sin( \Delta k a /2 )},
\end{dmath}
%
with intensity
%
\begin{subequations}
\label{eqn:modernOptics:problemSet3:3:280a}
\boxedEquation{eqn:modernOptics:problemSet3:3:280}{
I(\theta_s, \theta)
\sim
\frac{\sin^2( \Delta k b/2 )}{ (\Delta k b/2)^2}
\frac
{\sin^2( \Delta k a N/2 )}
{N^2 \sin^2( \Delta k a /2 )},
}
\begin{dmath}\label{eqn:modernOptics:problemSet3:3:300}
\Delta k = \frac{2 \pi}{\lambda} ( \sin\theta - \sin\theta_s ).
\end{dmath}
\end{subequations}
%
\makeSubAnswer{Peak width and wavelength resolution for normal incidence}{modernOptics:problemSet3:3a}

Here we work with a normal incident \(\theta_s = 0\) plane wave source, and write
%
\begin{dmath}\label{eqn:modernOptics:problemSet3:3:320}
\gamma
= \frac{\Delta k a}{2}
= \frac{k a}{2} \sin\theta
= \frac{\pi a}{\lambda} \sin\theta.
\end{dmath}
%
and seek to understand the characteristics of the Intensity envelope
%
\begin{dmath}\label{eqn:modernOptics:problemSet3:3:340}
\frac{\sin^2( N \gamma / 2 )}
{N^2 \sin^2( \gamma/2) }.
\end{dmath}
%
To get a feel for what this may look like this is plotted for two wave lengths \(\lambda = 3 \pi a\), \(\lambda' = 4 \pi a\), \(a = 1\) in \cref{fig:modernOpticsProblemSet3Problem3:modernOpticsProblemSet3Problem3Fig3}.
%
\imageFigure{../figures/phy485-optics/modernOpticsProblemSet3Problem3Fig3}{Intensity envelope sample plot.}{fig:modernOpticsProblemSet3Problem3:modernOpticsProblemSet3Problem3Fig3}{0.2}
%
Observe that this ratio of sines has a unit value for any \(\gamma/2 = m \pi\), for integer \(m\) since by H'\^{o}pital's rule we have
%
\begin{dmath}\label{eqn:modernOptics:problemSet3:3:360}
\lim_{\gamma/2 \rightarrow m \pi}
\frac{\sin( N \gamma / 2 )}
{N \sin( \gamma/2) }
=
\evalbar{\frac{\cos( N \gamma / 2 )}
{\cos( \gamma/2) }}{\gamma = m \pi}
= (-1)^{(N -1) m}.
\end{dmath}
%
So for any \(N \gamma/2 = l \pi\), provided \(\gamma/2 \ne m \pi\) we have a zero.  We find those at
%
\begin{dmath}\label{eqn:modernOptics:problemSet3:3:380}
N \frac{\pi a}{\lambda} \sin\theta = l \pi,
\end{dmath}
%
or
\begin{dmath}\label{eqn:modernOptics:problemSet3:3:400}
\sin\theta_l = \frac{l \lambda}{N a},
\end{dmath}
%
For the distance between zeros past the center \(\theta = 0\) lobe for a fixed wavelength, we have
%
\begin{dmath}\label{eqn:modernOptics:problemSet3:3:420}
\sin\theta_{l+1} - \sin\theta_l =
\frac{(l+1) \lambda}{N a}
-
\frac{l \lambda}{N a}
=
\frac{\lambda}{N a}.
\end{dmath}
%
If \(\Delta \theta_l\), or just \(\Delta \theta\) (assuming that the peak or zero separation is about the same, although this is artificial in general as we see from the plot), we can compute this by examining the difference
%
\begin{dmath}\label{eqn:modernOptics:problemSet3:3:440}
\sin\theta_{l+1} - \sin\theta_l
\sim
\sin(\theta_l + \Delta \theta/2)
-\sin(\theta_l + \Delta \theta/2)
= 2 \cos \theta_l \sin (\Delta \theta/2)
\sim \Delta \theta \cos\theta_l.
\end{dmath}
%
This gives us the desired relationship (for the \(l\)th zero)
%
\boxedEquation{eqn:modernOptics:problemSet3:3:460}{
\Delta \theta_l \sim \frac{\lambda}{N a \cos\theta_l}.
}

Suppose we rather loosely identify this as the peak width, and look at the image around the \(m\)th peak, as in \cref{fig:modernOpticsProblemSet3Problem3:modernOpticsProblemSet3Problem3Fig4}.  This is about as close as the wavelengths can be in order that a superposition of the two would be distinguishable as separate (humped near center).
%
\imageFigure{../figures/phy485-optics/modernOpticsProblemSet3Problem3Fig4}{Resolvable peak to peak separation.}{fig:modernOpticsProblemSet3Problem3:modernOpticsProblemSet3Problem3Fig4}{0.2}
%
That separation of wavelength \(\lambda' = \lambda + \Delta \lambda\) is
%
\begin{dmath}\label{eqn:modernOptics:problemSet3:3:480}
\Delta \theta
=
\sin^{-1}(m (\lambda + \Delta \lambda)/a)
-
\sin^{-1}(m \lambda/ a)
\sim
\frac{d}{d (m\lambda/a)} \left( \sin^{-1}(m \lambda/ a) \right) \frac{ m \Delta \lambda}{a}
=
\inv{\cos\sin^{-1} (m \lambda/a)}
\frac{ m \Delta \lambda}{a},
\end{dmath}
%
but we also have
\begin{dmath}\label{eqn:modernOptics:problemSet3:3:500}
\Delta \theta
=
\frac{\lambda}{N a \cos\theta}
=
\frac{\lambda}{N a \cos\sin^{-1}(m \lambda/a)}.
\end{dmath}
%
Comparing the two
%
\begin{dmath}\label{eqn:modernOptics:problemSet3:3:520}
\inv{\cancel{\cos\sin^{-1} (m \lambda/a)}}
\frac{ m \Delta \lambda}{\cancel{a}}
=
\frac{\lambda}{N \cancel{a} \cancel{\cos\sin^{-1}(m \lambda/a)}},
\end{dmath}
%
or
%
\boxedEquation{eqn:modernOptics:problemSet3:3:540}{
\Delta \lambda
=
\frac{\lambda}{N m},
}

which is the wavelength resolution desired.

\makeSubAnswer{Angular spread}{modernOptics:problemSet3:3c}

Neglecting the width of the slits, we've found in \eqnref{eqn:modernOptics:problemSet3:3:280a} that our intensity due to a plane wave source and incident light at angle \(\theta_s\), we have
%
\begin{dmath}\label{eqn:modernOptics:problemSet3:3:560}
I \sim
\frac
{\sin^2( \pi a N (\sin\theta - \sin\theta_s)/\lambda )}
{N^2 \sin^2( \pi a k (\sin\theta - \sin\theta_s)/\lambda )}.
\end{dmath}
%
However, with an angular source spread, presumably from a point source at some distance \(L\), we have a configuration like \cref{fig:modernOpticsProblemSet3Problem3:modernOpticsProblemSet3Problem3Fig5}.
%
\imageFigure{../figures/phy485-optics/modernOpticsProblemSet3Problem3Fig5}{Spread source.}{fig:modernOpticsProblemSet3Problem3:modernOpticsProblemSet3Problem3Fig5}{0.2}
%
Our angle of incidence varies with each slit, so this plane wave result doesn't seem applicable.  There's no obvious way to get what we want out of this result, so let's start from scratch.  Let's assume an even number of slits with a symmetric setup, so that our Fraunhofer geometry is
%
\begin{subequations}
\begin{dmath}\label{eqn:modernOptics:problemSet3:3:580}
h_m = \pm \left( m - \inv{2} \right) a.
\end{dmath}
\begin{dmath}\label{eqn:modernOptics:problemSet3:3:600}
\Bk_m = k \frac{ L \zcap \pm h_m \ycap }{\sqrt{ L^2 + h_m^2} } \sim \frac{k h_m}{L} \ycap.
\end{dmath}
\begin{dmath}\label{eqn:modernOptics:problemSet3:3:620}
{\Br'}_m = \ycap h_m.
\end{dmath}
\begin{dmath}\label{eqn:modernOptics:problemSet3:3:640}
\rcap \cdot {\Br'}_m = \pm h_m \sin\theta.
\end{dmath}
\begin{dmath}\label{eqn:modernOptics:problemSet3:3:660}
\Bk_m \cdot {\Br'}_m \sim k (\pm h_m)^2/L.
\end{dmath}
\end{subequations}
%
Summing over both positive and negative \(m\) our Fraunhofer sum becomes
%
\begin{dmath}\label{eqn:modernOptics:problemSet3:3:680}
\Psi
= 2 \sum_{m = 1} e^{i k h_m^2/L} \cos( k h_m \sin\theta )
= 2 \sum_{m = 1} e^{i \frac{k a^2}{L} \left(m - \inv{2}\right)^2} \cos\left( k a \left(m - \inv{2}\right) \sin\theta \right).
\end{dmath}
%
We have two cases to consider.  The first is that our the spread completely covers all the slits, in which case the upper bound of the sum above is \(m = N/2\).  Otherwise, if the slits extend beyond the range of the source spread, we have to sum over an interval where \(m_{\mathrm{max}}\) is the largest integer such that
%
\begin{dmath}\label{eqn:modernOptics:problemSet3:3:700}
\frac{h_{\mathrm{max}}}{L} = a \left( m_{\mathrm{max}} - \inv{2} \right)
\le
\tan \left( \frac{\Delta\theta_s}{2} \right).
\end{dmath}
%
or
%
\begin{dmath}\label{eqn:modernOptics:problemSet3:3:720}
m_{\mathrm{max}} \le \inv{2} + \frac{L}{a}
\tan \left( \frac{\Delta\theta_s}{2} \right)
\sim
\frac{L \Delta \theta_s}{2 a}.
\end{dmath}
%
Let's write \(M\) for this sum where \(M = N/2\) for a source spread that encompasses than the grating, and \(M = m_{\mathrm{max}}\) from \eqnref{eqn:modernOptics:problemSet3:3:720} otherwise.

With \(\gamma = k a \sin\theta/2\), our intensity is
%
\begin{dmath}\label{eqn:modernOptics:problemSet3:3:740}
I
=
\Psi \Psi^\conj
=
\Real \Psi \Psi^\conj
\sim
\Real
\sum_{m,n = 1}^{M}
e^{
i \frac{k a^2}{L}
   \left(
      \left(m - \inv{2}\right)^2
      -\left(n - \inv{2}\right)^2
   \right)
}
\cos\left( 2 \gamma \left(m - \inv{2}\right) \right)
\cos\left( 2 \gamma \left(n - \inv{2}\right) \right)
\sim
\sum_{m,n = 1}^{M}
\cos
\left(
   \frac{k a^2}{L}
   \left(
      \left(m - \inv{2}\right)^2
      -\left(n - \inv{2}\right)^2
   \right)
\right)
(\cos (2 \gamma (m-n)) + \cos ( 2 \gamma (m+n-1)))
=
\sum_{m,n = 1}^{M}
\cos\left(
   \frac{k a^2}{L}
	(m - n) (m + n - 1)
\right)
(\cos (2 \gamma (m - n)) + \cos (2 \gamma (m + n - 1))).
\end{dmath}
%
Do we have any hope whatsoever to evaluate this sum in some sort of closed form?  If we are to try it seems clear that we need two sets of change of variables.  Should we try \(s = m - n\) as in \cref{fig:modernOpticsProblemSet3Problem3:modernOpticsProblemSet3Problem3Fig6}, we find
%
\imageFigure{../figures/phy485-optics/modernOpticsProblemSet3Problem3Fig6}{Points of constant \(m - n\).}{fig:modernOpticsProblemSet3Problem3:modernOpticsProblemSet3Problem3Fig6}{0.2}
%
\begin{equation}\label{eqn:modernOptics:problemSet3:3:760}
\begin{aligned}
\sum_{m, n = 1}^N &f(m - n, m + n - 1) \\
&=
\sum_{s = -N + 1}^{N - 1} \sum_{t = 1}^{N - \Abs{s}} f( s, \Abs{s} + 2 t - 1 ).
\end{aligned}
\end{equation}
%
With the opposite diagonal orientation, as in \cref{fig:modernOpticsProblemSet3Problem3:modernOpticsProblemSet3Problem3Fig7}, we find
%
\imageFigure{../figures/phy485-optics/modernOpticsProblemSet3Problem3Fig7}{Points of constant \(m + n\).}{fig:modernOpticsProblemSet3Problem3:modernOpticsProblemSet3Problem3Fig7}{0.2}
%
\begin{dmath}\label{eqn:modernOptics:problemSet3:3:780}
\sum_{m, n = 1}^N f(m - n, m + n - 1)
=
\sum_{u = 1}^{2 N - 1} \sum_{t = 1}^{N - \Abs{u - N}} f( 2 t - (N - \Abs{u - N}) -1, u)
=
\sum_{u = 1}^{N - 1} \sum_{t = 1}^{u} f( 2 t - u -1, u)
+\sum_{u = N}^{2 N - 1} \sum_{t = 1}^{2 N - u} f( 2 t - (2 N - u) -1, u).
\end{dmath}
%
The intensity can now be written
%
\begin{equation}\label{eqn:modernOptics:problemSet3:3:800}
\begin{aligned}
I
&\sim
\sum_{m,n = 1}^{M}
\cos\left(
   \frac{k a^2}{L}
	(m - n) (m + n - 1)
\right)
\cos (2 \gamma (m - n)) \\
&\quad +
\sum_{m,n = 1}^{M}
\cos\left(
   \frac{k a^2}{L}
	(m - n) (m + n - 1)
\right)
\cos (2 \gamma (m + n - 1)) \\
&=
\sum_{s = -M + 1}^{M - 1} \sum_{t = 1}^{M - \Abs{s}}
\cos\left(
   \frac{k a^2}{L}
	s ( 2 t + \Abs{s} - 1 )
\right)
\cos (2 \gamma s) \\
&\quad +
\sum_{s = 1}^{2 M - 1} \sum_{t = 1}^{M - \Abs{s - M}}
\cos\left(
   \frac{k a^2}{L}
	s ( 2 t - (M - \Abs{s - M}) -1)
\right)
\cos (2 \gamma s)
\end{aligned}
\end{equation}
%
Let's now impose the condition
%
\begin{dmath}\label{eqn:modernOptics:problemSet3:3:820}
k a^2 M/L \sim \frac{\pi a \Delta \theta_s }{\lambda} \ll 1,
\end{dmath}
%
leaving just
%
\begin{dmath}\label{eqn:modernOptics:problemSet3:3:840}
I \sim
\left( \sum_{s = -M + 1}^{M - 1}
+
\sum_{s = 1}^{2 M - 1} \right)
\cos ( 2 \gamma s).
\end{dmath}
%
Using
%
\begin{dmath}\label{eqn:modernOptics:problemSet3:3:860}
\sum_a^b e^{i \alpha m} = e^{i \alpha (b + a)/2} \frac{\sin(\alpha (b - a + 1)/2)}{\sin( \alpha/2)},
\end{dmath}
%
the intensity sums to
%
\begin{dmath}\label{eqn:modernOptics:problemSet3:3:880}
I \sim
(1 + \cos( 2 \gamma M ) )
\frac{\sin(\gamma (2 M - 1))}{\sin( \gamma)}
\sim
(1 + \cos( 2 \gamma M ) )
\frac{\sin(2 \gamma M) }{\sin( \gamma )}.
\end{dmath}
%
Employing half angle formulas and normalizing once more, we get
%
\begin{dmath}\label{eqn:modernOptics:problemSet3:3:900}
I \sim \left( \frac{\cos( \gamma M )}{\gamma M} \right)^2 \frac{\sin(\gamma M) }{ M \sin( \gamma)},
\end{dmath}
%
Since this can be negative, I must have made an error above.  My first attempt on paper had this sine ratio squared and \(2 \gamma M\) instead of \(\gamma M\).  I did find an error in that first attempt, and corrected that here, but in order to make progress, let's ``average the errors'', and assume that the intensity should be
%
\begin{dmath}\label{eqn:modernOptics:problemSet3:3:920}
I \sim \left( \frac{\cos( 2 \gamma M )}{2 \gamma M} \right)^2
\left( \frac{\sin(2 \gamma M) }{ 2 M \sin( \gamma)} \right)^2.
\end{dmath}
%
Specifying that we have more slits than the beam spread so that we use \eqnref{eqn:modernOptics:problemSet3:3:720}, and writing \(\Delta H = L \Delta \theta_s\) for the total illuminated height of the diffraction device we have
%
\begin{dmath}\label{eqn:modernOptics:problemSet3:3:940}
2 \gamma M \sim
2 \frac{2 \pi}{\lambda} \frac{a \sin\theta}{2} \frac{L \Delta \theta_s}{2 a}
=
\frac{\pi L \sin\theta \Delta \theta_s}{\lambda}
=
\frac{\pi \sin\theta \Delta H}{\lambda},
\end{dmath}
%
Let's introduce \(\Delta \theta_a\) for the angular spread of the diffraction regions separating the slits.  Then noting that for the plane wave case where we obtained the wavelength resolution \eqnref{eqn:modernOptics:problemSet3:3:540} from \eqnref{eqn:modernOptics:problemSet3:3:460}, we can make the substitution \(\Delta H \leftrightarrow N a\), to obtain the wavelength resolution for this spread incident beam case (for the \(m = 1\) order peaks)
%
\boxedEquation{eqn:modernOptics:problemSet3:3:960}{
\Delta \lambda = \frac{\Delta \theta_a \lambda}{\Delta \theta_s}.
}

\paragraph{Grading notes (\(-2\))}
\begin{enumerate}
\item The sentence that states ``Let's introduce \(\\Delta \theta_a\)'' was underlined and marked with the question ``What is \(\Delta \theta_a\)?''
\item \(m = 1\) was underlined with question: What about general \(m\)?  I think it was just a scaling by \(m\) for the general case, but when I typed up my solution that didn't strike me as important.
\item \eqnref{eqn:modernOptics:problemSet3:3:960} marked with a question mark and note \(\Delta \lambda \propto \Delta \theta_s\).
\end{enumerate}

The ratio of the angular spread to the separation spread, takes the place of \(N\) in the plane wave case
%
\begin{dmath}\label{eqn:modernOptics:problemSet3:3:980}
\frac{\Delta \theta_s }{\Delta \theta_a} \leftrightarrow N.
\end{dmath}
%
\makeSubAnswer{Using coherence length}{modernOptics:problemSet3:3d}

%Consider two sketches, one with a plane wave source \cref{fig:problemSet3Problem3d:problemSet3Problem3dFig1}, and another with a point source that has some spread \(\Delta \theta_s\) \cref{fig:problemSet3Problem3d:problemSet3Problem3dFig2}.
%
%\imageFigure{../figures/phy485-optics/problemSet3Problem3dFig1}{Plane wave incident light.}{fig:problemSet3Problem3d:problemSet3Problem3dFig1}{0.2}
%\imageFigure{../figures/phy485-optics/problemSet3Problem3dFig2}{Incident light with angular spread.}{fig:problemSet3Problem3d:problemSet3Problem3dFig2}{0.2}
%
%In both of these images the scale of the depicted diffraction slits and the incident wavelength is the same.  With a normal plane wave source, we have coherent diffraction from the first six slits, where the interfering portions of the sources are all within one wavelength of path difference.  In the spread diffraction picture for that illustrated wavelength, we have coherent diffraction for only the first four slits.  The total pathlength on the slit 5 and slit 6 paths both exceed one wavelength, and should reduce the visible fringes.
%\cref{fig:modernOpticsProblemSet3Problem3:modernOpticsProblemSet3Problem3Fig3x}.
%\imageFigure{../figures/phy485-optics/modernOpticsProblemSet3Problem3Fig3x}{CAPTION.}{fig:modernOpticsProblemSet3Problem3:modernOpticsProblemSet3Problem3Fig3x}{0.2}
%
If the transverse coherence length is defined as \(l_{\mathrm{tc}} = \Delta \theta_s/\lambda\), then this has a clear geometric interpretation shown in \cref{fig:modernOpticsProblemSet3Problem3:modernOpticsProblemSet3Problem3Fig8}.
%
\imageFigure{../figures/phy485-optics/modernOpticsProblemSet3Problem3Fig8}{Transverse coherence length geometrically.}{fig:modernOpticsProblemSet3Problem3:modernOpticsProblemSet3Problem3Fig8}{0.2}
%
To match \(N_{\mathrm{eff}}\) to this we write
%
\begin{dmath}\label{eqn:modernOptics:problemSet3:3:1000}
a N_{\mathrm{eff}} = l_{\mathrm{tc}} \Delta \theta_s.
\end{dmath}
%
The angular spread of a single slit is
%
\begin{dmath}\label{eqn:modernOptics:problemSet3:3:1020}
\Delta \theta_a = \frac{a}{l_{\mathrm{tc}}},
\end{dmath}
%
\paragraph{Grading note: (\(-1\))}
\eqnref{eqn:modernOptics:problemSet3:3:1020} was circled with comment ``why?''.

so we have
%
\begin{dmath}\label{eqn:modernOptics:problemSet3:3:1040}
\Delta \lambda
= \frac{\lambda}{ N_{\mathrm{eff}} }
= \frac{a \lambda}{
l_{\mathrm{tc}} \Delta \theta_s
}
=
\frac{\Delta \theta_a \lambda}{
\Delta \theta_s
}.
\end{dmath}
%
Rather remarkably, considering the fudging that was done in \partref{modernOptics:problemSet3:3c}, this matches the wavelength resolution result from \eqnref{eqn:modernOptics:problemSet3:3:960}.
} % makeanswer
