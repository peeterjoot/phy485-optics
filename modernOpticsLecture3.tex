%
% Copyright � 2012 Peeter Joot.  All Rights Reserved.
% Licenced as described in the file LICENSE under the root directory of this GIT repository.
%
%\input{../blogpost.tex}
%\renewcommand{\basename}{modernOpticsLecture3}
%\renewcommand{\dirname}{notes/phy485/}
%\newcommand{\keywords}{Eikonal equation, ray equation, Optics, PHY485H1F}
%\input{../peeter_prologue_print2.tex}
%\beginArtNoToc
%\generatetitle{PHY485H1F Modern Optics.  Lecture 3: GRIN (Graded Refractive INdex).  Taught by Prof.\ Joseph Thywissen}
\label{chap:modernOpticsLecture3}

%\section{Disclaimer}
%
%Peeter's lecture notes from class.  May not be entirely coherent.
%
\section{Gradium Lens.}
\index{gradium lens}

%FIXME: Reading: \S 8 \citep{saleh1991fundamentals}?  Was this reading for
% grin fibers?
\paragraph{Reading}: \S 6.4 \citep{hecht1998hecht}.%  The portion that includes figure 6.42.

We'll use Fermat's theorem
\index{Fermat's theorem}

\makedefinition{Fermat's theorem}{dfn:modernOpticsLecture3:10}{
The pathlength is the same for all rays.
}

Looking to \cref{fig:modernOpticsLecture3:modernOpticsLecture3Fig1} we will now consider a cylindrical lens constructed out of a non-uniform index material.  Suppose we have an index of refraction that is distributed parabolicly
\imageFigure{../figures/phy485-optics/modernOpticsLecture3Fig1}{Gradium lens.}{fig:modernOpticsLecture3:modernOpticsLecture3Fig1}{0.3}
%and \cref{fig:modernOpticsLecture3:modernOpticsLecture3Fig2}
%\imageFigure{../figures/phy485-optics/modernOpticsLecture3Fig2}{Cylindrical coordinates for the problem.}{fig:modernOpticsLecture3:modernOpticsLecture3Fig2}{0.2}
%
\begin{dmath}\label{eqn:modernOpticsLecture3:10}
n(\Brho) = n_0 - \inv{2} \alpha \Brho^2.
\end{dmath}
%
Consider the ray paths that pass through the lens at \(\rho = 0\).  These will have pathlength
%
\begin{dmath}\label{eqn:modernOpticsLecture3:450}
\evalbar{OPL}{\rho = 0} = OPL_{B_1} + OPL_{B_2} = d n_0 + f.
\end{dmath}
%
Note that we assume \(n = 1\) outside of the lens so that portion of the pathlength is \(f\) and not \(n_{\mathrm{outside}} f\).  A ray passing through the lens at \(\rho \ne 0\) will have a pathlength of approximately (neglecting any curvature or ray angle within the lens)
%
\begin{dmath}\label{eqn:modernOpticsLecture3:470}
\evalbar{OPL}{\rho \ne 0} = OPL_{A_1} + OPL_{A_2} = d n(\rho) + \sqrt{f^2 + \rho^2}.
\end{dmath}
%
For these to come together to the same focal point, the two pathlengths must be equal (Fermat's theorem), so we have
%
\begin{dmath}\label{eqn:modernOpticsLecture3:490}
n(\rho) = - \inv{d} \sqrt{f^2 + \rho^2} + n_0 + \frac{f}{d}.
\end{dmath}
%
Assuming a paraxial approximation where \(\rho \ll f\) we have
%
\begin{dmath}\label{eqn:modernOpticsLecture3:510}
n(\rho)
= n_0 + \frac{f}{d} \left( 1 - \sqrt{1 + \left( \frac{\rho}{f} \right)^2} \right)
\sim
n_0 + \frac{f}{d} \left( 1 - \left( 1 + \inv{2} \left( \frac{\rho}{f} \right)^2 \right) \right)
=
n_0 - \inv{2} \frac{\rho^2}{ f d }
\end{dmath}
%
We write
%
\begin{dmath}\label{eqn:modernOpticsLecture3:530}
\alpha = \inv{ 2 f d },
\end{dmath}
%
and seek to solve the Ray equation
%
\begin{dmath}\label{eqn:modernOpticsLecture3:550}
\dds{} \left( n(\rho) \dds{} \left( \Brho + z \zcap \right) \right) = \spacegrad n(\rho).
\end{dmath}
%
With \(\spacegrad n(\rho) \cdot \zcap = 0\), we can consider only the radial portion of this equation
%
\begin{dmath}\label{eqn:modernOpticsLecture3:570}
\dds{} \left( n(\rho) \dds{\Brho} \right) = \spacegrad n(\rho).
=
- \alpha \Brho.
\end{dmath}
%
We seek a relationship as potentially illustrated in \cref{fig:modernOpticsLecture3:modernOpticsLecture3Fig3} where given a paraxial approximation we have
%
\imageFigure{../figures/phy485-optics/modernOpticsLecture3Fig3}{Arc length.}{fig:modernOpticsLecture3:modernOpticsLecture3Fig3}{0.2}
%
\begin{dmath}\label{eqn:modernOpticsLecture3:30}
ds = dz \sqrt{ 1 + \Abs{ \ddz{\rho} }^2} \sim dz,
\end{dmath}
%
and can reduce the Ray equation to
%
\begin{dmath}\label{eqn:modernOpticsLecture3:70}
\ddz{} \left( n_0 - \frac{\alpha}{2} \rho^2 \right) \ddz{\Brho} = - \alpha \Brho.
\end{dmath}
%
We'd like to drop the \(\rho^2\) term above, and can do that provided
%
\begin{dmath}\label{eqn:modernOpticsLecture3:590}
\rho \ll \sqrt{ \frac{2 n_0}{ \alpha} } = 2 \sqrt{ n_0 f d }.
\end{dmath}
%
This leaves us with a plain old SHO

\boxedEquation{eqn:modernOpticsLecture3:90}{
n_0 \frac{d^2}{dz^2} \Brho = - \alpha \Brho
}

Let's write this out
%
\begin{dmath}\label{eqn:modernOpticsLecture3:110}
\Brho(z) =
\begin{bmatrix}
x(z) \\
y(z) \\
\end{bmatrix}
=
\begin{bmatrix}
A \cos \left( \sqrt{\frac{\alpha}{n_0}} z \right) + B \sin\left( \sqrt{\frac{\alpha}{n_0}} z \right) \\
C \cos \left( \sqrt{\frac{\alpha}{n_0}} z \right) + D \sin\left( \sqrt{\frac{\alpha}{n_0}} z \right) \\
\end{bmatrix}
\end{dmath}
%
We have 4 constants determined by the initial conditions
%
\begin{dmath}\label{eqn:modernOpticsLecture3:130}
\Brho(0) =
\begin{bmatrix}
A \\
C
\end{bmatrix}
\end{dmath}
%
\begin{dmath}\label{eqn:modernOpticsLecture3:150}
\evalbar{\ddz{\Brho}}{0} =
\sqrt{\frac{\alpha}{n_0}}
\begin{bmatrix}
B \\
D
\end{bmatrix}.
\end{dmath}
%
If we set \(\Brho(0) = 0\), then we are left with just
%
\begin{dmath}\label{eqn:modernOpticsLecture3:610}
\Brho(z) =
\begin{bmatrix}
B \\
D
\end{bmatrix}
\sin\left( \sqrt{\frac{\alpha}{n_0}} z \right).
\end{dmath}
%
This vectoral solution \(\Br = \Brho + z \zcap + \Bz_0\) (with \(\Bz_0 = 0\)) is illustrated in \cref{fig:modernOpticsLecture3:modernOpticsLecture3Fig4r}.
\imageFigure{../figures/phy485-optics/modernOpticsLecture3Fig4r}{Ray variation with position through GRIN material.}{fig:modernOpticsLecture3:modernOpticsLecture3Fig4r}{0.3}
%
Observe that the nodes are placed at \(\sqrt{\alpha/n_0} = n \pi\) with a complete cycle in distance as illustrated in \cref{fig:modernOpticsLecture3:modernOpticsLecture3Fig4}.
%
\begin{dmath}\label{eqn:modernOpticsLecture3:630}
L = 2 \pi \sqrt{ \frac{n_0}{\alpha} }.
\end{dmath}
%
\imageFigure{../figures/phy485-optics/modernOpticsLecture3Fig4}{Nodal distribution.}{fig:modernOpticsLecture3:modernOpticsLecture3Fig4}{0.2}
%
If the length is an integer \(L\) as in \cref{fig:modernOpticsLecture3:modernOpticsLecture3Fig5}.
\imageFigure{../figures/phy485-optics/modernOpticsLecture3Fig5}{first order solution.}{fig:modernOpticsLecture3:modernOpticsLecture3Fig5}{0.2}
%
If the length is a half integer \(L\) as in \cref{fig:modernOpticsLecture3:modernOpticsLecture3Fig6}.
\imageFigure{../figures/phy485-optics/modernOpticsLecture3Fig6}{second order solution.}{fig:modernOpticsLecture3:modernOpticsLecture3Fig6}{0.2}
%
If \(\text{length} = \left(n + \inv{4} \right)L\) as in \cref{fig:modernOpticsLecture3:modernOpticsLecture3Fig7}.
\imageFigure{../figures/phy485-optics/modernOpticsLecture3Fig7}{third order solution.}{fig:modernOpticsLecture3:modernOpticsLecture3Fig7}{0.2}
%
It was mentioned that this solved a problem with regular fiber optic cables illustrated in \cref{fig:modernOpticsLecture3:modernOpticsLecture3Fig8}
%
\imageFigure{../figures/phy485-optics/modernOpticsLecture3Fig8}{regular fiber effects.}{fig:modernOpticsLecture3:modernOpticsLecture3Fig8}{0.2}
%
so that for the GRIN configuration we have something more like \cref{fig:modernOpticsLecture3:modernOpticsLecture3Fig9}.
\imageFigure{../figures/phy485-optics/modernOpticsLecture3Fig9}{step index fiber.}{fig:modernOpticsLecture3:modernOpticsLecture3Fig9}{0.2}
%
This phenomena is well described in \S 5.6.1 \citep{hecht1998hecht}.

\subsection{Phase delay in GRIN lens?}
\index{graded refractive index!phase delay}

Are path lengths equal?  Instead of dropping all but the \(dz\) term in our \(ds\) approximation \eqnref{eqn:modernOpticsLecture3:30}, how about we retain the first order Taylor expansion
%
\begin{dmath}\label{eqn:modernOpticsLecture3:170}
\tau
= \int_0^L \frac{ds}{c/n(\Br)}
= \inv{c} \int_0^L dz \sqrt{ 1 + \Abs{\frac{d\rho}{dz}}^2 } \left( n_0 - \frac{\alpha}{2} \rho^2 \right)
\approx \frac{n_0}{c} \int_0^L dz \left( 1 + \inv{2} \Abs{ \ddz{\Brho}}^2 \right) \left(1 - \frac{\alpha}{2 n_0} \rho^2 \right)
\approx \frac{n_0}{c} \int_0^L dz \left( 1 + \inv{2} \Abs{ \ddz{\Brho} }^2 - \frac{\alpha}{2 n_0} \Brho^2 \right) + O(\text{higher order corrections}).
\end{dmath}
%
But
\begin{dmath}\label{eqn:modernOpticsLecture3:190}
\ddz{} \left( \Brho \cdot \ddz{\Brho} \right)
\approx
\Abs{ \ddz{\Brho} }^2 + \Brho \cdot \frac{d^2 \Brho }{d^2 z},
\end{dmath}
%
so our approximate path length can be written
%
\begin{dmath}\label{eqn:modernOpticsLecture3:210}
\tau
= \frac{n_0}{c} \int_0^L dz
\left( 1
       + \inv{2} \ddz{} \left( \Brho \cdot \ddz{\Brho}
\right)
       - \inv{2} \rho \cdot \frac{d^2 \Brho }{d^2 z}
       -\inv{2} \frac{\alpha}{n_0} \rho^2
\right)
\end{dmath}
%
But
\begin{dmath}\label{eqn:modernOpticsLecture3:230}
- \inv{2} \Brho \cdot \frac{d^2 \Brho }{d^2 z}
-\inv{2} \frac{\alpha}{n_0} \Brho^2
=
-\inv{2} \mathLabelBox{\left( \frac{d^2}{dz^2} \Brho - \frac{\alpha}{n_0} \Brho \right)}{\(= 0\) by Eikonal} \cdot \Brho
\end{dmath}
%
so
\begin{dmath}\label{eqn:modernOpticsLecture3:250}
\tau
= \frac{n_0 L}{c}
+ \frac{n_0}{2 c} \mathLabelBox{\evalrange{ \Brho \cdot \ddz{\Brho} }{0}{L}}{\(= 0\) if refocused}
\end{dmath}
%
Here if refocused means \(\Brho = 0\) at both sides, as we had for \(L = 2 \pi \sqrt{n_0/\alpha}\).  The conclusion is that when light traverses focal point to focal point within the GRIN material of this sort, it propagates without any sort of phase delay.

\section{Ray equation and action minimization.}
\index{ray equation}
\index{action minimization}

\paragraph{Reading}: \S 3 of \citep{born1980principles} for details on this topic.

Ray equation gives paths of stationary action

In general our action is
%
\begin{dmath}\label{eqn:modernOpticsLecture3:270}
S = \int_{t_1}^{t_2} dt \LL
\end{dmath}
%
where \(\LL\) is the Lagrangian.  We recall Hamilton's principle which states that if
%
\begin{dmath}\label{eqn:modernOpticsLecture3:290}
\delta S = 0,
\end{dmath}
%
(a path variation as illustrated in \cref{fig:modernOpticsLecture3:modernOpticsLecture3Fig10}
%
\imageFigure{../figures/phy485-optics/modernOpticsLecture3Fig10}{Action minimization.}{fig:modernOpticsLecture3:modernOpticsLecture3Fig10}{0.2}
%
), then the statement \eqnref{eqn:modernOpticsLecture3:290} gives us Hamiltonian dynamics (Hamilton's equations).

Why does this work?  One explanation is that we have in quantum mechanics the most general action
%
\begin{dmath}\label{eqn:modernOpticsLecture3:410}
\text{amplitude} (\text{path}) = \exp( i S[\text{path}]/\Hbar ),
\end{dmath}
%
This is called Feynman's Path Integral.

We have a Lagrangian for electromagnetism
%
\begin{dmath}\label{eqn:modernOpticsLecture3:430}
\LL_{\mathrm{EM}} = \inv{c} \rho \phi + \inv{c} \Bj \cdot \BA + \inv{8 \pi} \BE^2 - \inv{8 \pi} \BB^2.
\end{dmath}
%
Using this we can derive Maxwell's equations.

As a problem we are going to calculate the amplitude for the Cornu spiral \cref{fig:modernOpticsLecture3:modernOpticsLecture3Fig11}.
%
\imageFigure{../figures/phy485-optics/modernOpticsLecture3Fig11}{Cornu spiral path of interest.}{fig:modernOpticsLecture3:modernOpticsLecture3Fig11}{0.2}
%
Propose some experiments

1) Block the primary path \cref{fig:modernOpticsLecture3:modernOpticsLecture3Fig12}.
%
\imageFigure{../figures/phy485-optics/modernOpticsLecture3Fig12}{Single slit.}{fig:modernOpticsLecture3:modernOpticsLecture3Fig12}{0.2}
%
2) Block 2 paths \cref{fig:modernOpticsLecture3:modernOpticsLecture3Fig13}.
%
\imageFigure{../figures/phy485-optics/modernOpticsLecture3Fig13}{Double slit.}{fig:modernOpticsLecture3:modernOpticsLecture3Fig13}{0.2}
%
3) Block half paths \cref{fig:modernOpticsLecture3:modernOpticsLecture3Fig14}.
%
\imageFigure{../figures/phy485-optics/modernOpticsLecture3Fig14}{Wall blocking half path.}{fig:modernOpticsLecture3:modernOpticsLecture3Fig14}{0.2}
%
4) Allow paths with the same phase.

\Cref{fig:modernOpticsLecture3:modernOpticsLecture3Fig15}.
\imageFigure{../figures/phy485-optics/modernOpticsLecture3Fig15}{Many slits.}{fig:modernOpticsLecture3:modernOpticsLecture3Fig15}{0.2}
%
We'll see that the Action principle does in fact provide us all the real physical effects (Fresnel diffraction, multi-slit diffraction, ...)
%\vcsinfo
%\EndArticle
