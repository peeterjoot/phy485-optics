%
% Copyright � 2012 Peeter Joot.  All Rights Reserved.
% Licenced as described in the file LICENSE under the root directory of this GIT repository.
%
%\input{../blogpost.tex}
%\renewcommand{\basename}{modernOpticsLecture9}
%\renewcommand{\dirname}{notes/phy485/}
%\newcommand{\keywords}{Optics, PHY485H1F}
%\input{../peeter_prologue_print2.tex}
%\beginArtNoToc
%\generatetitle{PHY485H1F Modern Optics.  Lecture 9: Mathematical treatment of coherence.  Taught by Prof.\ Joseph Thywissen}
%%\chapter{Mathematical treatment of coherence}
\index{coherence}
%\label{chap:modernOpticsLecture9}
%
%\section{Disclaimer}
%
%Peeter's lecture notes from class.  May not be entirely coherent.
%
\index{coherence!mutual}

Observing intensity in a two-path interferometer
%
\begin{dmath}\label{eqn:modernOpticsLecture9:10}
I = I_1 + I_2 + 2 \Real \Gamma_{12}.
\end{dmath}
%
where
\begin{subequations}
\begin{dmath}\label{eqn:modernOpticsLecture9:30}
I_1 = \expectation{ \Psi_1 \Psi_1^\conj }.
\end{dmath}
\begin{dmath}\label{eqn:modernOpticsLecture9:50}
I_2 = \expectation{ \Psi_2 \Psi_2^\conj }.
\end{dmath}
\begin{dmath}\label{eqn:modernOpticsLecture9:70}
\Gamma_{12}  = \expectation{ \Psi_1 \Psi_2^\conj }.
\end{dmath}
\end{subequations}
%
Here \(\Gamma_{12}\) is the mutual coherence and \(\expectation{...}\) indicates the time average.

We'll consider a Michaelson interferometer setup as illustrated in \cref{fig:modernOpticsLecture9:modernOpticsLecture9Fig1}.
%
\imageFigure{../figures/phy485-optics/modernOpticsLecture9Fig1}{Michaelson.}{fig:modernOpticsLecture9:modernOpticsLecture9Fig1}{0.3}
%
The net effect of this is as if a phase delay in a linear system had been introduced, as illustrated in \cref{fig:modernOpticsLecture9:modernOpticsLecture9Fig2}.
%
\imageFigure{../figures/phy485-optics/modernOpticsLecture9Fig2}{Equivalent to Michaelson.}{fig:modernOpticsLecture9:modernOpticsLecture9Fig2}{0.2}
%
With
%
\begin{subequations}
\begin{dmath}\label{eqn:modernOpticsLecture9:90}
\Psi(\Br_k, t) = \sqrt{I(\Br_k)} \exp\left( i \phi(\Br_k, t) \right).
\end{dmath}
\begin{dmath}\label{eqn:modernOpticsLecture9:110}
\Psi_{1,2} = \Psi(\Br_{1,2}, t).
\end{dmath}
\end{subequations}
%
We work with \(\Abs{\Br} \gg \lambda\), so that we neglect any small intensity change, and make the approximation
%
\begin{dmath}\label{eqn:modernOpticsLecture9:130}
I(\Br_1) \sim I(\Br_2).
\end{dmath}
%
and define

\boxedEquation{eqn:modernOpticsLecture9:150}{
\tau = \frac{s_2 - s_1}{c}
}

We write
%
\begin{dmath}\label{eqn:modernOpticsLecture9:170}
\gamma_{12} = \frac{\Gamma_{12}}{\sqrt{I_1} \sqrt{I_2}}.
\end{dmath}
%
So that our total intensity is just
%
\begin{dmath}\label{eqn:modernOpticsLecture9:190}
I = 2 I_1 ( 1 + \Real \gamma_{12} ).
\end{dmath}
%
Our task is to calculate
%
\begin{dmath}\label{eqn:modernOpticsLecture9:210}
\gamma_{12} = \expectation{ e^{i\phi_1 - i\phi_2} }.
\end{dmath}
%
With
%
\begin{dmath}\label{eqn:modernOpticsLecture9:230}
\phi(t) = \omega t + \Delta(t),
\end{dmath}
%
we suppose that we have some sort of system, perhaps due to atomic interactions, we have random discrete phase jumps at regular intervals, as in \cref{fig:modernOpticsLecture9:modernOpticsLecture9Fig3} and \cref{fig:modernOpticsLecture9:modernOpticsLecture9Fig4}.
\imageFigure{../figures/phy485-optics/modernOpticsLecture9Fig3}{Random step phase changes.}{fig:modernOpticsLecture9:modernOpticsLecture9Fig3}{0.3}
\imageFigure{../figures/phy485-optics/modernOpticsLecture9Fig4}{Effect of random phase changes.}{fig:modernOpticsLecture9:modernOpticsLecture9Fig4}{0.3}
%
This isn't necessarily a realistic system, but it one that we can calculate.
%
\begin{dmath}\label{eqn:modernOpticsLecture9:250}
\gamma_{12}
= \expectation{ e^{i \phi(t) } e^{ -i\phi(t + \tau) } }
= e^{i \omega \tau} \lim_{T \rightarrow \infty} \int_0^T e^{i \Delta(t) - i\Delta(t + \tau)} dt.
\end{dmath}
%
With the illustrated example phase transitions above we have \cref{fig:modernOpticsLecture9:modernOpticsLecture9Fig5}.
\imageFigure{../figures/phy485-optics/modernOpticsLecture9Fig5}{Differences of random phases after time delay.}{fig:modernOpticsLecture9:modernOpticsLecture9Fig5}{0.3}
%
Integrating across one time interval and then summing we have
%
\begin{dmath}\label{eqn:modernOpticsLecture9:270}
\inv{N \tau_0} \sum_{n = 1}^{N} \int_{n \tau_0}^{(n+1) \tau_0} e^{i (\Delta(t) - \Delta(t+\tau)} dt
=
\frac{\sum{n=1}^{N}}{N \tau_0} (\tau_0 - \tau)
+\cancel{\frac{\sum_{n=1}^{N}}{N \tau_0} \tau e^{-i\Delta_i}}.
\end{dmath}
%
To account for the cancellation, note that we are summing over a number of complex numbers, like
%
\begin{dmath}\label{eqn:modernOpticsLecture9:410}
e^{i\Delta_1}
+e^{i\Delta_2}
+e^{i\Delta_3}.
\end{dmath}
%
where the \(\Delta\)'s are random.  This is illustrated in \cref{fig:modernOpticsLecture9:modernOpticsLecture9Fig7}.
\imageFigure{../figures/phy485-optics/modernOpticsLecture9Fig7}{Random walk evolution.}{fig:modernOpticsLecture9:modernOpticsLecture9Fig7}{0.3}
%
(we do get somewhere in a random walk, but it is approximately \(\sqrt{N}\) on average, so we have \(\sqrt{N}/N\) in the sum which goes to zero).
%
%\fxwarning{exercise, random (directional change) walk}{Try such a random summation as a problem.}
%
Putting results together we have
%
\begin{dmath}\label{eqn:modernOpticsLecture9:290}
\gamma_{12} =
\left\{
\begin{array}{l l}
\left( 1 - \frac{\Abs{\tau}}{\tau_0} \right) e^{-i\omega \tau}
& \quad \mbox{if \(\tau \le \tau_0\)} \\
0 & \quad \mbox{if \(\tau \ge \tau_0\)} \\
\end{array}
\right.
\end{dmath}
Also see \S 3.5
%(\fxwarning{reading}{check section number})
in \citep{fowles1989introduction} for this derivation.

The intensity output of the interferometer is
%
\begin{dmath}\label{eqn:modernOpticsLecture9:310}
I
= 2 I_0 + 2 I_0 \Real \gamma_{12}
= 2 I_0 + 2 I_0 \Abs{\gamma_{12}} \cos(\omega \tau)
= 2 I_0 + 2 I_0 \cos(\omega \tau)
\left( 1 - \frac{\tau}{\tau_0} \right) e^{-i\omega \tau}
\left\{
\begin{array}{l l}
1 - \Abs{\tau}{\tau_0}
& \quad \mbox{\(\tau \le \tau_0\)} \\
0 & \quad \mbox{\(\tau \ge \tau_0\)} \\
\end{array}
\right.
\end{dmath}
%
as in \cref{fig:modernOpticsLecture9:modernOpticsLecture9Fig6}.
\imageFigure{../figures/phy485-optics/modernOpticsLecture9Fig6}{Resulting interference intensity.}{fig:modernOpticsLecture9:modernOpticsLecture9Fig6}{0.3}
%
One fringe is at
%
\begin{dmath}\label{eqn:modernOpticsLecture9:330}
\omega \tau = 2 \pi,
\end{dmath}
%
or
\begin{dmath}\label{eqn:modernOpticsLecture9:350}
\omega \frac{s_2 - s_1}{c} = 2 \pi.
\end{dmath}
%
With
%
\begin{dmath}\label{eqn:modernOpticsLecture9:370}
2 \pi \frac{c }{\omega} = \frac{\omega \lambda}{\omega} = \lambda.
\end{dmath}
%
we have
%
\begin{dmath}\label{eqn:modernOpticsLecture9:390}
s_2 - s_1 = \lambda.
\end{dmath}
%
typically require \(500\) nm for visible light.  We need micron scale control.

\section{More general mutual coherence.}

What do we know about \(\expectation{ \Psi_1 \Psi_2^\conj }\) ?
%
\begin{dmath}\label{eqn:modernOpticsLecture9:450}
\expectation{ \Psi_1(t) \Psi_2^\conj(t + \tau) }
=
\lim_{T \rightarrow \infty} \int_0^T dt \Psi_1(t) \Psi_2(t + \tau).
\end{dmath}
%
This (integral) is just a convolution, so we can compute this by performing Fourier transforms and inverse Fourier transforms.  If
%
\begin{dmath}\label{eqn:modernOpticsLecture9:470}
f(x) = g * h = \int_{-\infty}^\infty dx' g(x') h(x - x').
\end{dmath}
%
Then
\begin{dmath}\label{eqn:modernOpticsLecture9:490}
F(k) = G(k) H(k),
\end{dmath}
%
so that
%
\begin{dmath}\label{eqn:modernOpticsLecture9:510}
f(x) = \calF^{-1} G(k) H(k).
\end{dmath}
%
We see that
\(\gamma_{12}(\tau)\)
\textunderline{is a Fourier transform of power spectrum of the source}.  Explicitly, that is
%
\begin{dmath}\label{eqn:modernOpticsLecture9:530}
\Abs{\Gamma_{12}} = e^{i \alpha_{12}(\tau)} = 4
\int_0^\infty G_{12}(\omega) e^{-i(\omega - \overbar{\omega}) \tau } d\omega.
\end{dmath}
%
where
%
\begin{dmath}\label{eqn:modernOpticsLecture9:550}
\Gamma_{12} = \Abs{\Gamma_{12}} e^{i \alpha_{12}(\tau) - i \omega \tau }.
\end{dmath}
%
and
%
\begin{subequations}
\begin{dmath}\label{eqn:modernOpticsLecture9:570}
G_{12}(\omega) = \lim_{T \rightarrow \infty} \inv{2 T}
V_T(\Br_1, \omega)
V_T^\conj(\Br_2, \omega)
\sim V_1(\omega) V_2^\conj(\omega).
\end{dmath}
\begin{dmath}\label{eqn:modernOpticsLecture9:590}
V(\omega) = \calF(\Psi).
\end{dmath}
\begin{dmath}\label{eqn:modernOpticsLecture9:610}
V_T = \int_{-\infty}^\infty \Psi_T^{(r)} (\Br, t) e^{i \omega t}.
\end{dmath}
\begin{dmath}\label{eqn:modernOpticsLecture9:630}
\Psi_T^{(r)} =
\left\{
\begin{array}{l l}
\Real \Psi & \quad \mbox{\(\Abs{t} \le T\)} \\
0 & \quad \mbox{\(\Abs{t} \ge T\)} \\
\end{array}
\right.
\end{dmath}
\end{subequations}
%
Reading: These results weren't derived here.  For that see \S 10.3.2 \citep{born1980principles}.

\paragraph{Example. Gaussian}
\index{Gaussian!power spectrum}
\index{Gaussian!correlation}

%\cref{fig:modernOpticsLecture9:modernOpticsLecture9Fig8}.
\imageFigure{../figures/phy485-optics/modernOpticsLecture9Fig8}{Gaussian power spectrum and correlation.}{fig:modernOpticsLecture9:modernOpticsLecture9Fig8}{0.2}
%
(Fourier transform of a Gaussian is a Gaussian)

\paragraph{Example. Lorenzian}
\index{Lorenzian}

%\cref{fig:modernOpticsLecture9:modernOpticsLecture9Fig9}.
\imageFigure{../figures/phy485-optics/modernOpticsLecture9Fig9}{Lorentzian power spectrum and correlation.}{fig:modernOpticsLecture9:modernOpticsLecture9Fig9}{0.3}
%
%\vcsinfo
%\EndArticle
