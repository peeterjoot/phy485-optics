%
% Copyright � 2012 Peeter Joot.  All Rights Reserved.
% Licenced as described in the file LICENSE under the root directory of this GIT repository.
%
%\input{../blogpost.tex}
%\renewcommand{\basename}{modernOpticsLecture8}
%\renewcommand{\dirname}{notes/phy485/}
%\newcommand{\keywords}{Optics, PHY485H1F}
%\input{../peeter_prologue_print2.tex}
%\beginArtNoToc
%\generatetitle{PHY485H1F Modern Optics.  Lecture 8: Coherence (cont.).  Taught by Prof.\ Joseph Thywissen}
%%\chapter{Coherence (cont.)}
%\label{chap:modernOpticsLecture8}
%
%\section{Disclaimer}
%
%Peeter's lecture notes from class.  May not be entirely coherent.
%
\section{Zoology of interferometers.}
\index{interferometer}

\index{coherence!defined}
\makedefinition{Coherence}{dfn:modernOpticsLecture8:1}{(Operational definition)  Something measured by an interferometer}

\paragraph{Types of dual path interferometers}
\index{interferometer!dual path}

Some types of single path interferometers

\index{interferometer!Young's}
\index{interferometer!Michaelson's}
\index{interferometer!Fresnel Biprism}
\index{interferometer!Lloyd's mirror}
\index{interferometer!Mach-Zender}

\begin{itemize}
\item Young's \cref{fig:modernOpticsLecture8:modernOpticsLecture8Fig1}.
\item Michaelson's \cref{fig:modernOpticsLecture8:modernOpticsLecture8Fig2}.
\item Fresnel Biprism \cref{fig:modernOpticsLecture8:modernOpticsLecture8Fig3}.
\item Lloyd's mirror \cref{fig:modernOpticsLecture8:modernOpticsLecture8Fig4}.
\item Mach-Zender \cref{fig:modernOpticsLecture8:modernOpticsLecture8Fig5}.
\end{itemize}
%
\imageFigure{../figures/phy485-optics/modernOpticsLecture8Fig1}{Wavefront splitting.  Young's interferometer.}{fig:modernOpticsLecture8:modernOpticsLecture8Fig1}{0.2}
\imageFigure{../figures/phy485-optics/modernOpticsLecture8Fig2}{Amplitude splitting.  Michaelson's interferometer.}{fig:modernOpticsLecture8:modernOpticsLecture8Fig2}{0.2}
\imageFigure{../figures/phy485-optics/modernOpticsLecture8Fig3}{Fresnel Biprism (wavefront splitter).}{fig:modernOpticsLecture8:modernOpticsLecture8Fig3}{0.2}
\imageFigure{../figures/phy485-optics/modernOpticsLecture8Fig4}{Lloyd's mirror.  Interference from different path lengths.}{fig:modernOpticsLecture8:modernOpticsLecture8Fig4}{0.2}
\imageFigure{../figures/phy485-optics/modernOpticsLecture8Fig5}{Mach-Zender interferometer.  Temporal fringe if moving mirror.}{fig:modernOpticsLecture8:modernOpticsLecture8Fig5}{0.2}
%
\paragraph{Types of multi-path interferometers}
\index{interferometer!multi-path}

Some types of multiple path interferometers

\begin{itemize}
\item Wavefront splitting \cref{fig:modernOpticsLecture8:modernOpticsLecture8Fig6}.
\item Infinite reflection in multiple mirrors (accidental) \cref{fig:modernOpticsLecture8:modernOpticsLecture8Fig7}.
\item Infinite reflection in multiple mirrors (Fabry-Perot Cavity) \cref{fig:modernOpticsLecture8:modernOpticsLecture8Fig8}.
\end{itemize}
\imageFigure{../figures/phy485-optics/modernOpticsLecture8Fig6}{Wavefront splitting.}{fig:modernOpticsLecture8:modernOpticsLecture8Fig6}{0.2}
\imageFigure{../figures/phy485-optics/modernOpticsLecture8Fig7}{Bathroom cabinet setup, with reflection within reflection within ....}{fig:modernOpticsLecture8:modernOpticsLecture8Fig7}{0.2}
\imageFigure{../figures/phy485-optics/modernOpticsLecture8Fig8}{Fabry-Perot Cavity (repeated reflection on purpose).}{fig:modernOpticsLecture8:modernOpticsLecture8Fig8}{0.2}
%
\section{Lloyd's interferometer.}
\index{interferometer!Lloyd's}

Using a virtual ray we can think of the Lloyd's interferometer setup as equivalent to a Young's double slit setup as illustrated in \cref{fig:modernOpticsLecture8:modernOpticsLecture8Fig9} and
\cref{fig:modernOpticsLecture8:modernOpticsLecture8Fig10}.
%
\imageFigure{../figures/phy485-optics/modernOpticsLecture8Fig9}{Virtual beam with mirror.}{fig:modernOpticsLecture8:modernOpticsLecture8Fig9}{0.2}
\imageFigure{../figures/phy485-optics/modernOpticsLecture8Fig10}{Can think of virtual beam as diffraction source.}{fig:modernOpticsLecture8:modernOpticsLecture8Fig10}{0.2}
%
Consider two sources as in \cref{fig:modernOpticsLecture8:modernOpticsLecture8Fig11}.
\imageFigure{../figures/phy485-optics/modernOpticsLecture8Fig11}{Extended source.}{fig:modernOpticsLecture8:modernOpticsLecture8Fig11}{0.2}
%
Looking at this mathematically we have
%
\begin{equation}\label{eqn:modernOpticsLecture8:10}
\begin{aligned}
I
&= \expectation{ \Abs{\Psi}^2 } \\
&= \expectation{ \Abs{\Psi(\Br_1, t) + \Psi(\Br_2, t) }^2 } \\
&= I(\Br_1) + I(\Br_2) + 2 \Real \expectation{ \Psi(\Br_1, t) \Psi^\conj(\Br_2, t) }.
\end{aligned}
\end{equation}
%
All the action is in the cross term.  The portion of this that is hard to calculate, we call the \underlineAndIndex{Mutual coherence}
%
\begin{equation}\label{eqn:modernOpticsLecture8:30}
\Gamma_{12} \equiv
\expectation{ \Psi(\Br_1, t) \Psi^\conj(\Br_2, t) }.
\end{equation}
%
\section{Types of coherence.}
\index{coherence}

\subsection{Longitudinal coherence.}
\index{coherence!longitudinal}

Consider the measurement of the relative interference at two points as in \cref{fig:modernOpticsLecture8:modernOpticsLecture8Fig12}.
%
\imageFigure{../figures/phy485-optics/modernOpticsLecture8Fig12}{Multiple paths along one ray direction.}{fig:modernOpticsLecture8:modernOpticsLecture8Fig12}{0.2}
%
where we have a device that measures the relative interference at these points as in \cref{fig:modernOpticsLecture8:modernOpticsLecture8Fig13} and \cref{fig:modernOpticsLecture8:modernOpticsLecture8Fig14}.
%
\imageFigure{../figures/phy485-optics/modernOpticsLecture8Fig13}{Imagine exaggerated refraction and reflection from cavity at end of ray.}{fig:modernOpticsLecture8:modernOpticsLecture8Fig13}{0.2}
\imageFigure{../figures/phy485-optics/modernOpticsLecture8Fig14}{But with cavity aligned.}{fig:modernOpticsLecture8:modernOpticsLecture8Fig14}{0.2}
%
where we suppose that there's something that has introduced a small amount of delay or path length.  The extra pathlength like a time delay
%
\begin{equation}\label{eqn:modernOpticsLecture8:50}
\tau = \frac{s_2 - s_1}{c}.
\end{equation}
%
With a \underlineAndIndex{coherence time} defined as
%
\begin{equation}\label{eqn:modernOpticsLecture8:70}
\tau_{\mathrm{coh}} = \inv{\Delta w}.
\end{equation}
%
where \(\Delta w\) is the spectral width of the source.

We will show that if
%
\begin{equation}\label{eqn:modernOpticsLecture8:90}
s_2 - s_1 \ll c \tau_{\mathrm{coh}}.
\end{equation}
%
we have good visibility.

We want to think about what happens when the source gets broad as in \cref{fig:modernOpticsLecture8:modernOpticsLecture8Fig16}.
%
\imageFigure{../figures/phy485-optics/modernOpticsLecture8Fig16}{Power distribution with interference due to extended source.}{fig:modernOpticsLecture8:modernOpticsLecture8Fig16}{0.2}
%
\subsection{Transverse coherence.}
\index{coherence!transverse}

As illustrated in \cref{fig:modernOpticsLecture8:modernOpticsLecture8Fig15}.
%
\imageFigure{../figures/phy485-optics/modernOpticsLecture8Fig15}{Interference from extended source.}{fig:modernOpticsLecture8:modernOpticsLecture8Fig15}{0.2}
%
with
%
\begin{equation}\label{eqn:modernOpticsLecture8:110}
\text{length} = \frac{\lambda}{\Delta \theta_s}.
\end{equation}
%
we will show that we get a good fringe if
%
\begin{equation}\label{eqn:modernOpticsLecture8:130}
x \ll
\frac{\lambda}{\Delta \theta_s}.
\end{equation}
%
A \textunderline{point source} is one for which \(\Delta \theta_s \rightarrow 0\), so that \(\lambda/\Delta \theta_s \rightarrow \infty\).

We want to think about what happens when the source gets big.

\paragraph{Doesn't the intensity loss in the \(P_1,P_2\) linear interference setup matter?}

Consider \cref{fig:modernOpticsLecture8:modernOpticsLecture8Fig19}.
%
\imageFigure{../figures/phy485-optics/modernOpticsLecture8Fig19}{Intensity differences after cavity reflection.}{fig:modernOpticsLecture8:modernOpticsLecture8Fig19}{0.2}
%
If \(R\) is small, then the resulting intensities are similar.

% place for these:
% FIXME: ???
%\cref{fig:modernOpticsLecture8:modernOpticsLecture8Fig17}.
%\imageFigure{../figures/phy485-optics/modernOpticsLecture8Fig17}{Extended source with phase interference.}{fig:modernOpticsLecture8:modernOpticsLecture8Fig17}{0.2}
%\cref{fig:modernOpticsLecture8:modernOpticsLecture8Fig17b}.
%\imageFigure{../figures/phy485-optics/modernOpticsLecture8Fig17b}{Extended source with intensity addition and overall reduced visibility.}{fig:modernOpticsLecture8:modernOpticsLecture8Fig17b}{0.2}
%
%\vcsinfo
%\EndArticle
%\EndNoBibArticle
