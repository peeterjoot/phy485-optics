%
% Copyright � 2013 Peeter Joot.  All Rights Reserved.
% Licenced as described in the file LICENSE under the root directory of this GIT repository.
%
\makeoproblem
%{Comparing the M\"obius transform to sequential matrix transforms}
{M\"obius transform vs. matrix transformations.}
{modernOptics:problemSet4:4}
{2012 Ps4, P4}
{
Using the M\"obius transform \(q' = (A q + B)/(C q + D)\), show that transformation using
\(\{A_1, B_1, C_1, D_1 \}\) followed by
\(\{A_2, B_2, C_2, D_2 \}\) is equivalent to transformation using the elements of the single matrix
\(M = M_2 M_1\), where
\[
M_1 = \left( \begin{array}{cc} A_1 & B_1 \\ C_1 & D_1  \end{array} \right)
\quad  \mbox{and} \quad
M_2 = \left( \begin{array}{cc} A_2 & B_2 \\ C_2 & D_2  \end{array} \right) .
\]
} % makeoproblem

\makeanswer{modernOptics:problemSet4:4}{

Proceeding directly with the double application of the M\"obius transform, we have

\begin{dmath}\label{eqn:problemSet4Problem4:10}
q''
= \frac{A_2 q' + B_2}{C_2 q' + D_2}
= \frac{A_2 \left( \frac{A_1 q + B_1}{C_1 q + D_1} \right) + B_2}{C_2 \left( \frac{A_1 q + B_1}{C_1 q + D_1} \right) + D_2}
=
\frac{A_2 (A_1 q + B_1 ) + B_2 (C_1 q + D_1)
}{C_2 (A_1 q + B_1) + D_2 (C_1 q + D_1)
}
=
\frac{
(A_2 A_1 + B_2 C_1) q + A_2 B_1 + B_2 D_1
}{
(C_2 A_1 + D_2 C_1) q + C_2 B_1 + D_2 D_1
}
\end{dmath}

Now compare to the double matrix product transformation

\begin{dmath}\label{eqn:problemSet4Problem4:30}
M
= M_2 M_1
=
\begin{bmatrix}
A_2 & B_2 \\
C_2 & D_2 \\
\end{bmatrix}
\begin{bmatrix}
A_1 & B_1 \\
C_1 & D_1 \\
\end{bmatrix}
=
\begin{bmatrix}
A_2 A_1 + B_2 C_1 & A_2 B_1 + B_2 D_1 \\
C_2 A_1 + D_2 C_1 & C_2 B_1 + D_2 D_1
\end{bmatrix}.
\end{dmath}

Writing out the transformation this way we find

\begin{dmath}\label{eqn:problemSet4Problem4:50}
q \rightarrow
\frac{(A_2 A_1 + B_2 C_1) q + A_2 B_1 + B_2 D_1}{
(C_2 A_1 + D_2 C_1) q +  C_2 B_1 + D_2 D_1
},
\end{dmath}

exactly as we found with double application of the M\"obius transformation.

}
