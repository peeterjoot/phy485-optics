%
% Copyright � 2012 Peeter Joot.  All Rights Reserved.
% Licenced as described in the file LICENSE under the root directory of this GIT repository.
%
%\input{../blogpost.tex}
%\renewcommand{\basename}{modernOpticsLecture10}
%\renewcommand{\dirname}{notes/phy485/}
%\newcommand{\keywords}{Optics, PHY485H1F}
%\input{../peeter_prologue_print2.tex}
%\beginArtNoToc
%\generatetitle{PHY485H1F Modern Optics.  Lecture 10: Spatial Coherence.  Taught by Prof.\ Joseph Thywissen}
%%\chapter{Spatial Coherence}
\index{spatial coherence}
%\label{chap:modernOpticsLecture10}
%
%\section{Disclaimer}
%
%Peeter's lecture notes from class.  May not be entirely coherent.
%
\section{Temporal Coherence (cont.)}

For two source interference
%
\begin{dmath}\label{eqn:modernOpticsLecture10:10}
I = I_1 + I_2 + 2 \sqrt{I_1 I_2} \Real \gamma_{12}
\end{dmath}
%
We call \(I_1 + I_2\) the incoherent sum, and now know that \(\gamma_{12}\) is the Fourier transform of the spectral intensity
%
\begin{dmath}\label{eqn:modernOpticsLecture10:30}
\gamma_{12} = \calF \{ I(\omega) \}
\end{dmath}
%
when
%
\begin{dmath}\label{eqn:modernOpticsLecture10:50}
\gamma_{12} = \gamma(\tau)
\end{dmath}
%
and
%
\begin{dmath}\label{eqn:modernOpticsLecture10:70}
\tau = \frac{s_2 - s_1}{c}
\end{dmath}
%
Example in \cref{fig:modernOpticsLecture10:modernOpticsLecture10Fig1}.
%
\imageFigure{../figures/phy485-optics/modernOpticsLecture10Fig1}{Intensity example}{fig:modernOpticsLecture10:modernOpticsLecture10Fig1}{0.3}

Beyond the coherent time \(\tau_{\mathrm{COH}}\) we have only the incoherent intensity, what we'd expect from two flashlights for example.  We can write this down in a nice format
%
\begin{dmath}\label{eqn:modernOpticsLecture10:90}
I = I_1 + I_2 + 2 \sqrt{I_1 I_2} \Abs{ \gamma_{12} } \cos \left( \alpha_{12}(\tau) - \delta \right),
\end{dmath}
%
where \(\delta \equiv \overbar{\omega} \tau\).  Plotting the absolute value \(\Abs{ \gamma_{12}}\) we have something like \cref{fig:modernOpticsLecture10:modernOpticsLecture10Fig2}.
%
\imageFigure{../figures/phy485-optics/modernOpticsLecture10Fig2}{Absolute \(\gamma_{12}\)}{fig:modernOpticsLecture10:modernOpticsLecture10Fig2}{0.2}

Here \(\alpha_{12}\) is the difference in the phase from the average.  As an illustration we may be considering a phase shift at one of the points \cref{fig:modernOpticsLecture10:modernOpticsLecture10Fig3}.
\imageFigure{../figures/phy485-optics/modernOpticsLecture10Fig3}{\(\alpha_{12}\) illustrated}{fig:modernOpticsLecture10:modernOpticsLecture10Fig3}{0.3}

Resulting in a non-zero \(\alpha_{12}\) as in \cref{fig:modernOpticsLecture10:modernOpticsLecture10Fig4}.
%
\imageFigure{../figures/phy485-optics/modernOpticsLecture10Fig4}{Non zero \(\alpha_{12}\)}{fig:modernOpticsLecture10:modernOpticsLecture10Fig4}{0.3}

As opposed to \cref{fig:modernOpticsLecture10:modernOpticsLecture10Fig5}.
%
\imageFigure{../figures/phy485-optics/modernOpticsLecture10Fig5}{Zero \(\alpha_{12}\)}{fig:modernOpticsLecture10:modernOpticsLecture10Fig5}{0.3}

where the figure had been drawn with \(\alpha_{12} = 0\).

Now suppose we rewrite things as:
%
\begin{dmath}\label{eqn:modernOpticsLecture10:110}
I =
\Abs{\gamma_{12}}
\mathLabelBox{
\left(
I_1 + I_2 + 2 \sqrt{I_1 I_2} \cos \left( \alpha_{12}(\tau) - \delta \right)
\right)
}{\(I_{\mathrm{coh}}\)}
+ \left( 1 - \Abs{\gamma_{12}} \right)
\mathLabelBox{
\left( I_1 + I_2 \right)
}{\(I_{\mathrm{incoh}}\)}
\end{dmath}
%
\begin{dmath}\label{eqn:modernOpticsLecture10:130}
\left\{
\begin{array}{l l}
\Abs{\gamma_{12}} = 1 & \quad \mbox{complete coherence} \\
\gamma = 0 & \quad \mbox{incoherent} \\
0 < \Abs{\gamma} < 1 & \quad \mbox{partially coherent light} \\
\end{array}
\right.
\end{dmath}
%
The first case \(\Abs{\gamma_{12}} = 1\) is the easiest case to deal with \cref{fig:modernOpticsLecture10:modernOpticsLecture10Fig6}.
%
\imageFigure{../figures/phy485-optics/modernOpticsLecture10Fig6}{Quasi-monochromatic}{fig:modernOpticsLecture10:modernOpticsLecture10Fig6}{0.2}

We can speak of \underlineAndIndex{Quasi-monochromatic} as the case when
%
\begin{dmath}\label{eqn:modernOpticsLecture10:150}
\tau \ll \tau_{\mathrm{coh}},
\end{dmath}
%
so that we are ignoring finite coherence time, \(\omega \rightarrow \overbar{\omega}\)

We can get this in the lab, by taking an spectrally distributed source like \cref{fig:modernOpticsLecture10:modernOpticsLecture10Fig7a} and filtering it as in \cref{fig:modernOpticsLecture10:modernOpticsLecture10Fig7b}.
%
\imageFigure{../figures/phy485-optics/modernOpticsLecture10Fig7a}{Spectrally distributed source}{fig:modernOpticsLecture10:modernOpticsLecture10Fig7a}{0.2}
\imageFigure{../figures/phy485-optics/modernOpticsLecture10Fig7b}{Filtered source, no longer spectrally distributed}{fig:modernOpticsLecture10:modernOpticsLecture10Fig7b}{0.2}

If our original intensity looked like \cref{fig:modernOpticsLecture10:modernOpticsLecture10Fig7c}, perhaps we now have \cref{fig:modernOpticsLecture10:modernOpticsLecture10Fig7d}.
%
\imageFigure{../figures/phy485-optics/modernOpticsLecture10Fig7c}{Intensity for distributed source}{fig:modernOpticsLecture10:modernOpticsLecture10Fig7c}{0.2}
\imageFigure{../figures/phy485-optics/modernOpticsLecture10Fig7d}{Intensity for filtered source}{fig:modernOpticsLecture10:modernOpticsLecture10Fig7d}{0.2}

We loose some of the maximum possible intensity, but can introduce a lot more fringes.  I'm assuming here that the point here is to use one source to explicitly interfere with another for measurement, so that we want interference, and can make this more severe by reducing the spectral width of the source.

\section{Spatial coherence}
\index{spatial coherence}

We'll talk a bit by how a spatially broad source will mess up the fringes we could measure.

Consider two sources with no mutual coherence as in \cref{fig:modernOpticsLecture10:modernOpticsLecture10Fig8}.
%
\imageFigure{../figures/phy485-optics/modernOpticsLecture10Fig8}{Spatially distributed source}{fig:modernOpticsLecture10:modernOpticsLecture10Fig8}{0.3}

Here what is \(\Gamma_{12}\)?  Recall that
%
\begin{dmath}\label{eqn:modernOpticsLecture10:170}
\Gamma_{12} = \expectation{ \Psi^\conj(\Br_1) \Psi(\Br_2) }
\end{dmath}
%
NOTE: switch of convention here!  In \eqnref{eqn:modernOpticsLecture8:30} we used opposite conjugation.

It could be that we've scrambled up any possible fringes.  We'll eventually be considering a spatially extended source (i.e. a filament in a light bulb \cref{fig:modernOpticsLecture10:modernOpticsLecture10Fig9}), and will deal with that by summing over a source distribution, and first need to know how to deal with a pair of sources.
%
\imageFigure{../figures/phy485-optics/modernOpticsLecture10Fig9}{Spatially distributed source, only when close up}{fig:modernOpticsLecture10:modernOpticsLecture10Fig9}{0.2}

If are kilometers away from the light bulb, the spatial distribution of this source will not matter.

We will find that
%
\begin{dmath}\label{eqn:modernOpticsLecture10:190}
\Gamma_{12} \rightarrow 0,
\end{dmath}
%
over distance
%
\begin{dmath}\label{eqn:modernOpticsLecture10:210}
\inv{\Delta k} = l_{\mathrm{tc}},
\end{dmath}
%
or
\begin{dmath}\label{eqn:modernOpticsLecture10:230}
\frac{\lambda}{\Delta \theta} = l_{\mathrm{tc}}.
\end{dmath}
%
i.e. there's only spatial properties here being considered.  Writing things out
%
\begin{dmath}\label{eqn:modernOpticsLecture10:250}
\Gamma_{12}
= \expectation{ \Psi^\conj(\Br_1) \Psi(\Br_2) }
=
\expectation{
\left( \Psi_{1a} +\Psi_{1b} \right)^\conj
\left( \Psi_{2a} +\Psi_{2b} \right)
}
=
\mathLabelBox{
\expectation{
\Psi_{1a}^\conj
\Psi_{2a}
}
}{\(\Gamma_{12}^a\)}
+
\mathLabelBox{
\expectation{
\Psi_{1a}^\conj
\Psi_{2b}
}
}{\(= 0\)}
+
\mathLabelBox{
\expectation{
\Psi_{1b}^\conj
\Psi_{2a}
}
}{\(= 0\)}
+
\mathLabelBox{
\expectation{
\Psi_{1b}^\conj
\Psi_{2b}
}
}{\(\Gamma_{12}^b\)}
\end{dmath}
%
Here we kill the middle terms because \(a\) and \(b\) have no phase correlation.  Now let's think about what these things look like.  We had an example where we had
%
\begin{subequations}
\begin{dmath}\label{eqn:modernOpticsLecture10:270}
\Gamma_{12}^a = \sqrt{I_1 I_2} \Abs{\gamma_{12}^a} e^{-i\omega \tau_a}
\end{dmath}
\begin{dmath}\label{eqn:modernOpticsLecture10:290}
\Gamma_{12}^b = \sqrt{I_1 I_2} \Abs{\gamma_{12}^b} e^{-i\omega \tau_b}
\end{dmath}
\end{subequations}
%
where
\begin{subequations}
\begin{dmath}\label{eqn:modernOpticsLecture10:310}
\tau_a \equiv \frac{r_{1a} - r_{2a}}{c}
\end{dmath}
\begin{dmath}\label{eqn:modernOpticsLecture10:330}
\tau_b \equiv \frac{r_{1b} - r_{2b}}{c}
\end{dmath}
\end{subequations}
%
For Quasi-monochromatic sources we assume that we have approximately
%
\begin{equation}\label{eqn:modernOpticsLecture10:350}
\Abs{\gamma_{12}^a} = \Abs{\gamma_{12}^b} = 1,
\end{equation}
%
so that
%
\begin{subequations}
\begin{dmath}\label{eqn:modernOpticsLecture10:370}
\Gamma_{12}^a = \sqrt{I_1 I_2} e^{-i\omega \tau_a}
\end{dmath}
\begin{dmath}\label{eqn:modernOpticsLecture10:390}
\Gamma_{12}^b = \sqrt{I_1 I_2} e^{-i\omega \tau_b}
\end{dmath}
\end{subequations}
%
Suppose also that we have equal intensities (we are in the \underlineAndIndex{Far field})
%
\begin{equation}\label{eqn:modernOpticsLecture10:410}
I_1^a = I_2^a = I_1^b = I_2^b,
\end{equation}
%
This will be valid when
%
\begin{dmath}\label{eqn:modernOpticsLecture10:430}
\Abs{r} \gg \Abs{r_1^a - r_2^a} \quad \mbox{etc.}
\end{dmath}
%
We are left with
%
\begin{dmath}\label{eqn:modernOpticsLecture10:450}
\gamma_{12} =
\inv{2} \gamma_{12}^a
+\inv{2} \gamma_{12}^b
=
\inv{2} e^{-i \omega \tau_a}
+\inv{2} e^{-i \omega \tau_b}
\end{dmath}
%
Recall that \(\gamma_{12}\) was defined in \eqnref{eqn:modernOpticsLecture9:170}.  We have
%
\begin{dmath}\label{eqn:modernOpticsLecture10:470}
\gamma_{12} = \cos\left(
\frac{\omega(\tau_a - \tau_b)}{2}
\right)
\end{dmath}
%
In absolute value, this is plotted in \cref{fig:modernOpticsLecture11:modernOpticsLecture11Fig2}.
%
\imageFigure{../figures/phy485-optics/modernOpticsLecture11Fig2}{\(\Abs{\gamma_{12}}\)}{fig:modernOpticsLecture11:modernOpticsLecture11Fig2}{0.3}
%\cref{fig:modernOpticsLecture10:modernOpticsLecture10Fig10}.
%\imageFigure{../figures/phy485-optics/modernOpticsLecture10Fig10}{Plot of \(\Abs{\gamma_{12}}\)}{fig:modernOpticsLecture10:modernOpticsLecture10Fig10}{0.2}

%\EndNoBibArticle
