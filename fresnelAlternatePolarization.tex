%
% Copyright � 2012 Peeter Joot.  All Rights Reserved.
% Licenced as described in the file LICENSE under the root directory of this GIT repository.
%
% pick one:
%\input{../blogpost.tex}
%\renewcommand{\basename}{fresnelAlternatePolarization}
%\renewcommand{\dirname}{notes/phy485/}
%\newcommand{\keywords}{Fresnel equations, Geometric Algebra, Polarization, TE, TM}
%\input{../peeter_prologue_print2.tex}
%\beginArtNoToc
%\label{chap:\basename}
\section{Motivation.}

In \citep{hecht1998hecht} we have a derivation of the Fresnel equations for the TE and TM polarization modes.  Can we do this for an arbitrary polarization angles?

\section{Setup.}

The task at hand is to find evaluate the boundary value constraints.  Following the interface plane conventions of \citep{griffiths1999introduction}, and his notation that is
%
\begin{subequations}
\begin{equation}\label{eqn:fresnelAlternatePolarization:10}
\epsilon_1 ( \BE_i + \BE_r )_z = \epsilon_2 ( \BE_t )_z.
\end{equation}
\begin{equation}\label{eqn:fresnelAlternatePolarization:30}
( \BB_i + \BB_r )_z = ( \BB_t )_z.
\end{equation}
\begin{equation}\label{eqn:fresnelAlternatePolarization:50}
( \BE_i + \BE_r )_{x,y} = ( \BE_t )_{x,y}.
\end{equation}
\begin{equation}\label{eqn:fresnelAlternatePolarization:70}
\inv{\mu_1} ( \BB_i + \BB_r )_{x,y} = \inv{\mu_2} ( \BB_t )_{x,y}.
\end{equation}
\end{subequations}
%
I'll work here with a phasor representation directly and not bother with taking real parts, or using tilde notation to mark the vectors as complex.

Our complex magnetic field phasors are related to the electric fields with
%
\begin{equation}\label{eqn:fresnelAlternatePolarization:90}
\BB = \inv{v} \kcap \cross \BE.
\end{equation}
%
Referring to \cref{fig:fresnelAlternatePolarization:reflectionAtInterfaceLabelledFig2} shows the geometrical task to tackle, since we've got to express all the various unit vectors algebraically.  I'll use Geometric Algebra here to do that for its compact expression of rotations.  With

\pdfTexFigure{../figures/phy485-optics/reflectionAtInterfaceLabelledFig2.pdf_tex}{Reflection and transmission of light at an interface}{fig:fresnelAlternatePolarization:reflectionAtInterfaceLabelledFig2}{0.8}
%
\begin{equation}\label{eqn:fresnelAlternatePolarization:110}
j = \Be_3 \Be_1,
\end{equation}
%
we can express each of the \(k\) vector directions by inspection.  Those are
%
\begin{subequations}
\begin{equation}\label{eqn:fresnelAlternatePolarization:130}
\kcap_i = \Be_3 e^{j \theta_i} = \Be_3 \cos\theta_i + \Be_1 \sin\theta_i.
\end{equation}
\begin{equation}\label{eqn:fresnelAlternatePolarization:150}
\kcap_r = -\Be_3 e^{-j \theta_r} = -\Be_3 \cos\theta_r +\Be_1 \sin\theta_r.
\end{equation}
\begin{equation}\label{eqn:fresnelAlternatePolarization:170}
\kcap_t = \Be_3 e^{j \theta_t} = \Be_3 \cos\theta_t + \Be_1 \sin\theta_t.
\end{equation}
\end{subequations}
%
Similarly, the perpendiculars \(\mcap_p = \Be_2 \cross \kcap_p\) are
\begin{subequations}
\begin{equation}\label{eqn:fresnelAlternatePolarization:190}
\mcap_i
%= -\Be_{1 2 3 2 3} e^{j \theta_i}
= \Be_{1} e^{j \theta_i}
= \Be_1 \cos\theta_i - \Be_3 \sin\theta_i
= \Be_3 j e^{j \theta_i}.
\end{equation}
\begin{equation}\label{eqn:fresnelAlternatePolarization:210}
\mcap_r
%= -\Be_{1 2 3 2 3} -e^{-j \theta_r}
= -\Be_{1} e^{-j \theta_r}
= -\Be_1 \cos\theta_r - \Be_3 \sin\theta_r
= -\Be_3 j e^{-j \theta_r}.
\end{equation}
\begin{equation}\label{eqn:fresnelAlternatePolarization:230}
\mcap_t
%= -\Be_{1 2 3 2 3} e^{j \theta_t}
= \Be_{1 } e^{j \theta_t} = \Be_1 \cos\theta_t - \Be_3 \sin\theta_t
= \Be_3 j e^{j \theta_t}.
\end{equation}
\end{subequations}
%
In \citep{griffiths1999introduction} problem 9.14 we had to show that the polarization angles for normal incident (\(\BE \parallel \Be_1\)) must be the same due to the boundary constraints.  Can we also tackle that problem for both this more general angle of incidence and a general polarization?  Let's try so, allowing temporarily for different polarizations of the reflected and transmitted components of the light, calling those polarization angles \(\phi_i\), \(\phi_r\), and \(\phi_t\) respectively.  Let's set the \(\phi_i = 0\) polarization aligned such that \(\BE_i\), \(\BB_i\) are aligned with the \(\Be_2\) and \(-\mcap_i\) directions respectively, so that the generally polarized phasors are
%
\begin{equation}\label{eqn:fresnelAlternatePolarization:250}
\begin{bmatrix}
\BE_p \\
\BB_p \\
\end{bmatrix}
=
\begin{bmatrix}
\Be_2 \\
-\mcap_p \\
\end{bmatrix}
e^{ \mcap_p \Be_2 \phi_p }.
\end{equation}
%
We are now set to at least express our boundary value constraints
\begin{subequations}
\label{eqn:fresnelAlternatePolarization:300}
\begin{equation}\label{eqn:fresnelAlternatePolarization:310}
\epsilon_1 \left( \Be_2 E_i e^{ \mcap_i \Be_2 \phi_i } + \Be_2 E_r e^{ \mcap_r \Be_2 \phi_r } \right) \cdot \Be_3 = \epsilon_2 \left( \Be_2 E_t e^{ \mcap_t \Be_2 \phi_t } \right) \cdot \Be_3.
\end{equation}
\begin{equation}\label{eqn:fresnelAlternatePolarization:330}
\inv{v_1} \left( -\mcap_i E_i e^{ \mcap_i \Be_2 \phi_i } - \mcap_r E_r e^{ \mcap_r \Be_2 \phi_r } \right) \cdot \Be_3 = \inv{v_2} \left( -\mcap_t E_t e^{ \mcap_t \Be_2 \phi_t } \right) \cdot \Be_3.
\end{equation}
\begin{equation}\label{eqn:fresnelAlternatePolarization:350}
\left( \Be_2 E_i e^{ \mcap_i \Be_2 \phi_i } + \Be_2 E_r e^{ \mcap_r \Be_2 \phi_r } \right) \wedge \Be_3 = \left( \Be_2 E_t e^{ \mcap_t \Be_2 \phi_t } \right) \wedge \Be_3.
\end{equation}
\begin{equation}\label{eqn:fresnelAlternatePolarization:370}
\inv{\mu_1 v_1} \left( -\mcap_i E_i e^{ \mcap_i \Be_2 \phi_i } - \mcap_r E_r e^{ \mcap_r \Be_2 \phi_r } \right) \wedge \Be_3 = \inv{\mu_2 v_2} \left( -\mcap_t E_t e^{ \mcap_t \Be_2 \phi_t } \right) \wedge \Be_3.
\end{equation}
\end{subequations}
%
\section{Solving for the Fresnel equations.}
\index{Fresnel equations}

Let's try this in a couple of steps.  First with polarization angles set so that one of the fields lies in the plane of the interface (with both variations), and then attempt the general case, first posing the problem in the traditional way to see what equations fall out, and then using superposition.

Before doing so, let's introduce a bit of notation to be used throughout.  When we wish to refer to all the fields or angles, for example, \(\BE_i, \BE_r, \BE_t\) then we'll write \(\BE_p\) where \(p \in \{i, r, t\}\).  Similarly, to refer to just the incident and transmitted components (or angles) we'll use \(\BE_q\) where \(q \in \{i, t\}\).  Following \citep{griffiths1999introduction} we'll also write
%
\begin{subequations}
\label{eqn:fresnelAlternatePolarization:750a}
\boxedEquation{eqn:fresnelAlternatePolarization:750}{
\beta = \frac{\mu_1 v_1} {\mu_2 v_2}
}
\boxedEquation{eqn:fresnelAlternatePolarization:770}{
\alpha = \frac{\cos\theta_t}{\cos\theta_i},
}.
\end{subequations}
%
\makeproblem{Sanity check.  Verify for \(\BE\) parallel to the interface.}{fresnelAlternatePolarization:pr1}{ }

\makeanswer{fresnelAlternatePolarization:pr1}{
For the \(\BE_p \parallel \Be_2\) polarization (\(\phi_i = \phi_r = \phi_t\)) our phasors are
%
\begin{subequations}
\begin{dmath}\label{eqn:fresnelAlternatePolarization:390}
\BE_p = \Be_2 E_p.
\end{dmath}
\begin{dmath}\label{eqn:fresnelAlternatePolarization:410}
\BB_p = -\inv{v_p} \mcap_p E_p.
\end{dmath}
\end{subequations}
%
Our boundary value constraints then become
\begin{subequations}
\label{eqn:fresnelAlternatePolarization:300a}
\begin{equation}\label{eqn:fresnelAlternatePolarization:310a}
\epsilon_1 \left( \Be_2 E_i  + \Be_2 E_r  \right) \cdot \Be_3 = \epsilon_2 \left( \Be_2 E_t  \right) \cdot \Be_3.
\end{equation}
\begin{equation}\label{eqn:fresnelAlternatePolarization:330a}
\inv{v_1} \left( \mcap_i E_i + \mcap_r E_r  \right) \cdot \Be_3 = \inv{v_2} \left( \mcap_t E_t  \right) \cdot \Be_3.
\end{equation}
\begin{equation}\label{eqn:fresnelAlternatePolarization:350a}
\left( \Be_2 E_i  + \Be_2 E_r  \right) \wedge \Be_3 = \left( \Be_2 E_t  \right) \wedge \Be_3.
\end{equation}
\begin{equation}\label{eqn:fresnelAlternatePolarization:370a}
\inv{\mu_1 v_1} \left( \mcap_i E_i  + \mcap_r E_r  \right) \wedge \Be_3 = \inv{\mu_2 v_2} \left( \mcap_t E_t  \right) \wedge \Be_3.
\end{equation}
\end{subequations}
%
With \(\mcap_p\) substitution this is
\begin{subequations}
\label{eqn:fresnelAlternatePolarization:300b}
\begin{equation}\label{eqn:fresnelAlternatePolarization:310b}
\epsilon_1 \gpgradezero{ \Be_3 \left( \Be_2 E_i  + \Be_2 E_r  \right) } = \epsilon_2 \gpgradezero{ \Be_3 \left( \Be_2 E_t  \right) }.
\end{equation}
\begin{equation}\label{eqn:fresnelAlternatePolarization:330b}
\inv{v_1} \gpgradezero{ \Be_3 \left( \Be_1 e^{j \theta_i} E_i  - \Be_1 e^{-j \theta_r} E_r  \right) } = \inv{v_2} \gpgradezero{ \Be_3 \left( \Be_1 e^{j \theta_t} E_t  \right) }.
\end{equation}
\begin{equation}\label{eqn:fresnelAlternatePolarization:350b}
\gpgradetwo{ \Be_3 \left( \Be_2 E_i  + \Be_2 E_r  \right) } = \gpgradetwo{ \Be_3 \left( \Be_2 E_t  \right) }.
\end{equation}
\begin{equation}\label{eqn:fresnelAlternatePolarization:370b}
\inv{\mu_1 v_1} \gpgradetwo{ \Be_3 \left( \Be_1 e^{j \theta_i} E_i  -\Be_1 e^{-j \theta_r} E_r  \right) } = \inv{\mu_2 v_2} \gpgradetwo{ \Be_3 \left( \Be_1 e^{j \theta_t} E_t  \right) }.
\end{equation}
\end{subequations}
%
Evaluating the grade selections we have a separation into an analogue of real and imaginary parts for
\begin{subequations}
\label{eqn:fresnelAlternatePolarization:300c}
\begin{equation}\label{eqn:fresnelAlternatePolarization:310c}
0 = 0.
\end{equation}
\begin{equation}\label{eqn:fresnelAlternatePolarization:330c}
\inv{v_1} \left( -\sin\theta_i E_i  - \sin\theta_r E_r  \right) = \inv{v_2} \left( -\sin\theta_t E_t  \right).
\end{equation}
\begin{equation}\label{eqn:fresnelAlternatePolarization:350c}
E_i + E_r = E_t.
\end{equation}
\begin{equation}\label{eqn:fresnelAlternatePolarization:370c}
\inv{\mu_1 v_1} \left( \cos{\theta_i} E_i  - \cos{\theta_r} E_r  \right) = \inv{\mu_2 v_2} \left( \cos{ \theta_t} E_t  \right).
\end{equation}
\end{subequations}
%
With \(\theta_i = \theta_r\) and \(\sin\theta_t/\sin\theta_i = n_1/n_2\) \eqnref{eqn:fresnelAlternatePolarization:330c} becomes
%
\begin{dmath}\label{eqn:fresnelAlternatePolarization:430}
E_i + E_r
= \frac{n_1 v_1}{n_2 v_2} E_t
= \frac{v_2 v_1}{v_1 v_2} E_t
= E_t,
\end{dmath}
%
so that we find \eqnref{eqn:fresnelAlternatePolarization:330c} and \eqnref{eqn:fresnelAlternatePolarization:350c} are dependent.  We are left with a pair of equations
\begin{subequations}
\label{eqn:fresnelAlternatePolarization:450x}
\begin{dmath}\label{eqn:fresnelAlternatePolarization:450}
E_i + E_r = E_t.
\end{dmath}
\begin{dmath}\label{eqn:fresnelAlternatePolarization:470}
E_i - E_r = \frac{\mu_1 v_1}{\mu_2 v_2} \frac{\cos{ \theta_t}}{\cos\theta_i} E_t,
\end{dmath}
\end{subequations}
%
Adding and subtracting we have
%
\begin{subequations}
\begin{dmath}\label{eqn:fresnelAlternatePolarization:490}
2 E_i = \left( 1 + \frac{\mu_1 v_1}{\mu_2 v_2} \frac{\cos{ \theta_t}}{\cos\theta_i} \right) E_t.
\end{dmath}
\begin{dmath}\label{eqn:fresnelAlternatePolarization:510}
2 E_r = \left( 1 - \frac{\mu_1 v_1}{\mu_2 v_2} \frac{\cos{ \theta_t}}{\cos\theta_i} \right) E_t,
\end{dmath}
\end{subequations}
%
with a final rearrangement to yield
%
\begin{subequations}
\begin{dmath}\label{eqn:fresnelAlternatePolarization:530}
\frac{E_t}{E_i}
=
\frac{2 \mu_2 v_2 \cos\theta_i}
{
\mu_2 v_2 \cos\theta_i
+\mu_1 v_1 \cos\theta_t
}.
\end{dmath}
\begin{dmath}\label{eqn:fresnelAlternatePolarization:550}
\frac{E_r}{E_i}
=
\frac
{
\mu_2 v_2 \cos\theta_i
-\mu_1 v_1 \cos\theta_t
}
{
\mu_2 v_2 \cos\theta_i
+\mu_1 v_1 \cos\theta_t
}.
\end{dmath}
\end{subequations}
%
The ratio of field strengths for \(\BE\) parallel to the interface, using notation \eqnref{eqn:fresnelAlternatePolarization:750a}, is
%
\begin{subequations}
\label{eqn:fresnelAlternatePolarization:530b}
\boxedEquation{eqn:fresnelAlternatePolarization:530a}{
\frac{E_t}{E_i}
=
\frac
{
2
}
{
1 + \alpha \beta
}
}
\boxedEquation{eqn:fresnelAlternatePolarization:550a}{
\frac{E_r}{E_i}
=
\frac
{
1 - \alpha \beta
}
{
1 + \alpha \beta
}
}.
\end{subequations}
} % makeanswer
%\shipoutAnswer

\makeproblem{Sanity check.  Verify for \(\BB\) parallel to the interface.}{fresnelAlternatePolarization:pr2}{ }
\makeanswer{fresnelAlternatePolarization:pr2}{

As a second sanity check let's rotate our field polarizations by applying a rotation \(e^{\Be_2 \mcap_p \pi/2} = \Be_2 \mcap_p\) (\(\phi_i = \phi_r = \phi_t = -\pi/2\)) so that
%
\begin{subequations}
\begin{dmath}\label{eqn:fresnelAlternatePolarization:570}
-\mcap_p \rightarrow -\mcap_p \Be_2 \mcap_p = \Be_2.
\end{dmath}
\begin{dmath}\label{eqn:fresnelAlternatePolarization:590}
\Be_2 \rightarrow \Be_2 \Be_2 \mcap_p = \mcap_p.
\end{dmath}
\end{subequations}
%
This time we have \(\BE_p \parallel \mcap_p\) and \(\BB_p \parallel \Be_2\).  Our boundary value equations become
%
\begin{subequations}
\label{eqn:fresnelAlternatePolarization:610}
\begin{equation}\label{eqn:fresnelAlternatePolarization:630}
\epsilon_1 \gpgradezero{ \Be_3 \left( \mcap_i E_i  + \mcap_r E_r  \right) } = \epsilon_2 \gpgradezero{ \Be_3 \left( \mcap_t E_t  \right) }.
\end{equation}
\begin{equation}\label{eqn:fresnelAlternatePolarization:650}
\inv{v_1} \gpgradezero{ \Be_3 \left( \Be_2 E_i + \Be_2 E_r  \right) } = \inv{v_2} \gpgradezero{ \Be_3 \left( \Be_2 E_t  \right) }.
\end{equation}
\begin{equation}\label{eqn:fresnelAlternatePolarization:670}
\gpgradetwo{ \Be_3 \left( \mcap_i E_i  + \mcap_r E_r  \right) } = \gpgradetwo{ \Be_3 \left( \mcap_t E_t  \right) }.
\end{equation}
\begin{equation}\label{eqn:fresnelAlternatePolarization:690}
\inv{\mu_1 v_1} \gpgradetwo{ \Be_3 \left( \Be_2 E_i  + \Be_2 E_r  \right) } = \inv{\mu_2 v_2} \gpgradetwo{ \Be_3 \left( \Be_2 E_t  \right) }.
\end{equation}
\end{subequations}
%
This second \eqnref{eqn:fresnelAlternatePolarization:650} is a \(0 = 0\) identity, and the remaining after \(\mcap_p\) substitution are
%
\begin{subequations}
\label{eqn:fresnelAlternatePolarization:610a}
\begin{equation}\label{eqn:fresnelAlternatePolarization:630a}
\epsilon_1 \gpgradezero{ \Be_3 \left( \Be_3 j e^{j \theta_i} E_i  + (-\Be_3) j e^{-j \theta_r} E_r  \right) } = \epsilon_2 \gpgradezero{ \Be_3 \left( \Be_3 j e^{j \theta_t} E_t  \right) }.
\end{equation}
\begin{equation}\label{eqn:fresnelAlternatePolarization:670a}
\gpgradetwo{ \Be_3 \left( \Be_3 j e^{j \theta_i} E_i  + (-\Be_3) j e^{-j \theta_r} E_r  \right) } = \gpgradetwo{ \Be_3 \left( \Be_3 j e^{j \theta_t} E_t  \right) }.
\end{equation}
\begin{equation}\label{eqn:fresnelAlternatePolarization:690a}
\inv{\mu_1 v_1} \gpgradetwo{ \Be_3 \left( \Be_2 E_i  + \Be_2 E_r  \right) } = \inv{\mu_2 v_2} \gpgradetwo{ \Be_3 \left( \Be_2 E_t  \right) }.
\end{equation}
\end{subequations}
%
Simplifying we have
%
\begin{subequations}
\label{eqn:fresnelAlternatePolarization:610b}
\begin{equation}\label{eqn:fresnelAlternatePolarization:630b}
\epsilon_1 \left(  -\sin \theta_i E_i  - \sin{\theta_r} E_r  \right) = - \epsilon_2 \sin{\theta_t} E_t.
\end{equation}
\begin{equation}\label{eqn:fresnelAlternatePolarization:670b}
\cos{ \theta_i} E_i  - \cos{ \theta_r} E_r = \cos{ \theta_t} E_t.
\end{equation}
\begin{equation}\label{eqn:fresnelAlternatePolarization:690b}
E_i  + E_r = \frac{\mu_1 v_1} {\mu_2 v_2} E_t.
\end{equation}
\end{subequations}
%
Noting that \(\epsilon_p v_p = 1/(v_p \mu_p)\) we find
%
\begin{dmath}\label{eqn:fresnelAlternatePolarization:730}
\frac{\epsilon_2 \sin\theta_t}{\epsilon_1 \sin\theta_i}
=
\frac{\epsilon_2 n_1}{\epsilon_1 n_2}
=
\frac{\epsilon_2 v_2}{\epsilon_1 v_1}
=
\frac{\mu_1 v_1}{\mu_2 v_2}.
\end{dmath}
%
showing that \eqnref{eqn:fresnelAlternatePolarization:630b} and \eqnref{eqn:fresnelAlternatePolarization:690b} are dependent.  We are left with the system
%
\begin{subequations}
\label{eqn:fresnelAlternatePolarization:610d}
\begin{equation}\label{eqn:fresnelAlternatePolarization:670d}
E_i - E_r = \alpha E_t.
\end{equation}
\begin{equation}\label{eqn:fresnelAlternatePolarization:690d}
E_i + E_r = \beta E_t.
\end{equation}
\end{subequations}
%
This time we find that the ratio of field strengths for \(\BB\) parallel to the interface, again using notation \eqnref{eqn:fresnelAlternatePolarization:750a}, is

with solution
\begin{subequations}
\label{eqn:fresnelAlternatePolarization:790a}
\boxedEquation{eqn:fresnelAlternatePolarization:790}{
\frac{E_t}{E_i} = \frac{2 }{\beta + \alpha}
}
\boxedEquation{eqn:fresnelAlternatePolarization:810}{
\frac{E_r}{E_i} = \frac{\beta - \alpha}{\beta + \alpha}
}.
\end{subequations}
} % makeanswer

%\shipoutAnswer
\makeproblem{General case.  Arbitrary polarization angle.}{fresnelAlternatePolarization:pr3}{ Determine the set of simultaneous equations that would have to be solved for if the incident polarization angle was allowed to be neither TE nor TM mode.}
\makeanswer{fresnelAlternatePolarization:pr3}{

Substituting our \(\mcap_p\) vector expressions into the boundary value constraints we have
%
\begin{subequations}
\label{eqn:fresnelAlternatePolarization:830}
\begin{equation}\label{eqn:fresnelAlternatePolarization:850}
\epsilon_1 \gpgradezero{ \Be_3 \Be_2 \left( E_i e^{ \mcap_i \Be_2 \phi_i } + E_r e^{ \mcap_r \Be_2 \phi_r } \right) } = \epsilon_2 \gpgradezero{ \Be_3 \Be_2 E_t e^{ \mcap_t \Be_2 \phi_t } }.
\end{equation}
\begin{equation}\label{eqn:fresnelAlternatePolarization:870}
\inv{v_1} \gpgradezero{ j e^{j \theta_i} E_i e^{ \mcap_i \Be_2 \phi_i } - j e^{-j \theta_r} E_r e^{ \mcap_r \Be_2 \phi_r } } = \inv{v_2} \gpgradezero{ j e^{j \theta_t} E_t e^{ \mcap_t \Be_2 \phi_t } }.
\end{equation}
\begin{equation}\label{eqn:fresnelAlternatePolarization:890}
\gpgradetwo{ \Be_3 \Be_2 \left( E_i e^{ \mcap_i \Be_2 \phi_i } + E_r e^{ \mcap_r \Be_2 \phi_r } \right) } = \gpgradetwo{ \Be_3 \Be_2 E_t e^{ \mcap_t \Be_2 \phi_t } }.
\end{equation}
\begin{equation}\label{eqn:fresnelAlternatePolarization:910}
\inv{\mu_1 v_1} \gpgradetwo{ j e^{j \theta_i} E_i e^{ \mcap_i \Be_2 \phi_i } - j e^{-j \theta_r} E_r e^{ \mcap_r \Be_2 \phi_r } } = \inv{\mu_2 v_2} \gpgradetwo{ j e^{j \theta_t} E_t e^{ \mcap_t \Be_2 \phi_t } }.
\end{equation}
\end{subequations}
%
We want to expand some intermediate multivector products
%
\begin{dmath}\label{eqn:fresnelAlternatePolarization:930}
\Be_{32} e^{\mcap_q \Be_2 \phi_q}
=
\Be_{32} \cos \phi_q
+\Be_{32} \mcap_q \Be_2 \sin{\phi_q}
=
\Be_{32} \cos \phi_q
+\Be_{32} \Be_3 j e^{j \theta_q} \Be_2 \sin{\phi_q}
=
\Be_{32} \cos \phi_q
-j e^{j \theta_q} \sin{\phi_q}
=
\Be_{32} \cos \phi_q - \Be_{31} \cos\theta_q \sin\phi_q
+ \sin\theta_q \sin{\phi_q}.
\end{dmath}
%
\begin{dmath}\label{eqn:fresnelAlternatePolarization:930b}
\Be_{32} e^{\mcap_r \Be_2 \phi_r}
=
\Be_{32} \cos \phi_r
+\Be_{32} \mcap_r \Be_2 \sin{\phi_r}
=
\Be_{32} \cos \phi_r
+\Be_{32} (-\Be_3) j e^{-j \theta_r} \Be_2 \sin{\phi_r}
=
\Be_{32} \cos \phi_r
+ j e^{-j \theta_r} \sin{\phi_r}
=
\Be_{32} \cos \phi_r + \Be_{31} \cos\theta_r \sin{\phi_r}
+ \sin \theta_r \sin{\phi_r}.
\end{dmath}
%
\begin{dmath}\label{eqn:fresnelAlternatePolarization:950}
j e^{j \theta_q} e^{\mcap_q \Be_2 \phi_q}
=
j e^{j \theta_q} \left(
\cos \phi_q
+ \mcap_q \Be_2 \sin{\phi_q}
\right)
=
j e^{j \theta_q} \left(
\cos \phi_q
+ \Be_3 j e^{j \theta_q} \Be_2 \sin{\phi_q}
\right)
=
j e^{j \theta_q} \left(
\cos \phi_q
- j e^{-j \theta_q} \Be_{32} \sin{\phi_q}
\right)
=
j e^{j \theta_q} \cos \phi_q
+ \Be_{32} \sin{\phi_q}
=
\Be_{31} \cos {j \theta_q} \cos \phi_q
+ \Be_{32} \sin{\phi_q}
- \sin{ \theta_q} \cos \phi_q.
\end{dmath}
%
\begin{dmath}\label{eqn:fresnelAlternatePolarization:970}
-j e^{-j \theta_r} e^{\mcap_r \Be_2 \phi_r}
=
-j e^{-j \theta_r} \left(
\cos\phi_r + \mcap_r \Be_2 \sin\phi_r
\right)
=
-j e^{-j \theta_r} \left(
\cos\phi_r - \Be_3 j e^{-j \theta_r} \Be_2 \sin\phi_r
\right)
=
-j e^{-j \theta_r} \left(
\cos\phi_r + j e^{j \theta_r} \Be_{32} \sin\phi_r
\right)
=
-j e^{-j \theta_r} \cos\phi_r
+ \Be_{32} \sin\phi_r
=
-\Be_{31} \cos{\theta_r} \cos\phi_r
+ \Be_{32} \sin\phi_r
- \sin{ \theta_r} \cos\phi_r.
\end{dmath}
%
Our boundary value conditions are then
%
\begin{subequations}
\begin{dmath}\label{eqn:fresnelAlternatePolarization:990}
\epsilon_1 \left(
E_i
\sin\theta_i \sin{\phi_i}
+ E_r
\sin \theta_r \sin{\phi_r}
\right)
= \epsilon_2 E_t
\sin\theta_t \sin{\phi_t}.
\end{dmath}
\begin{dmath}\label{eqn:fresnelAlternatePolarization:1010}
\inv{v_1}
\left(
E_i
\sin{ \theta_i} \cos \phi_i
+
E_r
\sin{ \theta_r} \cos\phi_r
\right)
=
\inv{v_2}
E_t
\sin{ \theta_t} \cos \phi_t.
\end{dmath}
%\begin{dmath}\label{eqn:fresnelAlternatePolarization:1030}
%E_i
%\left(
%\Be_{2} \cos \phi_i - \Be_{1} \cos\theta_i \sin\phi_i
%\right)
%+
%E_r
%\left(
%\Be_{2} \cos \phi_r + \Be_{1} \cos\theta_r \sin{\phi_r}
%\right)
%=
%E_t
%\left(
%\Be_{2} \cos \phi_t - \Be_{1} \cos\theta_t \sin\phi_t
%\right)
%\end{dmath}
%\begin{dmath}\label{eqn:fresnelAlternatePolarization:1050}
%\inv{\mu_1 v_1}
%\left(
%E_i
%\left(
%\Be_{1} \cos { \theta_i} \cos \phi_i + \Be_{2} \sin{\phi_i}
%\right)
%+
%E_r
%\left(
%-\Be_{1} \cos{\theta_r} \cos\phi_r + \Be_{2} \sin\phi_r
%\right)
%\right)
%=
%\inv{\mu_2 v_2}
%E_t
%\left(
%\Be_{1} \cos { \theta_t} \cos \phi_t + \Be_{2} \sin{\phi_t}
%\right)
%\end{dmath}
\begin{dmath}\label{eqn:fresnelAlternatePolarization:1030a}
E_i
\cos \phi_i
+
E_r
\cos \phi_r
=
E_t
\cos \phi_t.
\end{dmath}
%
\begin{dmath}\label{eqn:fresnelAlternatePolarization:1030b}
-E_i
\cos\theta_i \sin\phi_i
+
E_r
\cos\theta_r \sin{\phi_r}
=
-E_t
\cos\theta_t \sin\phi_t.
\end{dmath}
\begin{dmath}\label{eqn:fresnelAlternatePolarization:1050a}
\inv{\mu_1 v_1}
\left(
E_i
\cos { \theta_i} \cos \phi_i
-
E_r
\cos{\theta_r} \cos\phi_r
\right)
=
\inv{\mu_2 v_2}
E_t
 \cos { \theta_t} \cos \phi_t.
\end{dmath}
%
\begin{dmath}\label{eqn:fresnelAlternatePolarization:1050b}
\inv{\mu_1 v_1}
\left(
E_i
\sin{\phi_i}
+
E_r
\sin\phi_r
\right)
=
\inv{\mu_2 v_2}
E_t
\sin{\phi_t}.
\end{dmath}
\end{subequations}
%
Note that the wedge product equations above have been separated into \(\Be_3 \Be_1\) and \(\Be_3 \Be_2\) components, yielding two equations each.  Because of \eqnref{eqn:fresnelAlternatePolarization:730}, we see that \eqnref{eqn:fresnelAlternatePolarization:990} and \eqnref{eqn:fresnelAlternatePolarization:1050b} are dependent.  Also, as demonstrated in \eqnref{eqn:fresnelAlternatePolarization:430} we see that \eqnref{eqn:fresnelAlternatePolarization:1010} and \eqnref{eqn:fresnelAlternatePolarization:1030a} are also dependent.  We can therefore consider only the last four equations (and still have additional linear dependencies to be discovered.)

Let's write these as
\begin{subequations}
\label{eqn:fresnelAlternatePolarization:1070a}
\begin{dmath}\label{eqn:fresnelAlternatePolarization:1070}
E_i
\cos \phi_i
+
E_r
\cos \phi_r
=
E_t
\cos \phi_t.
\end{dmath}
%
\begin{dmath}\label{eqn:fresnelAlternatePolarization:1090}
-E_i
\sin\phi_i
+
E_r
\sin{\phi_r}
=
-E_t \alpha
\sin\phi_t.
\end{dmath}
\begin{dmath}\label{eqn:fresnelAlternatePolarization:1110}
E_i
\cos \phi_i
-
E_r
\cos\phi_r
=
\alpha \beta
E_t
\cos \phi_t.
\end{dmath}
%
\begin{dmath}\label{eqn:fresnelAlternatePolarization:1130}
E_i
\sin{\phi_i}
+
E_r
\sin\phi_r
=
\beta
E_t
\sin{\phi_t}.
\end{dmath}
\end{subequations}
%
Observe that if \(\phi_i = \phi_r = \phi_t = 0\) (killing all the sine terms) we recover \eqnref{eqn:fresnelAlternatePolarization:450x}, and with \(\phi_i = \phi_r = \phi_t = \pi/2\) (killing all the cosines) we recover \eqnref{eqn:fresnelAlternatePolarization:610d}.

Now, if \(\phi_i = n \pi/2\) we've got a different story.  Specifically it appears that should we wish to solve for the reflected and transmitted magnitudes, we also have to simultaneously solve for the polarization angles in the reflected and transmitted directions.  This is now a problem of solving four simultaneous equations in two linear and two non-linear variables.

Does it make sense that we would have polarization rotation should our initial polarization angle be rotated?  I think so.  In discussing this problem with Prof Thywissen, he strongly suggested treating the problem as a superposition of two light waves.  If we consider that, even without attempting to solve the problem, we see that we must have different reflected and transmitted magnitudes associated with the pair of incident waves since we have to calculate each of these with different Fresnel equations.  This would have an effect of scaling and rotating the superimposed reflected and transmitted waves.
} % makeanswer

%\shipoutAnswer
\makeproblem{General case using superposition.}{fresnelAlternatePolarization:pr4}{
Using superposition determine the Fresnel equations for an arbitrary incident polarization angle.  This should involve solving for both the magnitude and the polarization angle of the reflected and transmitted rays.
}
\makeanswer{fresnelAlternatePolarization:pr4}{

For a polarization of \(\phi = 0\) and \(\phi = \pi/2\) respectively, we found in
\eqnref{eqn:fresnelAlternatePolarization:530b}, and
\eqnref{eqn:fresnelAlternatePolarization:790a}
%problems
%\ref{fresnelAlternatePolarization:pr1-Answer}
%and
%\ref{fresnelAlternatePolarization:pr2-Answer}
%, or from \eqnref{eqn:fresnelAlternatePolarization:1070a} we have
%
\begin{subequations}
\begin{dmath}\label{eqn:fresnelAlternatePolarization:1150}
\frac{E_{r \parallel}}{E_{i \parallel}} = \frac{1 - \alpha \beta}{1 + \alpha \beta}.
\end{dmath}
\begin{dmath}\label{eqn:fresnelAlternatePolarization:1170}
\frac{E_{t \parallel}}{E_{i \parallel}} = \frac{2 }{1 + \alpha \beta}.
\end{dmath}
\begin{dmath}\label{eqn:fresnelAlternatePolarization:1190}
\frac{E_{r \perp}}{E_{i \perp}} = \frac{ \beta - \alpha }{\beta + \alpha}.
\end{dmath}
\begin{dmath}\label{eqn:fresnelAlternatePolarization:1210}
\frac{E_{t \perp}}{E_{i \perp}} = \frac{ 2 }{\beta + \alpha}.
\end{dmath}
\end{subequations}
%
%Thought I must have had my results mixed up.  Have the mixed sign terms only on the reflected component (which doesn't mesh with the idea of total internal reflection since it allows for zero internal reflection and total transmission?)  Also, don't get \(T + R = 1\) for the \(\BE \perp \Be_2\) case with these results. -- nope.  checked against Hecht (and \BE || \Be_3 in Griffiths).  Brewsters angle defines a condition for total TRANSMISSION (not reflection), and the above results are correct.

We can use these results to consider a polarization of \(\phi < \pi/2\) as illustrated in \cref{fig:fresnelAlternatePolarization:fresnelAlternatePolarizationFig2}.

\pdfTexFigure{../figures/phy485-optics/fresnelAlternatePolarizationFig2.pdf_tex}{Polarization of incident field to be considered}{fig:fresnelAlternatePolarization:fresnelAlternatePolarizationFig2}{0.8}

Our incident, reflected, and transmitted fields are
%
\begin{subequations}
\begin{dmath}\label{eqn:fresnelAlternatePolarization:1230}
\BE_i = E_{i} \Be_2 e^{\Be_2 \mcap_i \phi}.
\end{dmath}
\begin{dmath}\label{eqn:fresnelAlternatePolarization:1250}
\BE_r =
E_{i \parallel}
\frac{1 - \alpha\beta}{1 + \alpha\beta} \Be_2
+ E_{i \perp}
\frac{\beta - \alpha}{\beta + \alpha} \mcap_r.
\end{dmath}
\begin{dmath}\label{eqn:fresnelAlternatePolarization:1270}
\BE_t =
E_{i \parallel}
\frac{2}{1 + \alpha\beta}
\Be_2
+ E_{i \perp}
\frac{2}{\beta + \alpha}
\mcap_i.
\end{dmath}
\end{subequations}
%

However, \(E_{i \parallel} = E_i \cos \phi\) and \(E_{i \perp} = E_i \sin\phi\) leaving us with
%
\begin{subequations}
\begin{dmath}\label{eqn:fresnelAlternatePolarization:1290}
\BE_i = E_{i} \left( \Be_2 \cos\phi + \Be_1 e^{j \theta_i} \sin\phi \right).
\end{dmath}
\begin{dmath}\label{eqn:fresnelAlternatePolarization:1310}
\BE_r =
E_i
\left(
\cos\phi
\frac{1 - \alpha\beta}{1 + \alpha\beta} \Be_2
- \sin\phi
\frac{\beta - \alpha}{\beta + \alpha} \Be_1 e^{-j \theta_r}
\right).
\end{dmath}
\begin{dmath}\label{eqn:fresnelAlternatePolarization:1330}
\BE_t =
E_i
\left(
\cos\phi
\frac{2}{1 + \alpha\beta}
\Be_2
+ \sin\phi
\frac{2}{\beta + \alpha}
\Be_1 e^{j \theta_t}
\right).
\end{dmath}
\end{subequations}
%

We find that the reflected and transmitted polarization angles are respectively
\begin{subequations}
\begin{dmath}\label{eqn:fresnelAlternatePolarization:1350}
\tan \phi_r = \tan \phi
\frac{\beta - \alpha}{\beta + \alpha}
\frac{1 + \alpha \beta}{1 - \alpha \beta}.
\end{dmath}
\begin{dmath}\label{eqn:fresnelAlternatePolarization:1370}
\tan \phi_t = \tan \phi \frac{ 1 + \alpha \beta}{ \beta + \alpha}.
\end{dmath}
\end{subequations}
%
where the associated magnitudes are
%
\begin{subequations}
\begin{dmath}\label{eqn:fresnelAlternatePolarization:1310c}
\frac{E_r}{E_i}
=
\sqrt{
\left(\cos\phi
\frac{1 - \alpha\beta}{1 + \alpha\beta} \right)^2
+ \left( \sin\phi
\frac{\beta - \alpha}{\beta + \alpha}
\right)^2
}.
\end{dmath}
\begin{dmath}\label{eqn:fresnelAlternatePolarization:1330c}
\frac{E_t}{E_i}
=
\sqrt{
\left(
\cos\phi
\frac{2}{1 + \alpha\beta}
\right)^2
+\left(
\sin\phi
\frac{2}{\beta + \alpha}
\right)^2
}.
\end{dmath}
\end{subequations}
%
FIXME: in \citep{dirac1974principles} he claims in \S 2 that ``if polarized at an angle \(\phi\) to the axis, a fraction \(\sin^2 \phi\) will go through''.  Either I have my result above wrong, or this appears to be an approximate statement?

} % makeanswer

%\shipoutAnswer
%\makeproblem{Show that we have \(R + T = 1\)}{fresnelAlternatePolarization:pr5}{ }
%\makeanswer{fresnelAlternatePolarization:pr5}{ TODO. }
% ... see Griffiths ... have to scale the E_r/E_i and E_t/E_i values to do this calculation.  Otherwise it should be easy.

%\vcsinfo
%\EndArticle
