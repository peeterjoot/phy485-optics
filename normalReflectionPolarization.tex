%
% Copyright � 2013 Peeter Joot.  All Rights Reserved.
% Licenced as described in the file LICENSE under the root directory of this GIT repository.
%
%\input{../blogpost.tex}
%\renewcommand{\basename}{normalReflectionPolarization}
%\renewcommand{\dirname}{notes/phy485/}
%\newcommand{\keywords}{polarization, reflection, transmission, boundary value conditions, electromagnetism, Maxwell's equations}
%
%\input{../peeter_prologue_print2.tex}
%
%\beginArtNoToc
%
%\generatetitle{Polarization angles for normal transmission and reflection}
%\chapter{Polarization angles for normal transmission and reflection}
\index{polarization angle}
\index{normal transmission}
\index{normal reflection}
%\label{chap:normalReflectionPolarization}

\makeoproblem
%{Polarization angles for normal transmission and reflection.}
{Normal polarization angles.}
{pr:normalReflectionPolarization:1}{\citep{griffiths1999introduction} pr 9.14}{
For normal incidence, without assuming that the reflected and transmitted waves have the same polarization as the incident wave, prove that this must be so.
} % makeoproblem

\makeanswer{pr:normalReflectionPolarization:1}{
Working with coordinates as illustrated in \cref{fig:normalReflectionPolarization:normalReflectionPolarizationFig1}, the incident wave can be assumed to have the form
\imageFigure{../figures/phy485-optics/normalReflectionPolarizationFig1}{Normal incidence coordinates.}{fig:normalReflectionPolarization:normalReflectionPolarizationFig1}{0.3}
\begin{subequations}
\begin{equation}\label{eqn:normalReflectionPolarization:20}
\tilde{\BE}_{\txtI} = E_{\txtI} e^{i (k z - \omega t)} \xcap.
\end{equation}
\begin{equation}\label{eqn:normalReflectionPolarization:40}
\tilde{\BB}_{\txtI} = \inv{v} \zcap \cross \tilde{\BE}_{\txtI} = \inv{v} E_{\txtI} e^{i (k z - \omega t)} \ycap.
\end{equation}
\end{subequations}
%
Assuming a polarization \(\ncap = \cos\theta \xcap + \sin\theta \ycap\) for the reflected wave, we have
%
\begin{subequations}
\begin{equation}\label{eqn:normalReflectionPolarization:60}
\tilde{\BE}_{\txtR} = E_{\txtR} e^{i (-k z - \omega t)} (\xcap \cos\theta + \ycap \sin\theta).
\end{equation}
\begin{equation}\label{eqn:normalReflectionPolarization:80}
\tilde{\BB}_{\txtR} = \inv{v} (-\zcap) \cross \tilde{\BE}_{\txtR} = \inv{v} E_{\txtR} e^{i (-k z - \omega t)} (\xcap \sin\theta - \ycap \cos\theta).
\end{equation}
\end{subequations}
%
And finally assuming a polarization \(\ncap = \cos\phi \xcap + \sin\phi \ycap\) for the transmitted wave, we have
%
\begin{subequations}
\begin{equation}\label{eqn:normalReflectionPolarization:100}
\tilde{\BE}_{\txtT} = E_{\txtT} e^{i (k' z - \omega t)} (\xcap \cos\phi + \ycap \sin\phi).
\end{equation}
\begin{equation}\label{eqn:normalReflectionPolarization:120}
\tilde{\BB}_{\txtT} = \inv{v} \zcap \cross \tilde{\BE}_{\txtT} = \inv{v'} E_{\txtT} e^{i (k' z - \omega t)} (-\xcap \sin\phi + \ycap \cos\phi).
\end{equation}
\end{subequations}
%
With no components of any of the \(\tilde{\BE}\) or \(\tilde{\BB}\) waves in the \(\zcap\) directions the boundary value conditions at \(z = 0\) require the equality of the \(\xcap\) and \(\ycap\) components of
%
\begin{subequations}
\begin{equation}\label{eqn:normalReflectionPolarization:140}
\lr{\tilde{\BE}_{\txtI} + \tilde{\BE}_{\txtR}}
_{x,y} = \lr{\tilde{\BE}_{\txtT}}
_{x,y}.
\end{equation}
\begin{equation}\label{eqn:normalReflectionPolarization:160}
\lr{\inv{\mu}\lr{ \tilde{\BB}_{\txtI} + \tilde{\BB}_{\txtR}}}
_{x,y} =
\lr{\inv{\mu'} \tilde{\BB}_{\txtT}}
_{x,y}.
\end{equation}
\end{subequations}
%
With \(\beta = \mu v/\mu' v'\), those components are
%
\begin{subequations}
\label{eqn:normalReflectionPolarization:260}
\begin{equation}\label{eqn:normalReflectionPolarization:180}
E_{\txtI} + E_{\txtR} \cos\theta = E_{\txtT} \cos\phi.
\end{equation}
\begin{equation}\label{eqn:normalReflectionPolarization:200}
E_{\txtR} \sin\theta = E_{\txtT} \sin\phi.
\end{equation}
\begin{equation}\label{eqn:normalReflectionPolarization:220}
E_{\txtR} \sin\theta = - \beta E_{\txtT} \sin\phi.
\end{equation}
\begin{equation}\label{eqn:normalReflectionPolarization:240}
E_{\txtI} - E_{\txtR} \cos\theta = \beta E_{\txtT} \cos\phi.
\end{equation}
\end{subequations}
%
Equality of \eqnref{eqn:normalReflectionPolarization:200}, and \eqnref{eqn:normalReflectionPolarization:220} require
%
\begin{equation}\label{eqn:normalReflectionPolarization:280}
- \beta E_{\txtT} \sin\phi = E_{\txtT} \sin\phi,
\end{equation}
%
or \((\theta, \phi) \in \{(0, 0), (0, \pi), (\pi, 0), (\pi, \pi)\}\).  It turns out that all of these solutions correspond to the same physical waves.  Let's look at each in turn

\begin{enumerate}
\item \((\theta, \phi) = (0, 0)\).  The system \eqnref{eqn:normalReflectionPolarization:260} is reduced to
%
\begin{equation}\label{eqn:normalReflectionPolarization:300}
\begin{aligned}
E_{\txtI} + E_{\txtR} &= E_{\txtT} \\
E_{\txtI} - E_{\txtR} &= \beta E_{\txtT},
\end{aligned}
\end{equation}
%
with solution
%
\begin{equation}\label{eqn:normalReflectionPolarization:320}
\begin{aligned}
\frac{E_{\txtT}}{E_{\txtI}} &= \frac{2}{1 + \beta} \\
\frac{E_{\txtR}}{E_{\txtI}} &= \frac{1 - \beta}{1 + \beta}.
\end{aligned}
\end{equation}
%
\item \((\theta, \phi) = (\pi, \pi)\).  The system \eqnref{eqn:normalReflectionPolarization:260} is reduced to
%
\begin{equation}\label{eqn:normalReflectionPolarization:340}
\begin{aligned}
E_{\txtI} - E_{\txtR} &= -E_{\txtT} \\
E_{\txtI} + E_{\txtR} &= -\beta E_{\txtT},
\end{aligned}
\end{equation}
%
with solution
%
\begin{equation}\label{eqn:normalReflectionPolarization:360}
\begin{aligned}
\frac{E_{\txtT}}{E_{\txtI}} &= -\frac{2}{1 + \beta} \\
\frac{E_{\txtR}}{E_{\txtI}} &= -\frac{1 - \beta}{1 + \beta}.
\end{aligned}
\end{equation}
%
Effectively the sign for the magnitude of the transmitted and reflected phasors is toggled, but the polarization vectors are also negated, with \(\ncap = -\xcap\), and \(\ncap' = -\xcap\).  The resulting \(\tilde{\BE}_{\txtR}\) and \(\tilde{\BE}_{\txtT}\) are unchanged relative to those of the \((0,0)\) solution above.

\item \((\theta, \phi) = (0, \pi)\).  The system \eqnref{eqn:normalReflectionPolarization:260} is reduced to
%
\begin{equation}\label{eqn:normalReflectionPolarization:380}
\begin{aligned}
E_{\txtI} + E_{\txtR} &= -E_{\txtT} \\
E_{\txtI} - E_{\txtR} &= -\beta E_{\txtT},
\end{aligned}
\end{equation}
%
with solution
%
\begin{equation}\label{eqn:normalReflectionPolarization:400}
\begin{aligned}
\frac{E_{\txtT}}{E_{\txtI}} &= -\frac{2}{1 + \beta} \\
\frac{E_{\txtR}}{E_{\txtI}} &= \frac{1 - \beta}{1 + \beta}.
\end{aligned}
\end{equation}
%
Effectively the sign for the magnitude of the transmitted phasor is toggled.  The polarization vectors in this case are \(\ncap = \xcap\), and \(\ncap' = -\xcap\), so the transmitted phasor magnitude change of sign does not change \(\tilde{\BE}_{\txtT}\) relative to that of the \((0,0)\) solution above.

\item \((\theta, \phi) = (\pi, 0)\).  The system \eqnref{eqn:normalReflectionPolarization:260} is reduced to
%
\begin{equation}\label{eqn:normalReflectionPolarization:420}
\begin{aligned}
E_{\txtI} - E_{\txtR} &= E_{\txtT} \\
E_{\txtI} + E_{\txtR} &= \beta E_{\txtT},
\end{aligned}
\end{equation}
%
with solution
%
\begin{equation}\label{eqn:normalReflectionPolarization:440}
\begin{aligned}
\frac{E_{\txtT}}{E_{\txtI}} &= \frac{2}{1 + \beta} \\
\frac{E_{\txtR}}{E_{\txtI}} &= -\frac{1 - \beta}{1 + \beta}.
\end{aligned}
\end{equation}
%
This time, the sign for the magnitude of the reflected phasor is toggled.  The polarization vectors in this case are \(\ncap = -\xcap\), and \(\ncap' = \xcap\).  In this final variation the reflected phasor magnitude change of sign does not change \(\tilde{\BE}_{\txtR}\) relative to that of the \((0,0)\) solution.
\end{enumerate}

We see that there is only one solution for the polarization angle of the transmitted and reflected waves relative to the incident wave.  Although we fixed the incident polarization with \(\BE\) along \(\xcap\), the polarization of the incident wave is maintained regardless of TE or TM labeling in this example, since our system is symmetric with respect to rotation.

} % makeanswer

%\EndArticle
