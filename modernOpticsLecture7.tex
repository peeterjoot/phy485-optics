%
% Copyright � 2012 Peeter Joot.  All Rights Reserved.
% Licenced as described in the file LICENSE under the root directory of this GIT repository.
%
%\input{../blogpost.tex}
%\renewcommand{\basename}{modernOpticsLecture7}
%\renewcommand{\dirname}{notes/phy485/}
%\newcommand{\keywords}{Optics, PHY485H1F}
%\input{../peeter_prologue_print2.tex}
%% \gtrsim amssymb
%\beginArtNoToc
%\generatetitle{PHY485H1F Modern Optics.  Lecture 7: Coherence.  Taught by Prof.\ Joseph Thywissen}
\label{chap:modernOpticsLecture7}

%\section{Disclaimer}
%
%Peeter's lecture notes from class.  May not be entirely coherent.
%
\section{Interference.}
\index{interference}

Given a number of sources as illustrated in \cref{fig:modernOpticsLecture7:modernOpticsLecture7Fig1}, we can have interference (fringes, ... due to motion or polarization.  In general we can consider these multiple sources as a sum over all the electric fields
\imageFigure{../figures/phy485-optics/modernOpticsLecture7Fig1}{Multiple sources potentially interfering.}{fig:modernOpticsLecture7:modernOpticsLecture7Fig1}{0.2}
%
\begin{equation}\label{eqn:modernOpticsLecture7:10}
\BE = \BE_1 + \BE_2 + \BE_3 + \cdots.
\end{equation}
%
While there are some experiments that are sensitive to the actual fields, we will typically not care about those specifically, but instead will care about Intensity which has the form
%
\begin{equation}\label{eqn:modernOpticsLecture7:30}
I = (\text{const}) \expectation{ \Abs{\BE}^2 }.
\end{equation}
%
here the constant is unit dependent, such as \(c \epsilon_0\), but since we will typically be looking at ratios, we can ignore those.  We will move to an intensity based description, but first start with a field description.

Consider two sources

\begin{align}\label{eqn:modernOpticsLecture7:50}
\BE_1 &= \Bepsilon_1 E_1(\Br) e^{i \phi_1(\Br, t) } \\
\BE_2 &= \Bepsilon_2 E_2(\Br) e^{i \phi_2(\Br, t) }
\end{align}

Here \(\Bepsilon_i\) is a polarization constant, which we allow to be complex.  The values \(E_i\) are the amplitudes which we constrain to have real values here, and \(\phi_i\) is the (real) phase angle.

We observe
%
\begin{equation}\label{eqn:modernOpticsLecture7:70}
I = \expectation{ \Abs{\BE}^2 }.
\end{equation}
%
where
%
\begin{equation}\label{eqn:modernOpticsLecture7:90}
\expectation{ f(t) } = \lim_{T \rightarrow \infty } \inv{T} \int_0^T dt' f(t').
\end{equation}
%
% \gtrsym amssymb
This time averaging method makes sense for optics where we may have response times \( T \gtrsim 10^{-9} \mbox{s}\), so that \(w T \gg 1\) (\textunderline{slow!})

Forming the magnitude of the field square we have
%
\begin{dmath}\label{eqn:modernOpticsLecture7:110}
\expectation{ \Abs{\BE}^2 }
=
\expectation{
( \BE_1 + \BE_2 ) \cdot ( \BE_1 + \BE_2 )^\conj
}
=
\expectation{
\Abs{\BE_1 }^2
}
+
\expectation{
\Abs{\BE_2 }^2
}
+
\expectation{
\BE_1 \cdot \BE_2^\conj
+ \BE_2 \cdot \BE_1^\conj
}
=
I_1 + I_2 + 2 \Real\left(  \expectation{ \BE_1 \cdot \BE_2^\conj} \right).
\end{dmath}
%
Consider the cross term.  We have
%
\begin{equation}\label{eqn:modernOpticsLecture7:130}
\BE_1 \cdot \BE_2^\conj
= \Bepsilon_1 \cdot \Bepsilon_2^\conj E_1 E_2 e^{i \phi_1 - i \phi_2 }.
\end{equation}
%
We see immediately that if the polarization vectors \(\Bepsilon_1\) and \(\Bepsilon_2\) are orthogonal, we have no interference.  Let's consider some examples of some polarization vectors

\begin{itemize}
\item Linear polarization

\begin{align}\label{eqn:modernOpticsLecture7:150}
\Bepsilon_1 &= \xcap \\
\Bepsilon_2 &= \ycap
\end{align}

or
\begin{align}\label{eqn:modernOpticsLecture7:170}
\Bepsilon_1 &= \inv{\sqrt{2}} (\xcap + \ycap) \\
\Bepsilon_2 &= \inv{\sqrt{2}} (\xcap - \ycap)
\end{align}

\item circular polarization

\begin{align}\label{eqn:modernOpticsLecture7:190}
\Bepsilon_1 &=
\inv{\sqrt{2}}
\begin{bmatrix}
1 \\
i
\end{bmatrix}
= \sigma^{+} \\
\Bepsilon_2 &=
\inv{\sqrt{2}}
\begin{bmatrix}
1 \\
-i
\end{bmatrix}
= \sigma^{-}
\end{align}
(here we require the conjugation to make \(\Bepsilon_1 \cdot \Bepsilon_2^\conj = 0\))
\end{itemize}
READING: Polarization: \S 5 \citep{hecht1998hecht} and \S 2 \citep{fowles1989introduction}.  This will be considered background material and not covered here.

Two sources in a scalar theory are
\begin{align}\label{eqn:modernOpticsLecture7:210}
\Psi_1 &= \sqrt{I_1(\Br)} e^{i \phi_1(\Br, t)} \\
\Psi_2 &= \sqrt{I_2(\Br)} e^{i \phi_2(\Br, t)}.
\end{align}
Here \(\sqrt{I_2(\Br)}\) are the amplitudes (real).
\paragraph{Interference (scalar or identical polarizations)}
\index{interference!identical polarizations}
%
\begin{dmath}\label{eqn:modernOpticsLecture7:230}
I
= \Abs{\Psi_1 + \Psi_2}^2
= \Abs{\Psi_1}^2
+ \Abs{\Psi_2}^2
+ 2 \Real \expectation{ \Psi_1 \Psi_2^\conj}
= I_1 + I_2
+ 2 \sqrt{I_1 I_2} \Real \expectation{ e^{i \phi_1 - i \phi_2 } }
= I_1 + I_2
+ 2 \sqrt{I_1 I_2} \expectation{ \cos(\phi_1 - \phi_2) }.
\end{dmath}
%
Here we've made use of the fact that \(\Real(.)\) and \(\expectation{.}\) are both linear operators so we can reverse their order of operation.
\paragraph{Question: Isn't the average of cosine just zero?}  Answer: we are considering phases that can vary with time.  We don't necessarily have a constant phase difference here that would be wiped out in an average over one period.

\paragraph{Example: Diffraction}
\index{diffraction}
%
\begin{dmath}\label{eqn:modernOpticsLecture7:250}
\Psi = \sum_{\text{paths}} \Psi_i \rightarrow \int \sqrt{I_i} e^{i \phi_i} = (\text{prefactor}) \iint_{\text{aperture}} e^{i k f} da'.
\end{dmath}
%
Did we have interference in the diffraction example as illustrated in \cref{fig:modernOpticsLecture7:modernOpticsLecture7Fig2}.
\imageFigure{../figures/phy485-optics/modernOpticsLecture7Fig2}{A diffraction geometry to consider.}{fig:modernOpticsLecture7:modernOpticsLecture7Fig2}{0.2}
%
We had

\begin{align}\label{eqn:modernOpticsLecture7:270}
\phi_1 &= \Bk_1 \cdot \Br - \omega t + \theta_1 \\
\phi_2 &= \Bk_2 \cdot \Br - \omega t + \theta_2
\end{align}

This was monochromatic light (\(\omega\) was the same).  In this diffraction case we had
%
\begin{equation}\label{eqn:modernOpticsLecture7:290}
\phi_1 -\phi_2 = (\Bk_1 - \Bk_2) \cdot \Br + \Delta \theta.
\end{equation}
%
We can figure out from the geometry, and using the far-field limit (\(x \gg b\)) that we have a time independent phase difference
%
\begin{equation}\label{eqn:modernOpticsLecture7:310}
(\Bk_1 - \Bk_2) \cdot \Br = -2 \pi \frac{b y}{r \lambda}.
\end{equation}
%
What is our intensity?
%
\begin{equation}\label{eqn:modernOpticsLecture7:330}
I = I_1 + I_2 + 2 \sqrt{I_1 I_2} \cos\left( -2 \pi \frac{b}{r \lambda} y + \Delta \theta \right).
\end{equation}
%
Here \(y\) is the position of observation, as illustrated in \cref{fig:modernOpticsLecture7:modernOpticsLecture7Fig3}.
\imageFigure{../figures/phy485-optics/modernOpticsLecture7Fig3}{Some intensity variation with visibility.}{fig:modernOpticsLecture7:modernOpticsLecture7Fig3}{0.2}
%
\index{visibility}
\makedefinition{Visibility}{dfn:modernOpticsLecture7:11}{
%
\boxedEquation{eqn:modernOpticsLecture7:350}{
\calV \equiv \frac
{
I_{\mathrm{max}}
-I_{\mathrm{min}}}
{
I_{\mathrm{max}}
+I_{\mathrm{min}}}.
}
}

This is a quantity that is easy to measure in the lab.

In this (diffraction) case we have
%
\begin{equation}\label{eqn:modernOpticsLecture7:370}
\calV = 2 \frac{\sqrt{I_1 I_2}}{I_1 + I_2}.
\end{equation}
%
We illustrate this in \cref{fig:modernOpticsLecture7:modernOpticsLecture7Fig4}, where after zoom we see the same image.  This is called Heterodyne amplification.
%
\imageFigure{../figures/phy485-optics/modernOpticsLecture7Fig4}{Heterodyne detection.}{fig:modernOpticsLecture7:modernOpticsLecture7Fig4}{0.2}
%
\index{heterodyne detection}
\makedefinition{Heterodyne Detection}{dfn:modernOpticsLecture7:1}{
Measure a phase of a weak beam: Interfere with a strong beam ``local oscillator'' (\(I_{\mathrm{L0}}\))!  Interference : \(2 \sqrt{I_1 I_2}\).  Even if \(I_1\) is small this interference term can be big.
}
%
\begin{equation}\label{eqn:modernOpticsLecture7:390}
2 \sqrt{I_p I_{\mathrm{LO}}} \gg I_p.
\end{equation}
%
\begin{equation}\label{eqn:modernOpticsLecture7:410}
4 I_p I_{\mathrm{LO}} \gg I_p^2.
\end{equation}
%
\begin{equation}\label{eqn:modernOpticsLecture7:430}
I_{\mathrm{LO}} \gg I_p/4.
\end{equation}
%
%\vcsinfo
%\EndArticle
