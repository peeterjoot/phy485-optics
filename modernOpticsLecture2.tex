%
% Copyright � 2012 Peeter Joot.  All Rights Reserved.
% Licenced as described in the file LICENSE under the root directory of this GIT repository.
%
% pick one:
%\input{../assignment.tex}
%\input{../blogpost.tex}
%\renewcommand{\basename}{modernOpticsLecture2}
%\renewcommand{\dirname}{notes/phy485/}
%\newcommand{\keywords}{Optics, PHY485H1F, Rays, plane waves, index of refraction, Eikonal equation}
%\input{../peeter_prologue_print2.tex}
%\beginArtNoToc
%\generatetitle{PHY485H1F Modern Optics.  Lecture 2: Geometric optics: Rays.  Taught by Prof.\ Joseph Thywissen}
\label{chap:modernOpticsLecture2}
%\section{Disclaimer}
%
%Peeter's lecture notes from class.  May not be entirely coherent.

\section{Reading}

Reading: \S 3.1.1 \citep{born1980principles}

\section{Eikonal equation.}
We want to find the rays in Maxwell's equations.
\index{Eikonal equation}
\index{ray}
\index{Maxwell's equations}
\begin{equation}\label{eqn:modernOpticsLecture2:10}
\begin{aligned}
\spacegrad \cdot \BD &= 0 \\
\spacegrad \cdot \BB &= 0 \\
\spacegrad \cross \BE &= - \inv{c} \PD{t}{\BB} \\
\spacegrad \cross \BB &= \inv{c} \PD{t}{\BD}.
\end{aligned}
\end{equation}
%
Assume
\begin{enumerate}
\item
Material has no magnetic dependence (\(\mu = 1\)) so that we have \footnote{
Note the choice of units here, with no \(c\) in the definition of \(n\).  In SI we'd write
%
\begin{equation}\label{eqn:modernOpticsLecture2:670}
n = \frac{c}{v} = \frac{c}{1/\sqrt{\epsilon\mu}} \approx c \sqrt{\epsilon}.
\end{equation}
%
however, with these units (and \(\mu = 1\)) the wave equation operator takes the form
\begin{equation}\label{eqn:modernOpticsLecture2:770}
\spacegrad^2 - \frac{\epsilon}{c^2} \partial_{tt}.
\end{equation}
%
From this we deduce the wave velocity is \(c/\sqrt{\epsilon}\), and then find \(n = c/v\) matches \eqnref{eqn:modernOpticsLecture2:5} above.
}
\begin{equation}\label{eqn:modernOpticsLecture2:5}
n = \sqrt{\epsilon}.
\end{equation}
We'll neglect loss, with zero for the imaginary part of \(n\)
\item
Short wavelength limit \(\lambda \ll d\), any other length scale in problem.
\end{enumerate}

If \(n = \text{constant}\), we know that plane waves are solutions.  Try
\begin{equation}\label{eqn:modernOpticsLecture2:30}
\begin{bmatrix}
\BE \\
\BB \\
\end{bmatrix}
=
\begin{bmatrix}
\BE_0(\Br) \\
\BB_0(\Br) \\
\end{bmatrix}
e^{i \phi(\Br) - i \omega t}
\end{equation}
%
We know that for plane waves we'll have
\begin{equation}\label{eqn:modernOpticsLecture2:50}
\begin{aligned}
\BE_0(\Br) &\rightarrow \BE \\
\BB_0(\Br) &\rightarrow \BB \\
\phi(\Br) &\rightarrow \Bk \cdot \Br.
\end{aligned}
\end{equation}
%
The time derivatives are
\begin{equation}\label{eqn:modernOpticsLecture2:70}
\inv{c} \PD{t}{}
\begin{bmatrix}
\BE \\
\BB
\end{bmatrix}
=
- i \frac{\omega}{c}
\begin{bmatrix}
\BE \\
\BB
\end{bmatrix}
=
- i k_0
\begin{bmatrix}
\BE \\
\BB
\end{bmatrix}.
\end{equation}
%
For the spatial derivatives we have
%
\begin{equation}\label{eqn:modernOpticsLecture2:90}
\spacegrad \cdot \BE =
e^{-i \omega t}
\biglr{
\mathLabelBox{
e^{i \phi(\Br)}
\spacegrad \cdot \BE_0(\Br)
}{neglect this}
+
\BE_0(\Br) \cdot \left( \spacegrad e^{i \phi(\Br)} \right)
},
\end{equation}
%
and
\begin{equation}\label{eqn:modernOpticsLecture2:91}
\spacegrad \cross \BE =
e^{-i \omega t}
\biglr{
\mathLabelBox{
e^{i \phi(\Br)}
\spacegrad \cross \BE_0(\Br)
}{neglect this}
-
\BE_0(\Br) \cross \left( \spacegrad e^{i \phi(\Br)} \right)
}.
\end{equation}
%
We can computing the phase gradient directly
%
\begin{equation}\label{eqn:modernOpticsLecture2:790}
\begin{aligned}
\spacegrad e^{i\phi}
&=
\Be_m \partial_m e^{i \phi} \\
&=
i \Be_m \partial_m \phi e^{i \phi} \\
&=
i (\spacegrad \phi) e^{i \phi},
\end{aligned}
\end{equation}
%
and use this to justify the neglect of the gradients products of \(\BE_0\) above.  Since
%
\begin{equation}\label{eqn:modernOpticsLecture2:110}
\inv{E_0^n} \frac{d E_0^n}{dx_p} \ll \inv{\lambda}
\end{equation}
%
for all \(x_p \in \{x, y, z\}\).  Since
%
\begin{equation}\label{eqn:modernOpticsLecture2:130}
\spacegrad \phi \approx k_0 \sim \frac{2 \pi}{\lambda}
\end{equation}
%
we see that the non-neglected terms above are of order \(\Abs{\BE_0}/\lambda\), which justifies the action.

This leaves us with
\begin{equation}\label{eqn:modernOpticsLecture2:90a}
\spacegrad \cdot \BE \approx
i e^{i \phi -i \omega t} \BE_0(\Br) \cdot \spacegrad \phi
\end{equation}
\begin{equation}\label{eqn:modernOpticsLecture2:91b}
\spacegrad \cross \BE \approx
-i e^{i \phi -i \omega t}
\BE_0(\Br) \cross \spacegrad \phi.
\end{equation}
%
Maxwell's equations now take the form
%
\begin{subequations}
\label{eqn:modernOpticsLecture2:140}
\begin{equation}\label{eqn:modernOpticsLecture2:150}
\BE_0 \cdot \spacegrad \phi = 0
\end{equation}
\begin{equation}\label{eqn:modernOpticsLecture2:170}
\BB_0 \cdot \spacegrad \phi = 0
\end{equation}
\begin{equation}\label{eqn:modernOpticsLecture2:190}
\spacegrad \phi \cross \BE_0 = k_0 \BB_0
\end{equation}
\begin{equation}\label{eqn:modernOpticsLecture2:210}
\spacegrad \phi \cross \BB_0 = - \epsilon k_0 \BE_0
\end{equation}
\end{subequations}
%
Crossing \(\spacegrad \phi\) with \eqnref{eqn:modernOpticsLecture2:190} we have
%
\begin{equation}\label{eqn:modernOpticsLecture2:810}
\begin{aligned}
\spacegrad \phi \cross (\spacegrad \phi \cross \BE_0) &= k_0 (\spacegrad \phi \cross \BB_0) \\
\spacegrad \phi \left( \cancel{\spacegrad \phi \cdot \BE_0} \right) - \BE_0 (\spacegrad \phi)^2 &= - \epsilon k_0^2 \BE_0
\end{aligned}
\end{equation}
%
This is called the Eikonal equation and can be written as
%
\begin{equation}\label{eqn:modernOpticsLecture2:230}
\Abs{\spacegrad \phi}^2 = k_0^2 \epsilon(\Br)
\end{equation}
%
or
\boxedEquation{eqn:modernOpticsLecture2:250}{
\Abs{\spacegrad \phi} = k_0 n(\Br)
}

If \(n = \text{constant}\)
%
\begin{equation}\label{eqn:modernOpticsLecture2:270}
\Abs{\spacegrad \phi} = k_0 n
\end{equation}
%
This can be illustrated as in \cref{fig:modernOpticsLecture2:modernOpticsLecture2Fig1}.
%
\imageFigure{../figures/phy485-optics/modernOpticsLecture2Fig1}{Plane waves for constant index of refraction}{fig:modernOpticsLecture2:modernOpticsLecture2Fig1}{0.2}
%
If \(n \ne \text{constant}\) only locally would we have plane waves as in \cref{fig:modernOpticsLecture2:modernOpticsLecture2Fig2}.
%
\imageFigure{../figures/phy485-optics/modernOpticsLecture2Fig2}{Plane waves only locally with variation of index of refraction}{fig:modernOpticsLecture2:modernOpticsLecture2Fig2}{0.2}
%
\section{Poynting vector}
\index{Poynting vector}

How about the Poynting vector?  This is the direction of the ``ray'', the direction of the transport of energy and momentum.  That is
%
\begin{equation}\label{eqn:modernOpticsLecture2:290}
\BS = \frac{c}{4 \pi} \Real{\BE} \cross \Real{\BB},
\end{equation}
%
and after some math, taking the average we have
%
\begin{equation}\label{eqn:modernOpticsLecture2:310}
\expectation{\BS}_{\mathrm{time}} = \frac{c}{8 \pi k_0} \Abs{\BE_0}^2 \spacegrad \phi
\end{equation}
%
We see that the rays point along \(\spacegrad \phi\).

\section{Ray equation}
\index{ray equation}

Referring to \cref{fig:modernOpticsLecture2:modernOpticsLecture2Fig3} we let
\imageFigure{../figures/phy485-optics/modernOpticsLecture2Fig3}{Unit tangents on a curve}{fig:modernOpticsLecture2:modernOpticsLecture2Fig3}{0.2}
%
\begin{equation}\label{eqn:modernOpticsLecture2:330}
s = \text{distance along ray}
\end{equation}
\begin{equation}\label{eqn:modernOpticsLecture2:350}
\Bt = \text{tangent} = \frac{d\Br(s)}{ds}
\end{equation}
%
The unit vector, parallel to \(\spacegrad \phi\) is
%
\begin{equation}\label{eqn:modernOpticsLecture2:830}
\begin{aligned}
\frac{d\Br(s)}{ds} = \Bt
&= \frac{\spacegrad \phi}{\Abs{\spacegrad \phi}} \\
&= \frac{\spacegrad \phi}{ n(\Br) k_0},
\end{aligned}
\end{equation}
%
So we have
%
\begin{equation}\label{eqn:modernOpticsLecture2:370}
n(\Br) \frac{d\Br}{ds} = \inv{k_0} \spacegrad \phi.
\end{equation}
%
We'd like to get rid of the pesky dependence on the phase.  Let's take another derivative to attempt to get rid of \(\spacegrad \phi\).  Will this work?
%
\begin{equation}\label{eqn:modernOpticsLecture2:850}
\begin{aligned}
\frac{d}{ds} \left( n(\Br) \frac{d\Br}{ds} \right)
&=
\inv{k_0} \frac{d}{ds} \spacegrad \phi \\
&=
\inv{k_0} \left( \frac{d\Br}{ds} \cdot \spacegrad \right) \spacegrad \phi \\
&=
\inv{k_0} \left( \inv{ k_0 n(\Br) } \spacegrad \phi \cdot \spacegrad \right)
\spacegrad \phi
\end{aligned}
\end{equation}
%
Here we've used the convective derivative
%
\begin{equation}\label{eqn:modernOpticsLecture2:690}
\frac{d}{ds} = \PD{s}{} + \PD{s}{\Br} \cdot \spacegrad = \frac{d \Br}{ds} \cdot \spacegrad.
\end{equation}
%
In \cref{modernOpticsLecture2:pr2} we show that
%
\begin{equation}\label{eqn:modernOpticsLecture2:390}
(\spacegrad \phi \cdot \spacegrad ) \spacegrad \phi = k_0^2 n \spacegrad n,
\end{equation}
%
which gives us the \underlineAndIndex{Ray equation}

\boxedEquation{eqn:modernOpticsLecture2:410}{
\frac{d}{ds} \left( n(\Br) \frac{d\Br}{ds} \right) = \spacegrad n(\Br)
}

Note that this almost looks like a \(F = m a\) type of equation with time parameterization replaced by arc length along the ray (should we ignore the index of refraction on the LHS), and also ignore the lack of a minus sign.

The lack of minus sign we can interpret as something like ``bending to higher \(n\)''.

\section{GRIN (Graded Refractive INdex) optics}
\index{graded refractive index}
\index{GRIN|see {graded refractive index}}

With a constant index we have
%
\begin{equation}\label{eqn:modernOpticsLecture2:430}
\frac{d}{ds} \left( n \frac{d\Br}{ds} \right) = \spacegrad n = 0
\end{equation}
%
So
%
\begin{equation}\label{eqn:modernOpticsLecture2:450}
\frac{d^2}{ds^2} \Br(s) = 0
\end{equation}
%
Integrating twice we see this is the straight ray that we expect

%\dr/ds = const
\begin{equation}\label{eqn:modernOpticsLecture2:470}
\Br = s \Ba + \Br_0.
\end{equation}
%
We can compute the unit tangent
%
\begin{equation}\label{eqn:modernOpticsLecture2:490}
\frac{d\Br}{ds} = \Ba = \inv{n k_0} \spacegrad \phi,
\end{equation}
%
finding that our constant vector \(\Ba\), in this case, is a unit vector in the direction of the gradient.
This shows that the gradient of the phase lies along the ray path of wave front.

The ray \(\Br(t)\) for a fixed phase front (not a general expression) can be implicitly defined by
\begin{equation}\label{eqn:modernOpticsLecture2:710}
\phi(\Br, t) = \omega t,
\end{equation}
%
Or more generally
%
\begin{equation}\label{eqn:modernOpticsLecture2:510}
\Br \cdot \spacegrad \phi(\Br) = \omega t.
\end{equation}
%
We find
%
\begin{equation}\label{eqn:modernOpticsLecture2:530}
\Ba \cdot \Br = \frac{\omega t}{n \omega/c} = \frac{c}{n} t = vt,
\end{equation}
%
so the phase front of the wave moves with speed \(c/n\) along the ray direction.

\section{Trap a ray}
\index{ray!trap}

Let's have some fun with non-constant \(n\).  Can we trap a ray of light as in \cref{fig:modernOpticsLecture2:modernOpticsLecture2Fig4}?
%
\imageFigure{../figures/phy485-optics/modernOpticsLecture2Fig4}{Ray trap}{fig:modernOpticsLecture2:modernOpticsLecture2Fig4}{0.2}
%
If we have a circular trajectory
%
\begin{equation}\label{eqn:modernOpticsLecture2:550}
\Br = R
\begin{bmatrix}
\cos\theta(s) \\
\sin\theta(s) \\
0
\end{bmatrix}.
\end{equation}
%
We can imagine any sort of variation of \(n\) with \(\Br\), such as \cref{fig:modernOpticsLecture2:modernOpticsLecture2Fig5}, but
we want to figure out exactly what \(n(\Br)\) has to be.  We can do so by plugging into \eqnref{eqn:modernOpticsLecture2:410}.
%
\imageFigure{../figures/phy485-optics/modernOpticsLecture2Fig5}{Imagined possible relationship between index of refraction and position.}{fig:modernOpticsLecture2:modernOpticsLecture2Fig5}{0.2}
%
We'll assume that \(n(\Br)\) is radially symmetric, so that given the constant radius of the ray (a circle) we have \(dn/ds = 0\).  This gives
%
\begin{equation}\label{eqn:modernOpticsLecture2:570}
\frac{d^2}{ds^2} \Br(s) = \frac{\spacegrad n(\Br)}{n(\Br)},
\end{equation}
%
Taking derivatives, we have
%
\begin{equation}\label{eqn:modernOpticsLecture2:590}
\frac{d \Br}{ds} = R
\begin{bmatrix}
-\sin\theta(s) \\
\cos\theta(s) \\
0
\end{bmatrix}
\frac{d\theta}{ds}
\end{equation}
%
Since \(s = R \theta\), we have \(d\theta/ds = 1/R\), and
%
\begin{equation}\label{eqn:modernOpticsLecture2:590a}
\frac{d \Br}{ds} =
\begin{bmatrix}
-\sin\theta(s) \\
\cos\theta(s) \\
0
\end{bmatrix}.
\end{equation}
%
Taking the next derivative we have
%
\begin{equation}\label{eqn:modernOpticsLecture2:590b}
\frac{d^2 \Br}{ds^2} =
\begin{bmatrix}
-\cos\theta(s) \\
-\sin\theta(s) \\
0
\end{bmatrix}
\inv{R}
= -\inv{R^2} \Br
\end{equation}
%
We find
%
\begin{equation}\label{eqn:modernOpticsLecture2:630}
\frac{d^2 \Br}{ds^2} = \frac{\spacegrad n}{n}
\end{equation}
%
or
%
\begin{equation}\label{eqn:modernOpticsLecture2:650}
\spacegrad n = -n \frac{\Br}{R^2}.
\end{equation}
%
Trap your own ray today!


%\vcsinfo
%\EndArticle
%\EndNoBibArticle
