%
% Copyright � 2012 Peeter Joot.  All Rights Reserved.
% Licenced as described in the file LICENSE under the root directory of this GIT repository.
%
% pick one:
%\input{../assignment.tex}
%\input{../blogpost.tex}
%\renewcommand{\basename}{snellsFermat}
%\renewcommand{\dirname}{notes/phy485/}
%\newcommand{\dateintitle}{}
%\newcommand{\keywords}{}
%
%\input{../peeter_prologue_print2.tex}
%
%\beginArtNoToc
%
%\generatetitle{Derivation of Snell's law using Fermat's theorem}
\label{chap:snellsFermat}

\makeproblem{Derive Snell's law.}{snellsFermat:pr:1}{
Fermat's theorem, that light takes the path of least time, can be used to derive Snell's law without resorting to Maxwell's equations.

Note that a proof of Fermat's theorem using the Ray equation can be found in \S 3.3.2 \citep{born1980principles}.
}
\makeanswer{snellsFermat:pr:1}{

We refer to \cref{fig:snellsFermat:snellsFermatFig1}, and seek to express the optical path length.
%
\imageFigure{../figures/phy485-optics/snellsFermatFig1}{Snell's law light paths.}{fig:snellsFermat:snellsFermatFig1}{0.2}
%
Since \(n(s) = c/v(s)\), the time spent along any portion of the path is proportional to \(n(s) ds\).  For the two leg linear route that is
%
\begin{equation}\label{eqn:snellsFermat:10}
OPL = n r + n' r'.
\end{equation}
%
Since
%
\begin{subequations}
\begin{equation}\label{eqn:snellsFermat:30}
r = \sqrt{ h^2 + (L - x)^2 }.
\end{equation}
\begin{equation}\label{eqn:snellsFermat:50}
r' = \sqrt{ {h'}^2 + x^2 }.
\end{equation}
\end{subequations}
%
We want to find \(x\) such that
%
\begin{equation}\label{eqn:snellsFermat:70}
\begin{aligned}
0
&= \frac{d(OPL)}{dx} \\
&=
\ddx{} 
\lr{
n \sqrt{ h^2 + (L - x)^2 }
+ n' \sqrt{ {h'}^2 + x^2 }
}
\\
&=
n \inv{2 r} 2 (L - x)(-1)
+ n' \inv{2 r'} 2 x \\
&=
- n \sin\theta + n' \sin\theta'.
\end{aligned}
\end{equation}
%
This gives us
%
\boxedEquation{eqn:snellsFermat:90}{
n \sin\theta = n' \sin\theta',
}

as desired.
} % makeanswer
\shipoutAnswer

%\vcsinfo
%\EndNoBibArticle
