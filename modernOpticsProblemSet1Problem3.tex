%
% Copyright � 2012 Peeter Joot.  All Rights Reserved.
% Licenced as described in the file LICENSE under the root directory of this GIT repository.
%
%\input{../assignment.tex}
%\renewcommand{\basename}{modernOpticsProblem Set1}
%\renewcommand{\dirname}{notes/phy485/}
%\newcommand{\keywords}{Optics, PHY485H1F}
%\newcommand{\dateintitle}{}
%\input{../peeter_prologue_print2.tex}
%\beginArtNoToc
%\generatetitle{PHY485H1F Modern Optics.  Problem Set 1: Geometric Optics}
%\chapter{Problem set 1.  Geometric Optics}
\index{geometric optics}
%\label{chap:modernOpticsProblem Set1}

\makeoproblem{Ray equation at a surface.}{modernOptics:problemSet1:3}
{2012 Ps1, P3}
{
Show that Snell's law can be derived from the {\em transverse} component of the ray equation applied at an index step. Set up the problem with an index step from \(n_1\) in the half-plane \(x<0\); and \(n_2\) in the half-plane \(x>0\) \cref{fig:modernOpticsProblemSet1:FigureSnell}. Define your rays according to two straight-line trajectories: a ray in the \(xy\) plane defined by \(x=s \cos{\theta_1}\) and \(y=s \sin{\theta_1}\) for \(x<0\); and \(x=s \cos{\theta_2}\) and \(y=s \sin{\theta_2}\) for \(x>0\).
\begin{enumerate}
\item[(a)] Solve the { transverse} (or y-) component of the Ray Equation. Show that it gives Snell's law.
\item[(b)] Show that the {\em normal} (or x-) component of the Ray Equation is contradictory, unless the limit of a small index step is taken. Why is this? What is missing?
\end{enumerate}
%
\imageFigure{../figures/phy485-optics/FigureSnell}{.}{fig:modernOpticsProblemSet1:FigureSnell}{0.2}
} % makeoproblem

\makeanswer{modernOptics:problemSet1:3}{
\begin{enumerate}
\item[(a)]
The index of refraction \(n(x)\) has no y-component, so we have
%
\begin{equation}\label{eqn:modernOpticsProblemSet1P3:710}
\ycap \cdot \spacegrad n = 0.
\end{equation}
%
The y-component of the Ray equation
%
\begin{equation}\label{eqn:modernOpticsProblemSet1P3:910}
\dds{} \left( n(x) \frac{dy}{ds} \right) = 0,
\end{equation}
%
can therefore be integrated directly
%
\begin{equation}\label{eqn:modernOpticsProblemSet1P3:930}
n(x) \frac{dy}{ds} = \text{constant}.
\end{equation}
%
With the chosen ray parameterization we have for \(x < 0\) the y-component of the ray ``velocity''
%
\begin{dmath}\label{eqn:modernOpticsProblemSet1P3:670}
\ycap \cdot \dds{\Br_1}
= \ycap \cdot \dds{} s (\cos\theta_1, \sin\theta_1)
= \ycap \cdot (\cos\theta_1, \sin\theta_1)
= \sin\theta_1.
\end{dmath}
%
Similarly for the y-component in the \(x > 0\) region we have
%
\begin{equation}\label{eqn:modernOpticsProblemSet1P3:690}
\ycap \cdot \dds{\Br_2} = \sin\theta_2.
\end{equation}
%
We want to use this in the integrated Ray equation \eqnref{eqn:modernOpticsProblemSet1P3:930} which takes the form
%
\begin{dmath}\label{eqn:modernOpticsProblemSet1P3:950}
n_1 \frac{dy_1}{ds} = \text{constant} = n_2 \frac{dy_2}{ds},
\end{dmath}
%
but since we have found that \(dy_1/ds = \sin\theta_1\) and \(dy_2/ds = \sin\theta_2\), we have Snell's law
%
\boxedEquation{eqn:modernOpticsProblemSet1P3:770}{
n_2 \sin\theta_2 = n_1 \sin\theta_1.
}

\item[(b)]
We can produce a contradictory result if we avoid the origin when treating the x-component of the Ray equation.  Repeating the argument above for \(\Abs{x} > 0\) where \(\spacegrad n = 0\) would give us
%
\begin{equation}\label{eqn:modernOpticsProblemSet1P3:790}
n \dds{x} = \text{constant}.
\end{equation}
%
With \(dx_1/ds = \cos\theta_1\) and \(dx_2/ds = \cos\theta_2\) we would have
%
\begin{equation}\label{eqn:modernOpticsProblemSet1P3:810}
n_2 \cos\theta_2 = n_1 \cos\theta_1,
\end{equation}
%
which contradicts Snell's law.

This conclusion isn't valid because we have avoided the origin, where the index of refraction is not continuous.  What is missing is proper treatment of this step discontinuity.  To frame this properly, let's express the index of refraction a bit more precisely.  That is
%
\begin{equation}\label{eqn:modernOpticsProblemSet1P3:830}
n(x) = n_1 + \Delta n \theta(x).
\end{equation}
%
where \(\Delta n = n_2 - n_1\).  We now have a non-zero gradient
%
\begin{equation}\label{eqn:modernOpticsProblemSet1P3:850}
\spacegrad n = \xcap \Delta n \delta(x).
\end{equation}
%
The Ray equation, split by coordinates, now takes the form
%
\boxedEquation{eqn:modernOpticsProblemSet1P3:870}{
\begin{aligned}
\dds{} \left(
\left( \theta(x) n_2
+
\theta(-x) n_1 \right) \dds{x}
\right)
&= \Delta n \delta(x) \\
\left( \theta(x) n_2
+
\theta(-x) n_1 \right) \dds{y}
&= \text{constant}.
\end{aligned}
}

Note that any solution of the above must also take into account the dependence between \(s\), \(x\) and \(y\)
%
\begin{equation}\label{eqn:modernOpticsProblemSet1P3:1930}
ds^2 = dx^2 + dy^2,
\end{equation}
%
or
\begin{equation}\label{eqn:modernOpticsProblemSet1P3:1950}
1 = (dx/ds)^2 + (dy/ds)^2.
\end{equation}
%
%Should we need to make the angle position dependent, our originally chosen parameterization \(\Br(s) = s (\cos\theta(s), \sin(\theta(s))\) is no longer
While we can still directly integrate the y-component equation once (as done above), our original assumed parameterization of \(\Br(s) = s(\cos\theta, \sin\theta)\) looses it's convenient form since we now have \(\theta = \theta(s)\) in the neighborhood of the origin.  Once we choose to not neglect the step discontinuity, we have a coupled, much more difficult, system to deal with.

Can this system, or one for which a limiting form of the unit step and delta functions is used (i.e. the sinc representation of the delta function), be solved exactly?

%\fxwarning{rework Snell's law problem using Ray optics}{I think I approached this part of the problem in a much too complicated way}.  Recall that we had an implicit assumption of continuity in our derivation of the Ray equation \cref{modernOpticsLecture2:pr2}.  Because of that the second order treatment is probably bogus.  In \S 3.2.2 of \citep{born1980principles} is a more careful argument.  It uses only the first order treatment, starting with the origin of the ray equation
%
\begin{equation}\label{eqn:modernOpticsProblemSet1P3:1970}
n \frac{d\Br}{ds} = \grad \phi.
\end{equation}
%
Because \(\spacegrad \cross \spacegrad \phi = 0\) we have
%
\begin{equation}\label{eqn:modernOpticsProblemSet1P3:1990}
\spacegrad \cross \left(n \frac{d\Br}{ds} \right) = 0,
\end{equation}
%
and can now apply a differential loop boundary condition to find that only the tangential component contributes, which leads to Snell's law.

\end{enumerate}
} % makeanswer

%\vcsinfo
%\EndArticle
%\EndNoBibArticle
