%
% Copyright © 2012 Peeter Joot.  All Rights Reserved.
% Licenced as described in the file LICENSE under the root directory of this GIT repository.
%

\makeoproblem{ABCD Matrices.}{modernOptics:problemSet1:1}{2012 Ps1, P1}{
Using the ABCD matrices from the lecture, prove these well-known rules of geometric optics. In each case, {\bf make an illustration}, tracing some important rays that illustrate the rule.
\begin{enumerate}
\item[(a)] {\em An image is formed when \(1/f = 1/s_o + 1/s_i\).} Solve this problem using the result we found in class: when B=0 for a system matrix, the input and output are conjugate planes.
\item[(b)] {\em An image with magnification \(-x'/f\) is formed when \(x x' = f^2\).} Repeat part (a), but in ``Newton's form'': replace \(s_o\) with \(f + x\), and replace \(s_i\) with \(f + x'\).
\item[(c)] {\em The position distribution at the focus of a lens is the angular position of the incident beam.} (In other words, a lens does a kind of Fourier transform, as you may know already.) Find where the input plane has to be located for \(y_o = f \alpha_i\).
\item[(d)] {\em Two identical lenses spaced by \(2f\) image an object at \(f\) with unity magnification.}
\item[(e)] Two identical lenses spaced by \(2f\) are {\em telecentric}, meaning that an object at \(f+x\) from the first lens has a magnification independent of \(x\), in contrast to a simple lens.
\item[(f)] A lens and a flat mirror spaced by distance \(f\) create a {\em cat's eye}. What are its properties? Consider, in particular, an emitter located \(f\) in front of the Cat's eye and located at \(y_i = 0\).
\end{enumerate}
} % makeoproblem

\makeanswer{modernOptics:problemSet1:1}{

\begin{enumerate}
\item[(a)]

Our system and the associated transfer matrices labels are illustrated in \cref{fig:modernOpticsProblemSet1:modernOpticsProblemSet1Fig1aTake2}.

\imageFigure{../figures/phy485-optics/modernOpticsProblemSet1Fig1aTake2}{Input and output conjugate planes for paraxial thin lens}{fig:modernOpticsProblemSet1:modernOpticsProblemSet1Fig1aTake2}{0.4}

We form the system transfer matrix by applying first a free propagation matrix, then a thin lens paraxial matrix, and one more free propagation matrix

\begin{dmath}\label{eqn:modernOpticsProblemSet1P1:1790}
M
= M_3 M_2 M_1
=
\begin{bmatrix}
1 & s' \\
0 1
\end{bmatrix}
\begin{bmatrix}
1 & 0 \\
-1/f & 1
\end{bmatrix}
\begin{bmatrix}
1 & s \\
0 1
\end{bmatrix}
=
\begin{bmatrix}
1 - s'/f & s' \\
-1/f & 1
\end{bmatrix}
\begin{bmatrix}
1 & s \\
0 1
\end{bmatrix}
=
\begin{bmatrix}
1 - s'/f & s + s' - s s'/f \\
-1/f & -s/f + 1
\end{bmatrix}.
\end{dmath}

Consider ray \((B)\) from the figure, where we have

\begin{dmath}\label{eqn:modernOpticsProblemSet1P1:1810}
\begin{bmatrix}
0 \\
\alpha
\end{bmatrix}
\rightarrow
\alpha
\begin{bmatrix}
s + s' - s s'/f \\
-s/f + 1
\end{bmatrix}
=
\begin{bmatrix}
0 \\
\alpha'
\end{bmatrix}.
\end{dmath}

With

\begin{equation}\label{eqn:modernOpticsProblemSet1P1:1830}
y = y' = \alpha ( s + s' - s s'/f ) = 0,
\end{equation}

for all \(\alpha\).  We must have

\begin{equation}\label{eqn:modernOpticsProblemSet1P1:1850}
s + s' = \frac{s s'}{f}.
\end{equation}

Dividing through by \(s s'\) we have

\boxedEquation{eqn:modernOpticsProblemSet1P1:1870}{
\inv{s'} + \inv{s} = \inv{f},
}

as expected.

\item[(b)]

Let's consider the system as the compound action of three transfer matrices as illustrated in \cref{fig:modernOpticsProblemSet1:modernOpticsProblemSet1Fig2b1}, this time labeling the figure in terms of the variables for this problem.

\imageFigure{../figures/phy485-optics/modernOpticsProblemSet1Fig2b1}{Newton's form, an image with magnification}{fig:modernOpticsProblemSet1:modernOpticsProblemSet1Fig2b1}{0.4}

Compounding the transfer matrices we have

\begin{dmath}\label{eqn:modernOpticsProblemSet1P1:1510}
M
= M_3 M_2 M_1
=
\begin{bmatrix}
1 & x' + f \\
0 & 1
\end{bmatrix}
\begin{bmatrix}
1 & 0 \\
-1/f & 1
\end{bmatrix}
\begin{bmatrix}
1 & x + f \\
0 & 1
\end{bmatrix}
=
\begin{bmatrix}
-x'/f & x' + f \\
-1/f & 1
\end{bmatrix}
\begin{bmatrix}
1 & x + f \\
0 & 1
\end{bmatrix}
=
-\inv{f}
\begin{bmatrix}
x' & x x' - f^2 \\
1 & x
\end{bmatrix}.
\end{dmath}

Consider the ray \(A\) where the effect is

\begin{dmath}\label{eqn:modernOpticsProblemSet1P1:1530}
\begin{bmatrix}
y \\
0
\end{bmatrix}
\rightarrow
-\inv{f}
\begin{bmatrix}
y x' \\
y
\end{bmatrix}.
\end{dmath}

We see that \(y' = -y x'/f\) or

\boxedEquation{eqn:modernOpticsProblemSet1P1:1550}{
m = - \frac{x'}{f} = \frac{y'}{y}.
}

The quantity defined as the magnification is in fact the ratio of the output to the input size as intuitively expected.  Now consider a ray \(C\) originating at \(y = 0\) at the image source, and landing at \(y = 0\) on the conjugate output plane.  For this ray we have

\begin{equation}\label{eqn:modernOpticsProblemSet1P1:1570}
\begin{bmatrix}
0 \\
\alpha
\end{bmatrix}
\rightarrow
-\inv{f}
\begin{bmatrix}
x x' -f^2 \\
x
\end{bmatrix} \theta
=
\begin{bmatrix}
0 \\
\alpha'
\end{bmatrix}.
\end{equation}

Since this holds for all input angles originating at \(y = 0\) from the input plane, we must have
\boxedEquation{eqn:modernOpticsProblemSet1P1:1590}{
x x' = f^2,
}

as desired.

\item[(c)]

Here we refer to \cref{fig:modernOpticsProblemSet1:modernOpticsProblemSet1Fig1c}, this time considering no ray that passes the focus past the lens.  Our system transfer matrix, given the reduced free propagation distance past the lens is
\imageFigure{../figures/phy485-optics/modernOpticsProblemSet1Fig1c}{Position distribution at the focus of a lens}{fig:modernOpticsProblemSet1:modernOpticsProblemSet1Fig1c}{0.3}

\begin{dmath}\label{eqn:modernOpticsProblemSet1P1:1610}
M
= M_3 M_2 M_1
=
\begin{bmatrix}
1 & f \\
0 & 1
\end{bmatrix}
\begin{bmatrix}
1 & 0 \\
-1/f & 1
\end{bmatrix}
\begin{bmatrix}
1 & x + f \\
0 & 1
\end{bmatrix}
=
\begin{bmatrix}
0 & f \\
-1/f & 1
\end{bmatrix}
\begin{bmatrix}
1 & x + f \\
0 & 1
\end{bmatrix}
=
\begin{bmatrix}
0 & f \\
-1/f & -x/f
\end{bmatrix}.
\end{dmath}

A ray is transformed according to
\begin{dmath}\label{eqn:modernOpticsProblemSet1P1:1630}
\begin{bmatrix}
y \\
\theta
\end{bmatrix}
\rightarrow
\begin{bmatrix}
0 & f \\
-1/f & -x/f
\end{bmatrix}
\begin{bmatrix}
y \\
\theta
\end{bmatrix}
=
\begin{bmatrix}
f \theta \\
-\inv{f} ( y - x \theta )
\end{bmatrix}.
\end{dmath}

In particular
\boxedEquation{eqn:modernOpticsProblemSet1P1:1650}{
y' = f \theta,
}

demonstrating the claim that at the focus, the position is an angular distribution of the incident beam.  This is clearly independent of \(x\) so the input plane position is irrelevant.

\item[(d)]

Consider \cref{fig:modernOpticsProblemSet1:modernOpticsProblemSet1Fig1d}.

\imageFigure{../figures/phy485-optics/modernOpticsProblemSet1Fig1d}{Two identical lenses separated by twice focus}{fig:modernOpticsProblemSet1:modernOpticsProblemSet1Fig1d}{0.2}

The transfer matrix \(M = M_5 M_4 M_3 M_2 M_1\) for the system is

\begin{dmath}\label{eqn:modernOpticsProblemSet1P1:1670}
M
= M_5 M_4 M_3 M_2 M_1
=
\begin{bmatrix}
1 & x + f \\
0 & 1
\end{bmatrix}
\begin{bmatrix}
1 & 0 \\
-1/f & 1
\end{bmatrix}
\begin{bmatrix}
1 & 2 f \\
0 & 1
\end{bmatrix}
\begin{bmatrix}
1 & 0 \\
-1/f & 1
\end{bmatrix}
\begin{bmatrix}
1 & x + f \\
0 & 1
\end{bmatrix}
=
\begin{bmatrix}
-x/f & x + f \\
-1/f & 1
\end{bmatrix}
\begin{bmatrix}
1 & 2 f \\
0 & 1
\end{bmatrix}
\begin{bmatrix}
1 & x + f \\
-1/f & -x/f
\end{bmatrix}
=
\begin{bmatrix}
-x/f & -x + f \\
-1/f & -1
\end{bmatrix}
\begin{bmatrix}
1 & x + f \\
-1/f & -x/f
\end{bmatrix}
=
\begin{bmatrix}
-1 & -2 x \\
0 & -1
\end{bmatrix}.
\end{dmath}

Consider any ray from the source going towards the lens along the horizontal.  We have

\begin{dmath}\label{eqn:modernOpticsProblemSet1P1:1690}
\begin{bmatrix}
y \\
0
\end{bmatrix}
\rightarrow
\begin{bmatrix}
-y \\
0
\end{bmatrix},
\end{dmath}

The ratio of the output to the input height to be

\boxedEquation{eqn:modernOpticsProblemSet1P1:1710}{
\frac{y'}{y} = -1,
}

which is the unit magnitude magnification as desired.

\item[(e)]

This is actually demonstrated above.

\item[(f)]

Here we consider \cref{fig:modernOpticsProblemSet1:modernOpticsProblemSet1Fig1f}.

\imageFigure{../figures/phy485-optics/modernOpticsProblemSet1Fig1f}{Cat's eye.  Lens with mirror behind at focus}{fig:modernOpticsProblemSet1:modernOpticsProblemSet1Fig1f}{0.4}

Our system transfer matrix is

\begin{dmath}\label{eqn:modernOpticsProblemSet1P1:1730}
M =
M_7
M_6
M_5
M_4
M_3
M_2
M_1
=
\begin{bmatrix}
1 & s' \\
0 & 1
\end{bmatrix}
\begin{bmatrix}
1 & 0 \\
-1/f & 1
\end{bmatrix}
\begin{bmatrix}
1 & f \\
0 & 1
\end{bmatrix}
\begin{bmatrix}
1 & 0 \\
0 & 1
\end{bmatrix}
\begin{bmatrix}
1 & f \\
0 & 1
\end{bmatrix}
\begin{bmatrix}
1 & 0 \\
-1/f & 1
\end{bmatrix}
\begin{bmatrix}
1 & s \\
0 & 1
\end{bmatrix}
=
\begin{bmatrix}
1 - s'/f & s' \\
-1/f & 1
\end{bmatrix}
\begin{bmatrix}
1 & 2 f \\
0 & 1
\end{bmatrix}
\begin{bmatrix}
1 & s \\
-1/f & -s/f + 1
\end{bmatrix}
=
\begin{bmatrix}
1 - s'/f & 2f - s' \\
-1/f & -1
\end{bmatrix}
\begin{bmatrix}
1 & s \\
-1/f & -s/f + 1
\end{bmatrix}
=
\begin{bmatrix}
-1 & 2 f - s' -s \\
0 & -1
\end{bmatrix}
\end{dmath}

We see that the angle of the output light is unchanged except for sign, so we have no scattering in the paraxial limit.  Observe that if the emitter is positioned at \(s = f\) we have

\begin{dmath}\label{eqn:modernOpticsProblemSet1P1:1750}
M =
\begin{bmatrix}
-1 & f - s' \\
0 & -1
\end{bmatrix}
\end{dmath}

so

\begin{dmath}\label{eqn:modernOpticsProblemSet1P1:1770}
y' = -y + (f -s') \alpha.
\end{dmath}

The image is magnified (negatively) for any position \(\Abs{s'} > f\) without any angular distortion.  In fact, if the observation is also made at the focus, then the image magnification is unity.  Notice that at the focus we have both a sign change in the position and the angle coordinate, meaning that the output image is exactly the same as in the input image.  In retrospect, this is exactly the same system mathematically as the \(2f\) spaced lenses of parts (d) and (e), and we could have done the matrix products just once for all those parts of the problem!

\end{enumerate}
} % makeanswer

