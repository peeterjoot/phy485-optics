%
% Copyright � 2012 Peeter Joot.  All Rights Reserved.
% Licenced as described in the file LICENSE under the root directory of this GIT repository.
%
%\input{../blogpost.tex}
%\renewcommand{\basename}{modernOpticsLecture14}
%\renewcommand{\dirname}{notes/phy485/}
%\newcommand{\keywords}{Optics, PHY485H1F}
%\input{../peeter_prologue_print2.tex}
%\beginArtNoToc
%\generatetitle{PHY485H1F Modern Optics.  Lecture 14: Cavity as an oscillator.  Taught by Prof.\ Joseph Thywissen}
%%\chapter{Cavity as an oscillator}
\index{cavity oscillator}
%\label{chap:modernOpticsLecture14}

\section{Fabry-Perot Etalon review.}
\index{Fabry-Perot Etalon}

We've been discussing a Fabry-Perot Etalon as in \cref{fig:modernOpticsLecture14:modernOpticsLecture14Fig2}.
\imageFigure{../figures/phy485-optics/modernOpticsLecture14Fig2}{Fabry-Perot Etalon.}{fig:modernOpticsLecture14:modernOpticsLecture14Fig2}{0.2}
%
with input that is of a discrete frequency (a spectral line).  We'll get something like \cref{fig:modernOpticsLecture14:modernOpticsLecture14Fig1}.
%
\imageFigure{../figures/phy485-optics/modernOpticsLecture14Fig1}{Etalon response by frequency.}{fig:modernOpticsLecture14:modernOpticsLecture14Fig1}{0.2}
%
something that is an idealization of \cref{fig:modernOpticsLecture14:modernOpticsLecture14Fig3}.
\imageFigure{../figures/phy485-optics/modernOpticsLecture14Fig3}{Ideal Etalon response.}{fig:modernOpticsLecture14:modernOpticsLecture14Fig3}{0.2}
%
Since we have widening due to smaller than ideal reflectivity.  It's not clear to me what the measurement mechanism here is.  We are plotting something against frequency, but only sending in discrete frequencies.

What would make sense to me is to consider the angular dependence of \(\Delta\) for two sets of frequencies.  Plotting that for a narrow range of angles for 500 \si{nm} and 650 \si{nm} light, \(L = 1\, \si{m}\), and \(R = 0.97\) we have \cref{fig:etalonAngularDependencyForTwoWavelengths:etalonAngularDependencyForTwoWavelengthsFig11}, where the first and third peaks are for 500 \si{nm}, and the second and fourth peaks for 650 \si{nm}.  So, if we have a source at a distance, we can expect a different intensity result at the output for different frequencies, depending on the angle between the source and the device.  This I can picture as an experimental setup.
%
\imageFigure{../figures/phy485-optics/etalonAngularDependencyForTwoWavelengthsFig11b}{Etalon angular dependencies.}{fig:etalonAngularDependencyForTwoWavelengths:etalonAngularDependencyForTwoWavelengthsFig11}{0.2}
%
\paragraph{Returning to the notes.}

We found two frequencies \(\overbar{\omega} \pm \Delta \omega/2\) are resolved when
%
\begin{dmath}\label{eqn:modernOpticsLecture14:10}
\frac{\Delta \omega}{\overbar{\omega}} = \inv{ m \calF }.
\end{dmath}
%
Here \(m = \text{order of interference}\) so
%
\begin{dmath}\label{eqn:modernOpticsLecture14:30}
\Delta = m 2 \pi.
\end{dmath}
%
\begin{dmath}\label{eqn:modernOpticsLecture14:50}
\calF = \pi \frac{\sqrt{R}}{1 - R} =
\text{Finess of Etalon}.
\end{dmath}
%
%\cref{fig:modernOpticsLecture14:modernOpticsLecture14Fig4}.
\imageFigure{../figures/phy485-optics/modernOpticsLecture14Fig4}{Wavelength packing in a cavity.}{fig:modernOpticsLecture14:modernOpticsLecture14Fig4}{0.2}
%
Example numbers
%
\begin{subequations}
\begin{dmath}\label{eqn:modernOpticsLecture14:70}
L = 10 \text{cm} \rightarrow \text{RST} = 10^{10} \text{s}^{-1} \text{or} 1.5 \text{GHz}.
\end{dmath}
\begin{dmath}\label{eqn:modernOpticsLecture14:90}
R = 97 \% \rightarrow F = 100
\end{dmath}
\begin{dmath}\label{eqn:modernOpticsLecture14:110}
\overbar{\lambda} \sim 0.6 \mu m \rightarrow \text{visible light}.
\end{dmath}
\begin{dmath}\label{eqn:modernOpticsLecture14:130}
\overbar{\omega} = 3 \times 10^{15} \text{s}^{-1}.
\end{dmath}
\begin{dmath}\label{eqn:modernOpticsLecture14:150}
\overbar{\nu} = 500 \text{THz}.
\end{dmath}
\begin{dmath}\label{eqn:modernOpticsLecture14:170}
\frac{\Delta \omega}{\overbar{\omega}} = 3 \times 10^{-8}.
\end{dmath}
\begin{dmath}\label{eqn:modernOpticsLecture14:190}
m = \frac{\overbar{\omega}}{\text{FSR}}
= \frac{3 \times 10^{15} s^{-1}}{ 10^10 s^{-1}} = 3 \times 10^{5}.
\end{dmath}
\end{subequations}
%
\(\Delta \omega\) is the smallest separation of two frequencies that we can measure.

\section{Cavity (or Etalon) (Fabry-Perot) as an oscillator.}
\index{oscillator}

Why are we talking so much about a specific interferometer, when this is a class on Advanced Classical Optics.  It turns out that the interaction with light in a cavity, as in a large setup \cref{fig:modernOpticsLecture14:modernOpticsLecture14Fig5}.
\imageFigure{../figures/phy485-optics/modernOpticsLecture14Fig5}{Cavity as an oscillator.}{fig:modernOpticsLecture14:modernOpticsLecture14Fig5}{0.2}
%
is basically the same idea as in an implementation of a laser \cref{fig:modernOpticsLecture14:modernOpticsLecture14Fig6}.
\imageFigure{../figures/phy485-optics/modernOpticsLecture14Fig6}{Semiconductor cavity.}{fig:modernOpticsLecture14:modernOpticsLecture14Fig6}{0.2}
%
If we are saying that something is an oscillator, then we can ask a couple questions:

\begin{itemize}
\item
What is the resonant frequency?
\item
What is the alignment?
\end{itemize}

The resonant frequency occurs every time that we can get an integer number of half wavelengths in the cavity.

We could actually ask what are the resonant frequencies, since we could have a ``comb'' of resonances \cref{fig:modernOpticsLecture14:modernOpticsLecture14Fig7}.
\imageFigure{../figures/phy485-optics/modernOpticsLecture14Fig7}{Frequency comb.}{fig:modernOpticsLecture14:modernOpticsLecture14Fig7}{0.2}
%
(transmission of the Etalon: see slides)

Answering our question of what are the resonant frequencies, our answer is
%
\begin{dmath}\label{eqn:modernOpticsLecture14:210}
\omega_m = (\text{offset}) + \text{FSR} m.
\end{dmath}
%
where \(m\) is an integer.  For the question of line width, consider a \underlineAndIndex{Lorenztian}
%
\begin{dmath}\label{eqn:modernOpticsLecture14:230}
\Gamma = \frac{\text{FSR}}{2 \calF}.
\end{dmath}
%
as in \cref{fig:modernOpticsLecture14:modernOpticsLecture14Fig8}.
\imageFigure{../figures/phy485-optics/modernOpticsLecture14Fig8}{Lorentzian.}{fig:modernOpticsLecture14:modernOpticsLecture14Fig8}{0.2}
%
We've got
%
\begin{dmath}\label{eqn:modernOpticsLecture14:250}
\frac{I}{I_0} = \inv{
1 + \frac{4 \calF^2}{\pi^2} \sin^2 \left( \Delta/2 \right)
}.
\end{dmath}
%
Consider the plot of \(\sin^2(\Delta/2)\) as in \cref{fig:modernOpticsLecture14:modernOpticsLecture14Fig9}.
%
\imageFigure{../figures/phy485-optics/modernOpticsLecture14Fig9}{Squared sine plot.}{fig:modernOpticsLecture14:modernOpticsLecture14Fig9}{0.2}
%
Our phase offset from the resonance is
%
\begin{dmath}\label{eqn:modernOpticsLecture14:270}
\Delta = 2 \pi m + \eta.
\end{dmath}
%
\begin{dmath}\label{eqn:modernOpticsLecture14:290}
\eta \ll 1.
\end{dmath}
%
In terms of \(\delta = \omega - \omega_m\)
%
\begin{dmath}\label{eqn:modernOpticsLecture14:310}
\eta = \frac{2 L}{c} \delta.
\end{dmath}
%
(because \(\Delta = \frac{2 L}{c} \omega + \text{offset}\)), we are left with
%
\begin{dmath}\label{eqn:modernOpticsLecture14:330}
\frac{I}{I_0}
= \inv{
1 + \frac{4 \calF^2}{\pi^2} \left( \frac{\eta}{2} \right)^2
}
= \inv{
1 + \left( \frac{ 2 L \calF \delta}{\pi c} \right)^2
}.
\end{dmath}
%
or
\boxedEquation{eqn:modernOpticsLecture14:350}{
\frac{I}{I_0}
= \inv{ 1 + \frac{\delta^2}{\Gamma^2}},
}

if
%
\begin{dmath}\label{eqn:modernOpticsLecture14:370}
\Gamma = \frac{ \pi c }{2 L \calF} = \frac{\text{FSR}}{ 2 \calF }.
\end{dmath}
%
What's the meaning of all of this?  It means that the Fabry-Perot oscillator is a device that traps light, and the resonance looks like a Lorentzian.

We need a very high Finess (high reflectivity) to get a good Lorentzian.

Recall that the Lorentzian is a Fourier transform of a damped exponential time domain signal \cref{fig:modernOpticsLecture14:modernOpticsLecture14Fig10}.
%
\imageFigure{../figures/phy485-optics/modernOpticsLecture14Fig10}{Exponential decay.}{fig:modernOpticsLecture14:modernOpticsLecture14Fig10}{0.2}
%
Light is trapped in the cavity for a time \(\tau \sim 1/\Gamma\).

%\EndArticle
%\EndNoBibArticle
