%
% Copyright � 2012 Peeter Joot.  All Rights Reserved.
% Licenced as described in the file LICENSE under the root directory of this GIT repository.
%
%\makeproblem{Full derivation of the Van Cittert-Zernike theorem}{modernOpticsMidtermReflection:pr:5}{
\section{Full derivation of the Van Cittert-Zernike theorem.}
\index{Van Cittert-Zernike theorem}

We never did the complete derivation of the \underlineAndIndex{Van Cittert-Zernike theorem} for spatial mutual coherence in class (or if we did I didn't understand it).  There were also aspects of the class notes derivation that I had trouble with.  Lets try this from scratch, going through all the details in sequence.

%}
%\makeanswer{modernOpticsMidtermReflection:pr:5}{

The geometry that we wish to consider is illustrated in \cref{fig:modernOpticsMidtermReflection:modernOpticsMidtermReflectionFig5}.
%
\imageFigure{../figures/phy485-optics/modernOpticsMidtermReflectionFig5}{Geometry for spatial coherence.}{fig:modernOpticsMidtermReflection:modernOpticsMidtermReflectionFig5}{0.2}
%
Our diffraction integral is
%
\begin{equation}\label{eqn:modernOpticsMidtermReflection:1040}
\Psi(\Br) = \int \Psi_s(\Br_s) \frac{e^{i k R}}{i\lambda R} d^2 r_s,
\end{equation}
%
with a correlation function of
%
\begin{equation}\label{eqn:modernOpticsMidtermReflection:1060}
\begin{aligned}
\Gamma_{12}
&= \expectation{\Psi(\Br_2, t) \Psi^\conj(\Br_1, t) } \\
&= \expectation{
\int d^2 r_s d^2 r_s'
\frac{e^{i k R_2(\Br_s')}}{i\lambda R_2(\Br_s')}
\frac{e^{-i k R_1(\Br_s)}}{-i\lambda R_1(\Br_s)}
\Psi_s(\Br_s', t)
\Psi_s^\conj(\Br_s, t)
}.
\end{aligned}
\end{equation}
%
where as in the figure we have
%
\begin{subequations}
\begin{equation}\label{eqn:modernOpticsMidtermReflection:1080}
\BR_2(\Br_s') + \Br_s' = \Br_2.
\end{equation}
\begin{equation}\label{eqn:modernOpticsMidtermReflection:1100}
\BR_1(\Br_s') + \Br_s = \Br_1.
\end{equation}
\end{subequations}
%
We can express an assumption that the wave sources at two different points are uncorrelated by writing
%
\begin{equation}\label{eqn:modernOpticsMidtermReflection:1120}
\expectation
{
\Psi_s(\Br_s', t)
\Psi_s^\conj(\Br_s, t)
}
= I(\Br_s) \delta(\Br_s - \Br_s').
\end{equation}
%
This makes some intuitive sense, but still seems sort of like it's been pulled from a magic hat.  I guess the idea is that if the source points are the same then the average of the autocorrelation is an integral of the intensity over all time (thus diverging), while for different points expressing no correlation over time.

Substitution of this delta function and integration over the \(d^2 r_s'\) source coordinates, leaves us
%
\begin{equation}\label{eqn:modernOpticsMidtermReflection:1140}
\Gamma_{12}
=
\inv{\lambda^2}
\int d^2 r_s
\frac{e^{i k R_2(\Br_s)}}{R_2(\Br_s)}
\frac{e^{-i k R_1(\Br_s)}}{R_1(\Br_s)}
I(\Br_s).
\end{equation}
%
We loose the dependence of \(R_1\) and \(R_2\) on the pair of source points and reformulate the \(R_2 - R_1\) difference in terms of the average distance between \(\Br_2\) and \(\Br_1\), plus the incremental distance between these.  That is
%
\begin{subequations}
\begin{equation}\label{eqn:modernOpticsMidtermReflection:1160}
\Br_2 = \Br_{\mathrm{av}} + \inv{2} \Delta \Br.
\end{equation}
\begin{equation}\label{eqn:modernOpticsMidtermReflection:1180}
\Br_1 = \Br_{\mathrm{av}} - \inv{2} \Delta \Br.
\end{equation}
\end{subequations}
%
so that
%
\begin{subequations}
\begin{equation}\label{eqn:modernOpticsMidtermReflection:1200}
\BR_2
= \Br_{\mathrm{av}} + \inv{2} \Delta \Br - \Br_s.
\end{equation}
\begin{equation}\label{eqn:modernOpticsMidtermReflection:1220}
\BR_1
= \Br_{\mathrm{av}} - \inv{2} \Delta \Br - \Br_s.
\end{equation}
\end{subequations}
%
These have magnitudes
%
\begin{equation}\label{eqn:modernOpticsMidtermReflection:1240}
\begin{aligned}
R_{2,1}
&= r_{\mathrm{av}}
\Abs{
\rcap_{\mathrm{av}} \pm \inv{2 r_{\mathrm{av}}} \Delta \Br - \inv{r_{\mathrm{av}}} \Br_s
}^{1/2} \\
&=
r_{\mathrm{av}}
\sqrt{
1 +
\left(
\pm \inv{2 r_{\mathrm{av}}} \Delta \Br - \inv{r_{\mathrm{av}}} \Br_s
\right)^2
+ 2 \rcap_{\mathrm{av}} \cdot
\left(
\pm \inv{2 r_{\mathrm{av}}} \Delta \Br - \inv{r_{\mathrm{av}}} \Br_s
\right)
} \\
&=
r_{\mathrm{av}}
\sqrt{
1 +
\frac{(\Delta \Br)^2}{4 r_{\mathrm{av}}^2}
+ \frac{r_s^2}{r_{\mathrm{av}}^2}
\mp \inv{r_{\mathrm{av}}} \Delta \Br \cdot \inv{r_{\mathrm{av}}} \Br_s
\pm \rcap_{\mathrm{av}} \cdot
\inv{r_{\mathrm{av}}} \Delta \Br
- \frac{2}{r_{\mathrm{av}}} \rcap_{\mathrm{av}} \cdot \Br_s
}.
\end{aligned}
\end{equation}
%
To first order, with the important parts highlighted, this is
%
\begin{equation}\label{eqn:modernOpticsMidtermReflection:1260}
R_{2,1} =
r_{\mathrm{av}} +
\frac{(\Delta \Br)^2}{8 r_{\mathrm{av}}}
+ \frac{r_s^2}{2 r_{\mathrm{av}}}
\mathLabelBox{
\mp \inv{2} \Delta \Br \cdot \inv{r_{\mathrm{av}}} \Br_s
\pm 2 \rcap_{\mathrm{av}} \cdot
\Delta \Br
}{Only these terms contribute to a difference}
-
\rcap_{\mathrm{av}} \cdot \Br_s.
\end{equation}
%
The difference is
\begin{equation}\label{eqn:modernOpticsMidtermReflection:1280}
R_{2} - R_{1} =
-
\Delta \Br \cdot \inv{r_{\mathrm{av}}} \Br_s
+ \rcap_{\mathrm{av}} \cdot
\Delta \Br
=
\Delta \Br
\cdot \left(
\rcap_{\mathrm{av}} - \inv{r_{\mathrm{av}}} \Br_s
\right).
\end{equation}
%
Our autocorrelation is now
%
\begin{equation}\label{eqn:modernOpticsMidtermReflection:1300}
\Gamma_{12}
=
\inv{\lambda^2}
e^{i k \Delta \Br \cdot \rcap_{\mathrm{av}}}
\int d^2 r_s
\frac{
I(\Br_s)
}{R_1(\Br_s) R_2(\Br_s)}
e^{-i k \Delta \Br \cdot \Br_s/r_{\mathrm{av}}}.
\end{equation}
%
It's been implied that the integration limits described the aperture.  Let's make that explicit with an aperture function \(g(\Br_s)\) so that we can allow the integration range to go to infinity in both directions without bound.  Let's also assume that the distances \(R_1\) and \(R_2\) don't vary much from their averages
%
\begin{equation}\label{eqn:modernOpticsMidtermReflection:1320}
\overbar{R_{2,1}} =
\frac{\int d^2 r_s g(\Br_s) R_{2,1}(\Br_s)}
{\int d^2 r_s g(\Br_s) }.
\end{equation}
%
so that the autocorrelation now takes the form
%
\begin{equation}\label{eqn:modernOpticsMidtermReflection:1340}
\Gamma_{12}
\sim
\inv{\lambda^2
\overbar{R_1}
\overbar{R_2}
}
e^{i k \Delta \Br \cdot \rcap_{\mathrm{av}}}
\int d^2 r_s
g(\Br_s) I(\Br_s)
e^{-i k \Delta \Br \cdot \Br_s/r_{\mathrm{av}}}.
\end{equation}
%
If both of the vectors \(\Br_s\) and the vector \(\Delta \Br\) lie in the same plane, then the autocorrelation is found to be the Fourier transform of \(g(\Br_s) I(\Br_s)\) evaluated at \(k \Delta \Br/r_{\mathrm{av}}\).

%} % makeanswer pr:5
