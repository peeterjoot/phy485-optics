%
% Copyright � 2012 Peeter Joot.  All Rights Reserved.
% Licenced as described in the file LICENSE under the root directory of this GIT repository.
%
%\input{../blogpost.tex}
%\renewcommand{\basename}{modernOpticsLecture6}
%\renewcommand{\dirname}{notes/phy485/}
%\newcommand{\keywords}{Optics, PHY485H1F}
%\input{../peeter_prologue_print2.tex}
%\beginArtNoToc
%\generatetitle{PHY485H1F Modern Optics.  Lecture 6: Fresnel diffraction.  Taught by Prof.\ Joseph Thywissen}
%\label{chap:modernOpticsLecture6}
%
%\section{Disclaimer}
%
%Peeter's lecture notes from class.  May not be entirely coherent.
%
\section{Fresnel diffraction from an edge}
\index{Fresnel diffraction!edge}

Consider the experiments illustrated in \cref{fig:modernOpticsLecture6:modernOpticsLecture6Fig1}, and \cref{fig:modernOpticsLecture6:modernOpticsLecture6Fig2}.
%
\imageFigure{../figures/phy485-optics/modernOpticsLecture6Fig1}{Intensity observed with no blockages.}{fig:modernOpticsLecture6:modernOpticsLecture6Fig1}{0.2}
\imageFigure{../figures/phy485-optics/modernOpticsLecture6Fig2}{Intensity observed with blockage just above line of sight.}{fig:modernOpticsLecture6:modernOpticsLecture6Fig2}{0.2}
%
Why, with such a carefully placed barrier, do we end up with \(I_0/4\)?  If we consider that the light takes all paths, and we have blocked half the paths, so that the amplitude of the wave function \(\Abs{\Psi_0/2}\) results in the factor of \(1/4\).  Let's do the math to see why this is the case more precisely.

We found
%
\begin{equation}\label{eqn:modernOpticsLecture6:10}
\Psi(\Br) = \frac{A}{i\lambda} \frac{e^{ik (r_s + r)}}{r_s r} \int_{\text{aperture}} e^{i k f(\Br')} da'
\end{equation}
%
In the Fraunhofer limit (the far field) we found \eqnref{eqn:modernOpticsLecture5:270} that
%
\begin{equation}\label{eqn:modernOpticsLecture6:30}
k f \rightarrow ( \Bk_s - \Bk ) \cdot \Br',
\end{equation}
%
where \(r \gg d^2/\lambda\), and \(d\) is a typical aperture size.  Recalling that the exact expression for \(f\) was
%
\begin{equation}\label{eqn:modernOpticsLecture6:50}
f(\Br') = - \left(\rcap + \rcap_s \right) \cdot \Br' + \inv{2r} \left( {\Br'}^2 - (\rcap \cdot \Br')^2 \right) + \inv{2 r_s} \left( (\Br')^2 - (\rcap_s \cdot \Br')^2 \right)
\end{equation}
%
We'll now consider the \underlineAndIndex{Fresnel} limit where \(\Bk_s = \Bk\), and
%
\begin{equation}\label{eqn:modernOpticsLecture6:70}
k f = \frac{k}{2} \left( \inv{r} + \inv{r_s} \right) {r'}^2
\end{equation}
%
These Fresnel terms are generally important when \(r \sim d^2/\lambda\) even if \(r \gg d\) (because \(\lambda \ll d\)).  We'd like to massage this \(k f\) expression
%
\begin{equation}\label{eqn:modernOpticsLecture6:90}
k f = \frac{k}{2} \left(
r_s^{-1}
+r^{-1} \right) \left({x'}^2 + {y'}^2 \right) = \frac{\pi}{2} \left( u^2 + v^2 \right),
\end{equation}
%
where we have made a change of variables
%
\begin{dmath}\label{eqn:modernOpticsLecture6:110}
\begin{bmatrix}
x' \\
y'
\end{bmatrix}
=
\sqrt{\frac{\pi/k}{r_s^{-1} + r^{-1}}}
\begin{bmatrix}
u \\
v
\end{bmatrix}
=
\sqrt{\frac{\lambda/2}{r_s^{-1} + r^{-1}}}
\begin{bmatrix}
u \\
v
\end{bmatrix}
\end{dmath}
%
Our area element is then
%
\begin{equation}\label{eqn:modernOpticsLecture6:130}
dx' dy'  =
\frac{\lambda/2}{r_s^{-1} + r^{-1}} du dv
\end{equation}
%
Our integral is now
%
\begin{dmath}\label{eqn:modernOpticsLecture6:150}
\Psi(\Br)
= \frac{A}{i\lambda} \frac{e^{ik (r_s + r)}}{r_s r} \int_{\text{aperture}} e^{i k f(\Br')} da'
= \frac{A}{i\lambda} \frac{e^{ik (r_s + r)}}{r_s r}
\frac{\lambda/2}{r_s^{-1} + r^{-1}}
\int_{\text{aperture}} e^{i \frac{\pi}{2} (u^2 + v^2)} du dv
=
\frac{A}{i\lambda} e^{ik (r_s + r)}
\frac{\lambda/2}{r_s + r}
\int_{\text{aperture}} e^{i \frac{\pi}{2} (u^2 + v^2)} du dv
\end{dmath}
%
Referring to \cref{fig:modernOpticsLecture6:modernOpticsLecture6Fig3}, let's do this integral.  Putting in our limits we have
%
\imageFigure{../figures/phy485-optics/modernOpticsLecture6Fig3}{Region of integration.}{fig:modernOpticsLecture6:modernOpticsLecture6Fig3}{0.3}
%
\begin{equation}\label{eqn:modernOpticsLecture6:170}
\Psi(\Br)
=
\frac{A}{ 2 i }
\frac{
e^{ik (r_s + r)}
}{r_s + r}
\int_{-\infty}^\infty du \int_{-\infty}^w dv
e^{i \frac{\pi}{2} \left(u^2 + v^2 \right) } du dv
\end{equation}
%
where
%
\begin{equation}\label{eqn:modernOpticsLecture6:190}
w = \sqrt{ \frac{2}{\lambda} \left( \inv{r_s} + \inv{r} \right) }
\end{equation}
%
Evaluating \(\int -z^2 dz\) over a pizza contour it can be demonstrated \citep{planetmath:fresnel} that

% Prof T. also mentioned use of Gaussian's ... was probably referring to this pizza contour.
\begin{equation}\label{eqn:modernOpticsLecture6:210}
\int_{-\infty}^\infty dv e^{i \frac{\pi}{2} v^2 } = 1 + i = \sqrt{2} e^{i \pi/4}
\end{equation}
%
\begin{equation}\label{eqn:modernOpticsLecture6:230}
\int_{-\infty}^w dv e^{i \frac{\pi}{2} v^2 }
=
\int_{-\infty}^0
\int_{0}^w
dv e^{i \frac{\pi}{2} v^2 }
=
\frac{1 + i}{2} + C(w) + S(w),
\end{equation}
%
where
%
\begin{subequations}
\begin{equation}\label{eqn:modernOpticsLecture6:250}
S(w) = \int_0^w \sin \left( \frac{\pi}{2} u^2 \right) du
\end{equation}
\begin{equation}\label{eqn:modernOpticsLecture6:270}
C(w) = \int_0^w \cos \left( \frac{\pi}{2} u^2 \right) du
\end{equation}
\end{subequations}
%
Parametrically plotting these we get the \underlineAndIndex{Cornu Spiral} as plotted in \cref{fig:modernOpticsLecture6:modernOpticsLecture6Fig4}.
%
\imageFigure{../figures/phy485-optics/modernOpticsProblemSet2Problem2CornuSpiralPlainPlotFig4a}{Cornu Spiral.}{fig:modernOpticsLecture6:modernOpticsLecture6Fig4}{0.3}
% origin figure: (generated mathematica figure for ps2)
%\imageFigure{../figures/phy485-optics/modernOpticsLecture6Fig4}{Cornu spiral (rough illustration).}{fig:modernOpticsLecture6:modernOpticsLecture6Fig4}{0.3}
%
There are some interesting features of this curve.

\begin{enumerate}
\item
The length along the curve is
%
\begin{dmath}\label{eqn:modernOpticsLecture6:290}
dl^2
= dS^2 + dC^2
=
\left( \frac{dS}{dw} \right)^2
+\left( \frac{dC}{dw} \right)^2
=
\left(
\sin^2\left( \frac{\pi}{2} w^2 \right)
+\cos^2\left( \frac{\pi}{2} w^2 \right)
\right) dw^2
= dw^2
\end{dmath}
%
so that

\boxedEquation{eqn:modernOpticsLecture6:310}{
dl = dw
}

\item How about the angle along the curve.  Stating the result, where the angle is given by

\boxedEquation{eqn:modernOpticsLecture6:330}{
\tan\theta = \frac{dy}{dx}
}

one can find that
%
\begin{dmath}\label{eqn:modernOpticsLecture6:350}
\theta = \frac{\pi}{2} w^2
\end{dmath}
%
\end{enumerate}

Going back to our diffraction integral we find
%
\begin{equation}\label{eqn:modernOpticsLecture6:370}
\Psi(\Br)
=
\frac{A}{ 2 i }
\frac{
e^{ik (r_s + r)}
}{r_s + r}
\left( 1 + i \right) \left( \frac{1 + i}{2} + C(w) + S(w) \right)
\end{equation}
%
\paragraph{Check: No obstruction?}

We've got \(w \rightarrow \infty\), so that \(C(w) = 1/2\) and \(S(w) = 1/2\).  This gives us
%
\begin{equation}\label{eqn:modernOpticsLecture6:390}
\Psi(\Br)
=
A
\frac{
e^{ik (r_s + r)}
}{r_s + r}
\equiv \Psi_\infty(\Br).
\end{equation}
%
Now let's consider our obstruction right along the line of sight (\(w = 0\)).  Now we have, since \(C(0) = S(0) = 0\)
%
\begin{equation}\label{eqn:modernOpticsLecture6:410}
\Psi(\Br)
=
\Psi_\infty(\Br)
\inv{2 i} (1 + i) \left( \frac{1 + i}{2} + 0 \right)
=
\inv{2} \Psi_\infty(\Br).
\end{equation}
%
We do see that we end up with half the amplitude, so that as claimed our intensity (which squares the amplitude) results in a factor of \(1/4\)

In general, for a barrier offset by \(d\), and a value of \(w\) that corresponds to that, our \underlineAndIndex{Intensity} is
%
\begin{dmath}\label{eqn:modernOpticsLecture6:430}
I
= \Abs{\Psi}^2
= \Abs{\Psi_\infty}^2 \inv{2} \left(
\left(
\inv{2} + C(w)
\right)^2
+
\left(
\inv{2} + S(w)
\right)^2
\right)
\end{dmath}
%
\paragraph{Check, again with \(w = 0\)}, we have
%
\begin{dmath}\label{eqn:modernOpticsLecture6:450}
I_\infty \inv{2} \left( \inv{4} + \inv{4} \right) = \frac{I_\infty}{4}.
\end{dmath}
%
Other examples.

Diffraction from an edge \(w = d\): \cref{fig:modernOpticsLecture6:modernOpticsLecture6Fig5}.
%
\imageFigure{../figures/phy485-optics/modernOpticsLecture6Fig5}{Diffraction spectrum with partial blockage above line of sight (brutally rough illustration).}{fig:modernOpticsLecture6:modernOpticsLecture6Fig5}{0.3}
%
Poisson spot.  Poisson crafted a counter argument for the wave theory of light stating that if it was true, then you should be able to see a spot behind a circular blockage, as if some of the light was going around the blockage.  This is illustrated in \cref{fig:modernOpticsLecture6:modernOpticsLecture6Fig6}, and can in fact be observed with the right setup.
%
\imageFigure{../figures/phy485-optics/modernOpticsLecture6Fig6}{Poisson spot.}{fig:modernOpticsLecture6:modernOpticsLecture6Fig6}{0.3}
%
Once we understand that light does in fact take all the paths, we can utilize this to build a \underlineAndIndex{Fresnel lens} by blocking selectively as illustrated very roughly in \cref{fig:modernOpticsLecture6:modernOpticsLecture6Fig7}.
%
\imageFigure{../figures/phy485-optics/modernOpticsLecture6Fig7}{Diffraction grating (imagined to have been constructed to focus x-rays).}{fig:modernOpticsLecture6:modernOpticsLecture6Fig7}{0.3}
%
A diffraction setup like allows us block all the portions of the phase that negatively interfere.  This can be used for example to focus x-rays.  That application will be explored in more detail in the problem set.

%\vcsinfo
%\EndArticle
%\EndNoBibArticle
