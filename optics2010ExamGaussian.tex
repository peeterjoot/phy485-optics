%
% Copyright � 2012 Peeter Joot.  All Rights Reserved.
% Licenced as described in the file LICENSE under the root directory of this GIT repository.
%
% pick one:
%\input{../assignment.tex}
%\input{../blogpost.tex}
%\renewcommand{\basename}{optics2010ExamGaussian}
%\renewcommand{\dirname}{notes/phy485/}
%\newcommand{\dateintitle}{}
%\newcommand{\keywords}{}
%
%\input{../peeter_prologue_print2.tex}
%
%\beginArtNoToc
%
%\generatetitle{FIXME put title here}
%\chapter{FIXME put title here}
%\label{chap:\basename}
%\section{Motivation}
%\section{Guts}

\makeoproblem{Gaussian beam.}
{pr:optics2010ExamGaussian:1}
{2010 final exam question 5}
{

\makesubproblem{Spot size}{modernOptics:optics2010ExamGaussian:a}

A Gaussian beam with wavelength \(0.8 \mu \mbox{m}\) has its minimum waist of \(0.5 \mbox{mm}\) located in the middle of a parallel glass plate of thickness \(5 \mbox{cm}\) and refractive index \(1.5\).  The axis of the beam is perpendicular to the surfaces of the glass plate.  The beam emerges from the glass plate and strikes a mirror at normal incidence \(10 \mbox{cm}\) away.  When the beam passes back through its original location what is its spot size?

\makesubproblem{Angular divergence}{modernOptics:optics2010ExamGaussian:b}

When the beam emerges from the plate again after re-passing its beam waist, what is its angular divergence?

} % makeoproblem

\makeanswer{pr:optics2010ExamGaussian:1}{

\makeSubAnswer{Spot size}{modernOptics:optics2010ExamGaussian:a}

Our optical system and beam has the following configuration \cref{fig:optics2010ExamGaussian:optics2010ExamGaussianFig1}.
%
\imageFigure{../figures/phy485-optics/optics2010ExamGaussianFig1}{Gaussian beam through glass then air.}{fig:optics2010ExamGaussian:optics2010ExamGaussianFig1}{0.3}
%
Going from the glass to the air we have \(n \sin\theta_i = \sin\theta_t\), or in the paraxial approximation
%
\begin{dmath}\label{eqn:optics2010ExamGaussian:20}
\begin{bmatrix}
y_t \\
\theta_t
\end{bmatrix}
=
\begin{bmatrix}
1 & 0 \\
0 & n
\end{bmatrix}
\begin{bmatrix}
y_i \\
\theta_i
\end{bmatrix}
\end{dmath}
%
The geometric optics for the round trip is
%
\begin{dmath}\label{eqn:optics2010ExamGaussian:40}
M =
\begin{bmatrix}
1 & L/2 \\
0 & 1
\end{bmatrix}
\begin{bmatrix}
1 & 0 \\
0 & 1/n
\end{bmatrix}
\begin{bmatrix}
1 & 2 D \\
0 & 1
\end{bmatrix}
\begin{bmatrix}
1 & 0 \\
0 & n
\end{bmatrix}
\begin{bmatrix}
1 & L/2 \\
0 & 1
\end{bmatrix}
=
\begin{bmatrix}
1 & L/2n \\
0 & 1/n
\end{bmatrix}
\begin{bmatrix}
1 & 2 D n \\
0 & n
\end{bmatrix}
\begin{bmatrix}
1 & L/2 \\
0 & 1
\end{bmatrix}
=
\begin{bmatrix}
1 & L/2n \\
0 & 1/n
\end{bmatrix}
\begin{bmatrix}
1 & L/2 + 2 D n \\
0 & n
\end{bmatrix}
=
\begin{bmatrix}
1 & L + 2 D n \\
0 & 1
\end{bmatrix}.
\end{dmath}
%
Our M\"obius transformation, in meters is
%
\begin{dmath}\label{eqn:optics2010ExamGaussian:60}
q'
= q + 2 D n
= q + 2 (0.1) 1.5
= q + 0.3.
\end{dmath}
%
In the inverse, this has the form
%
\begin{dmath}\label{eqn:optics2010ExamGaussian:80}
\inv{q + a}
= \inv{z - i z_0 + a}
= \frac{z + i z_0 + a}
{(z + a)^2 + z_0^2 }
=
\frac{z + a}
{(z + a)^2 + z_0^2 }
+ i \frac{z_0}
{(z + a)^2 + z_0^2 }.
\end{dmath}
%
Our waist \(z = 0\) was originally
%
\begin{dmath}\label{eqn:optics2010ExamGaussian:100}
\frac{\lambda_0}{\pi n w_0^2} = \frac{z_0}{z_0^2 } = \inv{z_0},
\end{dmath}
%
or
%
\begin{dmath}\label{eqn:optics2010ExamGaussian:120}
z_0 = \frac{ \pi n w_0^2}{\lambda_0}.
\end{dmath}
%
Assuming that we are given the wavelength within the glass \(\lambda = \lambda_0/n = 0.8 \mu \mbox{m}\) (and not the free propagation wavelength outside of the glass), then we have in meters
%
\begin{dmath}\label{eqn:optics2010ExamGaussian:140}
z_0 = \frac{ \pi (0.0005)^2}{0.8 \times 10^{-6} }
\approx 0.98.
\end{dmath}
%
Our new waist is
%
\begin{dmath}\label{eqn:optics2010ExamGaussian:160}
{w_0'}^2
= \frac{\pi}{\lambda} \frac{(0 - 0.3)^2 + z_0^2}{z_0}
\approx 0.52 \mbox{mm}.
\end{dmath}
%
Our waist widens slightly from the original after the round trip.

\makeSubAnswer{Angular divergence}{modernOptics:optics2010ExamGaussian:b}

To examine the beam characteristics after it continues through and out of the glass again, we have to apply another geometric transformation
%
\begin{dmath}\label{eqn:optics2010ExamGaussian:180}
M' =
\begin{bmatrix}
1 & 0 \\
0 & n
\end{bmatrix}
\begin{bmatrix}
1 & L/2 \\
0 & 1
\end{bmatrix}
=
\begin{bmatrix}
1 & L/2 \\
0 & n
\end{bmatrix}.
\end{dmath}
%
Our M\"obius transform is
%
\begin{dmath}\label{eqn:optics2010ExamGaussian:240}
q''
= \frac{ q' + L/2 }{n}
= \frac{ q + 2 D n + L/2 }{n},
\end{dmath}
%
We are looking at how our initial waist, found at \(z = 0\) transformed, so we have
%
\begin{dmath}\label{eqn:optics2010ExamGaussian:260}
\inv{ q''(0) }
= \frac{n}{ -i z_0 + 2 D n + L/2 }
= \frac{n( i z_0 + 2 D n + L/2)}{ z_0^2 + (2 D n + L/2)^2 }.
\end{dmath}
%
Our new waist is found outside the glass where \(n = 1\)
%
\begin{dmath}\label{eqn:optics2010ExamGaussian:280}
\Imag\left( \inv{ q''(0) } \right)
=
\frac{\lambda_0}{n_{\mathrm{air}}\pi {w'}^2(0)}
=
\frac{\lambda_0}{\pi {w'}^2(0)}
=
\frac{n_{\mathrm{glass}} \lambda}{\pi {w'}^2(0)}.
\end{dmath}
%
\begin{dmath}\label{eqn:optics2010ExamGaussian:200}
{w'}^2(0)
=
\frac{n_{\mathrm{glass}} \lambda}{\pi}
\frac{ z_0^2 + (2 D n_{\mathrm{glass}} + L/2)^2 } {n_{\mathrm{glass}} z_0 }.
\end{dmath}
%
With the divergence angle having the value
%
\begin{dmath}\label{eqn:optics2010ExamGaussian:220}
\Theta
\sim \frac{\sqrt{2} w_0'}{z_0'}
= \sqrt{2} \sqrt{
\frac{\lambda}{\pi} \frac{ z_0^2 + ( 2 D n_{\mathrm{glass}} + L/2 )^2}{z_0}
}
\frac{n_{\mathrm{glass}} z_0}{ z_0^2 + (2 D n_{\mathrm{glass}} + L/2)^2}
= \sqrt{
\frac{2 \lambda z_0}{\pi }
\frac{n_{\mathrm{glass}}^2}
{ z_0^2 + ( 2 D n_{\mathrm{glass}} + L/2 )^2}
}
= \sqrt{
\frac{2 \lambda \frac{\pi n_{\mathrm{glass}} w_0^2}{\lambda}}{\pi }
\frac{n_{\mathrm{glass}}^2}
{ \left( \frac{\pi n_{\mathrm{glass}} w_0^2}{\lambda} \right)^2 + ( 2 D n_{\mathrm{glass}} + L/2 )^2}
}
= \sqrt{
\frac{2 n_{\mathrm{glass}}^3 w_0^2}
{ \left( \frac{\pi n_{\mathrm{glass}} w_0^2}{\lambda} \right)^2 + ( 2 D n_{\mathrm{glass}} + L/2 )^2}
}
= \sqrt{
\frac{2 n_{\mathrm{glass}} w_0^2}
{ \frac{\pi^2 w_0^4}{\lambda^2} + \left( 2 D +
\frac{L}{2 n_{\mathrm{glass}}^2 } \right)^2}
}.
\end{dmath}
%
Plugging in the numbers, this is \(8.6 \times 10^{-4}\) radians or \(0.05\) seconds. That's a small seeming number, but still results in a \(1\) cm spread after only 5.8 meters.

} % makeanswer

%\EndNoBibArticle
