%
% Copyright � 2012 Peeter Joot.  All Rights Reserved.
% Licenced as described in the file LICENSE under the root directory of this GIT repository.
%
%\input{../blogpost.tex}
%\renewcommand{\basename}{modernOpticsLecture5}
%\renewcommand{\dirname}{notes/phy485/}
%\newcommand{\keywords}{Optics, PHY485H1F, Fraunhofer diffraction, Fresnel diffraction}
%\input{../peeter_prologue_print2.tex}
%\beginArtNoToc
%\generatetitle{PHY485H1F Modern Optics.  Lecture 5: Diffraction (cont.).  Taught by Prof.\ Joseph Thywissen}
%\chapter{Diffraction (cont.)}
%\index{diffraction}
%\label{chap:modernOpticsLecture5}

%\section{Disclaimer}
%
%Peeter's lecture notes from class.  May not be entirely coherent.
%
\section{Fresnel and Fraunhofer diffraction.}
\index{diffraction!Fresnel}
\index{diffraction!Fraunhofer}
Last time we got as far as finding
\begin{dmath}\label{eqn:modernOpticsLecture5:10}
\Psi(\Br) = -\inv{4\pi} \int da' \frac{e^{i k R}}{R} \left(
\spacegrad' \Psi' + \left( i k - \inv{R} \right) \Rcap \Psi
\right) \cdot \ncap.
\end{dmath}
%
We want to consider non-pinhole apertures.

\paragraph{Length scales}
\begin{itemize}
\item Wavelength \(\lambda = 2 \pi/k\)
\item Object size \(d\), where the object is larger than \(\lambda\)
\item Distance of observation \(r \gg d \gg \lambda\)
\item Sources \(r_s \gg d \gg \lambda\)
\end{itemize}

Observe that \(r \gg \lambda\) we have
%
\begin{dmath}\label{eqn:modernOpticsLecture5:50}
\left( i k - \inv{r} \right) = \frac{2 \pi i}{\lambda} \left( 1 - \frac{\lambda}{r} \right) \approx i k,
\end{dmath}
%
Another consequence is that in an integral like
%
\begin{dmath}\label{eqn:modernOpticsLecture5:70}
\int \frac{ \cdots }{ \Abs{ \Br_s - \Br' } } da' \approx \inv{ R_s} \int \cdots da',
\end{dmath}
%
because \(r \gg d\).

We still have to be \textunderline{careful} with something like
%
\begin{dmath}\label{eqn:modernOpticsLecture5:90}
\int \cdots e^{i k {\Br'}^2/r}.
\end{dmath}
%
since this exponential could matter very much.  Recalling that \(\BR = \Br - \Br'\) and expanding in Taylor series to second order
%\footnote{I've used \(1/\Br\) here, which is a valid construction in Geometric Algebra. Assuming that this notational freedom is not familiar, this can just be thought of as a shorthand for \(1/\Br = \rcap/r\).}, we have
%
\begin{dmath}\label{eqn:modernOpticsLecture5:430}
R
= \sqrt{\BR^2}
= \sqrt{ (\Br - \Br')^2 }
= \sqrt{ \Br^2 + {\Br'}^2 - 2 \Br \cdot \Br' }
= r \sqrt{ 1 + \frac{{\Br'}^2}{\Br^2} - 2 \frac{\rcap}{r} \cdot \Br' }
\approx
r
\left(
1 + \inv{2}
\left( \frac{{\Br'}^2}{\Br^2} - 2 \frac{\rcap}{r} \cdot \Br' \right)
-\inv{8} \left( \frac{{\Br'}^2}{\Br^2} - 2 \frac{\rcap}{r} \cdot \Br' \right)^2
+ \cdots
\right)
=
r
\left(
1
+ \inv{2} \frac{{\Br'}^2}{\Br^2} - \frac{\rcap}{r} \cdot \Br'
-\inv{8} \frac{{\Br'}^4}{\Br^4}
-\inv{2} \left( \frac{\rcap}{r} \cdot \Br' \right)^2
+ \inv{2} \frac{{\Br'}^2}{\Br^2} \frac{\rcap}{r} \cdot \Br'
+ \cdots
\right)
=
r
\left(
1
+ \inv{2 r^2} {r'}^2 - \frac{\rcap}{r} \cdot \Br'
-\inv{8 r^4} {r'}^4
-\inv{2 r^2} \left( \rcap \cdot \Br' \right)^2
+ \inv{2 r^3} {r'}^2 \rcap \cdot \Br'
+ \cdots
\right).
\end{dmath}
%
Grouping by order of significance we have
%
\begin{dmath}\label{eqn:modernOpticsLecture5:110}
k R = \mathLabelBox{k r}{\(O(r\lambda)\)} - \mathLabelBox{k \rcap \cdot \Br'}{\(O(d/\lambda)\)} + \mathLabelBox{\frac{k}{2r} \left( {\Br'}^2 - \left( \rcap \cdot \Br' \right)^2 \right)}{\(O(d^2/\lambda r)\)} + \mathLabelBox{\cdots}{\(O(d^3/\lambda r^2)\)}.
\end{dmath}
%
We need any exponential to be small with respect to \(1\) to neglect.
\begin{itemize}
\item In the above if \(r \gg d^2/\lambda\) we can neglect this third term, which is the \underlineAndIndex{Fraunhofer} case.
\item \underlineAndIndex{Fresnel} diffraction retains the \({\Br'}^2\) term! (with \(r^2 \gg d^3/\lambda\), we'll stop at this \({\Br'}^2\) term).
\end{itemize}

We'll treat the Fraunhofer case now killing the \(1/R\) term here:
%
\begin{dmath}\label{eqn:modernOpticsLecture5:130}
\Psi(\Br) = -\inv{4\pi} \int da' \frac{e^{i k R}}{R} \left(
\spacegrad' \Psi' + \left( i k - \cancel{\inv{R}} \right) \Rcap \Psi
\right) \cdot \ncap.
\end{dmath}
%
We'll use a point source
%
\begin{dmath}\label{eqn:modernOpticsLecture5:150}
\Psi(\Br') = \frac{e^{i k R_s }}{R_s}.
\end{dmath}
%
where
%
\begin{dmath}\label{eqn:modernOpticsLecture5:170}
\spacegrad' \Psi'
= -\Rcap_s \left( i k - \cancel{\inv{R_s}} \right) \frac{e^{i k R_s}}{R_s}
\approx \ncap i k \frac{e^{i k R_s}}{R_s}.
\end{dmath}
%
Our diffraction integral becomes
%
\begin{dmath}\label{eqn:modernOpticsLecture5:130b}
\Psi(\Br) = -\frac{i k}{4\pi} \int da'
\frac{e^{i k R}}{R}
\frac{e^{i k R_s}}{R_s}
\left( 1 + \ncap \cdot \BR \right).
\end{dmath}
%
With a small enough object \(d \ll r, r_s\), and writing \(k/2\pi = 1/\lambda\), we'll be able to pull the obliquity factor out of the integral
%
\begin{dmath}\label{eqn:modernOpticsLecture5:190}
\Psi(\Br)
= \frac{ A }{\lambda i} \iint da' \frac{e^{i k (R + R_s)}}{R_s R} k(\theta)
\approx \frac{ A k(\theta) }{\lambda i} \iint da' \frac{e^{i k (R + R_s)}}{R_s R}
\approx \frac{ A }{\lambda i} \iint da' \frac{e^{i k (R + R_s)}}{R_s R}.
\end{dmath}
%
We've also made the paraxial approximation, recalling that
%
\begin{dmath}\label{eqn:modernOpticsLecture5:210}
k(\theta) = \frac{1 + \cos\theta}{2},
\end{dmath}
%
so that for \(\theta \approx 0\), in the region illustrated in \cref{fig:modernOpticsLecture5:modernOpticsLecture5Fig1} and \cref{fig:modernOpticsLecture5:modernOpticsLecture5Fig2}.
\imageFigure{../figures/phy485-optics/modernOpticsLecture5Fig1}{Obliquity factor.}{fig:modernOpticsLecture5:modernOpticsLecture5Fig1}{0.2}
\imageFigure{../figures/phy485-optics/modernOpticsLecture5Fig2}{Obliquity factor.}{fig:modernOpticsLecture5:modernOpticsLecture5Fig2}{0.2}
%
we have \(k(\theta) \approx 1\).

Our problem is now reduced to
%
\begin{dmath}\label{eqn:modernOpticsLecture5:230}
\Psi(\Br)
= \frac{ A }{\lambda i} \frac{ e^{i k (r_s + r)}}{r_s r} \iint_{\text{aperture}} da' e^{i k f(\Br')}.
\end{dmath}
%
where
%
\begin{equation}\label{eqn:modernOpticsLecture5:250}
\begin{aligned}
f(\Br') &= 
- \left(\rcap + \rcap_s \right) \cdot \Br'  \\
&\quad + \inv{2r} \left( {\Br'}^2 - (\rcap \cdot \Br')^2 \right) + \inv{2 r_s} \left( (\Br')^2 - (\rcap_s \cdot \Br')^2 \right).
\end{aligned}
\end{equation}
%
the first term is the Fraunhofer term and the last two are the Fresnel contributions.

Referring to \cref{fig:modernOpticsLecture5:modernOpticsLecture5Fig3} we find
\imageFigure{../figures/phy485-optics/modernOpticsLecture5Fig3}{Defining k vectors.}{fig:modernOpticsLecture5:modernOpticsLecture5Fig3}{0.2}
%
\begin{dmath}\label{eqn:modernOpticsLecture5:270}
k f
= -k (\rcap + \rcap_s) \cdot \Br'
= -(\Bk -\Bk_s) \cdot \Br'
= (\Bk_s -\Bk) \cdot \Br'.
\end{dmath}
%
putting things back into the diffraction integral, we have something of the form
%
\begin{dmath}\label{eqn:modernOpticsLecture5:290}
\Psi(\Br)
= \text{constant} \iint_{\text{aperture}} d^2 \Br' e^{i (\Bk - \Bk_s) \cdot \Br' } g(\Br').
\end{dmath}
%
where \(g(\Br')\) is an ``aperture'' function defined in the open portion as illustrated in \cref{fig:modernOpticsLecture5:modernOpticsLecture5Fig4}.
%
\imageFigure{../figures/phy485-optics/modernOpticsLecture5Fig4}{Circular aperture.}{fig:modernOpticsLecture5:modernOpticsLecture5Fig4}{0.2}
%
If we define \(g(\Br')\) to be zero outside of the aperture
%
\begin{dmath}\label{eqn:modernOpticsLecture5:305}
g(\Br') =
\left\{
\begin{array}{l l}
1 & \quad \mbox{open} \\
0 & \quad \mbox{blocked}
\end{array}
\right.
\end{dmath}
%
then we can just write
%
\begin{dmath}\label{eqn:modernOpticsLecture5:310}
\Psi(\Br)
= \text{constant} \iint d^2 \Br' e^{i (\Bk - \Bk_s) \cdot \Br' } g(\Br').
\end{dmath}
%
so that
%
\begin{dmath}\label{eqn:modernOpticsLecture5:330}
\Psi = (\text{constant}) G(\Bk_s - \Bk).
\end{dmath}
%
where
%
\begin{dmath}\label{eqn:modernOpticsLecture5:350}
G(\Bk) = \iint e^{-i \Bk \cdot \Br'} g(\Br') d^2 \Br',
\end{dmath}
%
which is just a Fourier transform!  Our amplitude is
%
\begin{dmath}\label{eqn:modernOpticsLecture5:370}
I(\Br) = \Abs{\Psi(\Br)}^2 = (\text{constant})^2 \Abs{G(\Bk_s - \Bk)}^2.
\end{dmath}
%
Note that if the amplitude
%
\begin{dmath}\label{eqn:modernOpticsLecture5:390}
\Abs{\Psi(\Br')} = \Psi_0.
\end{dmath}
%
then this constant is
%
\begin{dmath}\label{eqn:modernOpticsLecture5:410}
\inv{2} \left( \frac{k \Psi_0}{2 \pi r} \right)^2.
\end{dmath}
%
Calculating this for a circular pattern is done in the class notes handout, where the result involved \(J_1\) Bessel functions.

We can deal with double slit by doing a convolution of a rectangle aperture with a pair of delta functions and then just multiply the Fourier transforms.

We will be applying this diffraction result to investigate coherence.  We'll find that if the source is not coherent, the chance of observing the fringe oscillations far from the source becomes very small.

%\vcsinfo
%\EndArticle
%\EndNoBibArticle
