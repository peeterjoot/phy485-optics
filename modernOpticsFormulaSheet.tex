%
% Copyright � 2012 Peeter Joot.  All Rights Reserved.
% Licenced as described in the file LICENSE under the root directory of this GIT repository.
%
% pick one:
%\input{../assignment.tex}
%\input{../blogpost.tex}
%\renewcommand{\basename}{modernOpticsCheatsheet}
%\renewcommand{\dirname}{notes/phy485/}
%%%\newcommand{\dateintitle}{}
%%%\newcommand{\keywords}{}
%%
%\input{../peeter_prologue_print2.tex}
%%
%\beginArtNoToc
%
%\generatetitle{Possible content for formula sheet}

\section{Rules.}

Rules: Handwritten.  No text or figures allowed.

\section{Geometric optics.}
\index{geometric optics}
%
\begin{equation}
n \sin\alpha = n' \sin\alpha'.
\end{equation}
%
\begin{equation}
\inv{f} = \inv{s} + \inv{s'}.
\end{equation}
%
\begin{equation}
\begin{bmatrix}
y_f \\
\alpha_f
\end{bmatrix}'
=
\begin{bmatrix}
1 & L \\
0 & 1
\end{bmatrix}
\begin{bmatrix}
y_i \\
\alpha_i
\end{bmatrix}
\end{equation}
%
\begin{equation}
\begin{bmatrix}
y_f \\
\alpha_f
\end{bmatrix}'
=
\begin{bmatrix}
1 & 0 \\
0 & \frac{n}{n'}
\end{bmatrix}
\begin{bmatrix}
y_i \\
\alpha_i
\end{bmatrix}
\end{equation}
%
\begin{equation}
\begin{bmatrix}
y_f \\
\alpha_f
\end{bmatrix}'
=
\begin{bmatrix}
1 & 0 \\
\inv{R}\left( \frac{n}{n'} -1 \right) & \frac{n}{n'}
\end{bmatrix}
\begin{bmatrix}
y_i \\
\alpha_i
\end{bmatrix}
\end{equation}
%
\begin{equation}
M =
\begin{bmatrix}
1 & 0 \\
2/R & 1
\end{bmatrix}
\end{equation}
%
\begin{equation}
M =
\begin{bmatrix}
1 & 0 \\
- 1/f & 1
\end{bmatrix}
\end{equation}
%
\begin{equation}
\inv{f} = \frac{n' - n}{n} \left( \inv{R_1} - \inv{R_2} \right), \qquad R_1 > 0, R_2 < 0.
\end{equation}
%
\begin{equation}
x x' = f^2, x + f = s, x' + f = s'.
\end{equation}
%
\begin{equation}
m = -\frac{s'}{s} = \frac{x'}{f}.
\end{equation}
%
\section{Misc trig.}
%
\begin{equation}
\sin(\pi - \theta) = \sin\theta.
\end{equation}
%
\begin{equation}
\sin( \alpha + \beta) = \sin \alpha \cos \beta + \cos \alpha \sin \beta.
\end{equation}
%
\begin{equation}
\cos(\alpha + \beta) = cos(\alpha)cos(\beta) - sin(\alpha)sin(\beta).
\end{equation}
%
\begin{equation}
\sin 2\theta = 2 \sin \theta \cos \theta.
\end{equation}
%
\begin{equation}
\cos 2\theta = 2 \cos^2 \theta - 1.
\end{equation}
%
\begin{equation}
\cos\Atan x = \inv{\sqrt{1 + x^2}}.
\end{equation}
\begin{equation}
\sin\Atan x = \frac{x}{\sqrt{1 + x^2}}.
\end{equation}
%
\section{Eikonal.}
\index{Eikonal equation}
%
\begin{equation}
\begin{bmatrix}
\BE \\
\BB \\
\end{bmatrix}
=
\begin{bmatrix}
\BE_0(\Br) \\
\BB_0(\Br) \\
\end{bmatrix}
e^{i \phi(\Br) - i \omega t}.
\end{equation}
%
%
\begin{equation}
n = \frac{c}{v} = \sqrt{\epsilon \mu}.
\end{equation}
%
\begin{equation}
\begin{bmatrix}
\BE \\
\BB \\
\end{bmatrix}
=
\begin{bmatrix}
\BE_0(\Br) \\
\BB_0(\Br) \\
\end{bmatrix}
e^{i \phi(\Br) - i \omega t}.
\end{equation}
%
%
\begin{equation}
\spacegrad \cdot \BE =
e^{-i \omega t}
\left(
\cancel{
e^{i \phi(\Br)}
\spacegrad \cdot \BE_0(\Br)
}
+
\BE_0(\Br) \cdot \left( \spacegrad e^{i \phi(\Br)} \right)
\right).
\end{equation}
%
and
\begin{equation}
\spacegrad \cross \BE =
e^{-i \omega t}
\left(
\cancel{
e^{i \phi(\Br)}
\spacegrad \cross \BE_0(\Br)
}
-
\BE_0(\Br) \cross \left( \spacegrad e^{i \phi(\Br)} \right)
\right).
\end{equation}
%
\begin{subequations}
%
\begin{equation}
\BE_0 \cdot \spacegrad \phi = 0.
\end{equation}
\begin{equation}
\BB_0 \cdot \spacegrad \phi = 0.
\end{equation}
\begin{equation}
\spacegrad \phi \cross \BE_0 = k_0 \BB_0.
\end{equation}
\begin{equation}
\spacegrad \phi \cross \BB_0 = - \epsilon k_0 \BE_0.
\end{equation}
\end{subequations}
%
\begin{equation}
\Abs{\spacegrad \phi}^2 = k_0^2 \epsilon(\Br).
\end{equation}
%
\begin{equation}
\Abs{\spacegrad \phi} = k_0 n(\Br).
\end{equation}
%
\begin{equation}
\expectation{\BS} = \frac{c}{8 \pi k_0} \Abs{\BE_0}^2 \spacegrad \phi.
\end{equation}
%
\begin{equation}
\Bt = \frac{d\Br(s)}{ds}
= \frac{\spacegrad \phi}{\Abs{\spacegrad \phi}} \\
= \frac{\spacegrad \phi}{ n(\Br) k_0}.
\end{equation}
%
\begin{equation}
n(\Br) \frac{d\Br}{ds} = \inv{k_0} \spacegrad \phi.
\end{equation}
%
\begin{equation}
\frac{d}{ds} \left( n(\Br) \frac{d\Br}{ds} \right) = \spacegrad n(\Br).
\end{equation}
%
\section{Wave relations.}
%
\begin{subequations}
\begin{equation}
k = \frac{2 \pi}{\lambda}.
\end{equation}
\begin{equation}
\omega = 2 \pi \nu.
\end{equation}
\begin{equation}
k = \frac{\omega}{c}.
\end{equation}
\begin{equation}
k = n k_0.
\end{equation}
\begin{equation}
\lambda = \lambda_0/n.
\end{equation}
\end{subequations}
%
\section{Electrodynamics.}

\begin{align}
\spacegrad \cdot \BD &= 0 \\
\spacegrad \cdot \BB &= 0 \\
\spacegrad \cross \BE &= - \inv{c} \PD{t}{\BB} \\
\spacegrad \cross \BB &= \inv{c} \PD{t}{\BD}
\end{align}
%
\begin{equation}
\BD = \epsilon \BE.
\end{equation}
%
\begin{equation}
\BS = \frac{c}{4 \pi} \Real{\BE} \cross \Real{\BB}.
\end{equation}
%
\begin{equation}
\LL_{\mathrm{EM}} = \inv{c} \rho \phi + \inv{c} \Bj \cdot \BA + \inv{8 \pi} \BE^2 - \inv{8 \pi} \BB^2.
\end{equation}
%
\begin{align}
\BE &= \Real\left( \BE_0 e^{i \Bk \cdot \Bx - \omega t} \right) \\
\BB &= \Real\left( \BB_0 e^{i \Bk \cdot \Bx - \omega t} \right)
\end{align}
%
\begin{equation}
\expectation{ \BE \cross \BB } = \inv{2} \Real( \BE_0 \cross \BB_0^\conj ).
\end{equation}
%
\begin{equation}
I
= c \epsilon_0 \expectation{ \Abs{\BE}^2 }
= \frac{c}{4\pi} \sqrt{ \frac{\epsilon}{\mu} } \expectation{ \Abs{\BE}^2 }.
\end{equation}
%
\section{Misc calculus results.}
%
\begin{equation}
\left(\spacegrad \phi \cdot \spacegrad \right) \spacegrad \phi
=
\inv{2} \spacegrad \left( \spacegrad \phi \right)^2.
\end{equation}
%
\begin{equation}
\spacegrad e^{i\phi} = i (\spacegrad \phi) e^{i \phi}.
\end{equation}
%
\begin{equation}
\spacegrad
=
\rcap \PD{r}{}
+\frac{\thetacap}{r} \PD{\theta}{}
+\zcap \PD{z}{}.
\end{equation}
%
\section{Diffraction.}
\index{diffraction}

\begin{align}
\BR &= \Br - \Br' \\
\BR_s &= \Br_s - \Br' \\
R &= \Abs{\BR} \\
R_s &= \Abs{\BR_s}
\end{align}
%
\begin{equation}
\left( \spacegrad^2 + \Bk^2 \right) \Psi(\Br) = 0.
\end{equation}
%
\begin{equation}
\Psi(\Br) = \iint da' \left( \Psi(\Br') \spacegrad' G - G \spacegrad' \Psi(\Br') \right) \cdot \ncap.
\end{equation}
%
\begin{equation}
G(\Br, \Br') = -\frac{e^{i k R}}{4 \pi R} = -\frac{e^{i k \Abs{\Br - \Br'}}}{4 \pi \Abs{\Br - \Br'}}.
\end{equation}
%
\begin{equation}
\spacegrad \left(
\frac{ e^{i k r} }{r}
\right) = \rcap \left( i k - \inv{r} \right)
\frac{e^{i k r}}{r}.
\end{equation}
%
\begin{equation}
\Psi(\Br) = -\inv{4 \pi} \iint \frac{e^{i k R}}{R}
\ncap \cdot
\left(
\spacegrad' \Psi(\Br') + \left( i k - \inv{R} \right) \frac{\BR}{R} \Psi(\Br') \right)
da'.
\end{equation}
%
\begin{equation}
\Psi(\Br) = \frac{A}{\lambda i} \iint da' \frac{e^{i k ( r + r_s)}}{r r_s} k(\theta).
\end{equation}
%
\begin{equation}
k(\theta) = \inv{2} \left( 1 + \cos \theta \right).
\end{equation}
%
\begin{equation}
k R =
%\mathLabelBox{
k r
%}{\(O(r\lambda)\)}
 -
%\mathLabelBox{
k \rcap \cdot \Br'
%}{\(O(d/\lambda)\)}
 +
%\mathLabelBox{
\frac{k}{2r} \left( {\Br'}^2 - \left( \rcap \cdot \Br' \right)^2 \right)
%}{\(O(d^2/\lambda r)\)}
 +
%\mathLabelBox{
\cdots
%}{\(O(d^3/\lambda r^2)\)}
\end{equation}
%
\begin{equation}
\Psi(\Br)
= \frac{ A }{\lambda i} \frac{ e^{i k (r_s + r)}}{r_s r} \iint_{A} da' e^{i k f(\Br')}.
\end{equation}
%
\begin{equation}
\begin{aligned}
f(\Br') &= - \left(\rcap + \rcap_s \right) \cdot \Br' \\
&\qquad + \inv{2r} \left( {\Br'}^2 - (\rcap \cdot \Br')^2 \right) + \inv{2 r_s} \left( (\Br')^2 - (\rcap_s \cdot \Br')^2 \right).
\end{aligned}
\end{equation}
%
\begin{equation}
\rcap \sim (\alpha, \beta, 1).
\end{equation}
%
\begin{equation}
\Psi(\Br) = \frac{\Psi_s}{i \lambda} \frac{e^{i k f}}{f} \iint_A e^{-i k ( \alpha x' + \beta y' ) } da'.
\end{equation}
%
\begin{equation}
k f = \frac{k}{2} \left(
r_s^{-1}
+r^{-1} \right) \left({x'}^2 + {y'}^2 \right) = \frac{\pi}{2} \left( u^2 + v^2 \right).
\end{equation}
%
\begin{equation}
\Psi(\Br)
=
\frac{A}{2 i} e^{ik (r_s + r)}
\frac{1}{r_s + r}
\int_{A} e^{i \frac{\pi}{2} (u^2 + v^2)} du dv.
\end{equation}
%
\begin{equation}
\int_{-\infty}^\infty dv e^{i \frac{\pi}{2} v^2 } = 1 + i = \sqrt{2} e^{i \pi/4}.
\end{equation}
%
\begin{subequations}
\begin{equation}
S(w) = \int_0^w \sin \left( \frac{\pi}{2} u^2 \right) du.
\end{equation}
\begin{equation}
C(w) = \int_0^w \cos \left( \frac{\pi}{2} u^2 \right) du.
\end{equation}
\end{subequations}
%
\begin{equation}
1 = \left(\frac{dS}{dw}\right)^2 + \left(\frac{dC}{dw}\right)^2.
\end{equation}
%
\begin{equation}
\tan
\left( \frac{\pi}{2} w^2 \right)
= \frac{dy}{dx}.
\end{equation}
%
\section{Coherence.}
\index{coherence}

\begin{align}
\Psi_1 &= \sqrt{I_1(\Br)} e^{i \phi_1(\Br, t)} \\
\Psi_2 &= \sqrt{I_2(\Br)} e^{i \phi_2(\Br, t)}
\end{align}
%
\begin{equation}
I
= \Abs{\Psi_1 + \Psi_2}^2
= \Abs{\Psi_1}^2
+ \Abs{\Psi_2}^2
+ 2 \Real \Gamma_{12}.
\end{equation}
%
\begin{equation}
\Gamma_{12} = \expectation{ \Psi_1 \Psi_2^\conj}.
\end{equation}
%
\begin{equation}
\calV
\equiv \frac
{
I_{\mathrm{max}}
- I_{\mathrm{min}}}
{
I_{\mathrm{max}}
+ I_{\mathrm{min}}
}
= \frac{2 \sqrt{I_1 I_2}}{I_1 + I_2} \Abs{\gamma_{12}} = \Abs{\gamma_{12}}.
\end{equation}
%
\begin{equation}
\gamma_{12} = \frac{\Gamma_{12}}{\sqrt{I_1} \sqrt{I_2}}.
\end{equation}
%
\subsection{Temporal coherence.}
\index{coherence!temporal}
%
\begin{equation}
\expectation{ f(t) } = \lim_{T \rightarrow \infty } \inv{T} \int_0^T dt' f(t').
\end{equation}
%
\begin{equation}
I
= \expectation{ \Abs{\Psi}^2 }
=
I(\Br_1) +
I(\Br_2) +
2 \Real \expectation{ \Psi(\Br_1, t) \Psi^\conj(\Br_2, t + \tau) }.
\end{equation}
%
\begin{equation}
\Gamma_{12} \equiv
\expectation{ \Psi(\Br_1, t) \Psi^\conj(\Br_2, t + \tau) }.
\end{equation}
%
\begin{equation}
\tau_{\mathrm{coh}} = \inv{\Delta \omega}.
\end{equation}
%
\begin{equation}
I = I_1 + I_2 + 2 \sqrt{I_1 I_2} \Real \gamma_{12}.
\end{equation}
%
\begin{equation}
\gamma(\tau) = \gamma_{12} = \calF \{ I(\omega) \}.
\end{equation}
%
\begin{subequations}
\begin{equation}
\Real(\Gamma(\tau)) = \int_0^\infty I(\omega) \cos(\omega \tau) d\omega.
\end{equation}
\begin{equation}
I(\omega)
= \int_0^\infty \Real\left( \Gamma(\tau) \right) \cos(\omega \tau) d\tau.
\end{equation}
\end{subequations}
%
\begin{equation}
\tau = \frac{s_2 - s_1}{c}.
\end{equation}
%
\begin{equation}
\begin{aligned}
I &=
\Abs{\gamma_{12}}
\left(
I_1 + I_2 + 2 \sqrt{I_1 I_2} \cos \left( \alpha_{12}(\tau) - \delta \right)
\right) \\
&\qquad +
\left( 1 - \Abs{\gamma_{12}} \right)
\left( I_1 + I_2 \right) \\
&=
\Abs{\gamma_{12}}
I_{\mathrm{coh}}
+
(1 - \Abs{\gamma_{12}})
I_{\mathrm{incoh}}.
\end{aligned}
\end{equation}
%
\begin{equation}
\tau_a - \tau_b = \frac{
r_{1a}
-r_{2a}
-r_{1b}
+r_{2b}
}{c}
=
\frac{l \theta_s}{c }.
\end{equation}
%
\subsection{Spatial coherence.}
\index{coherence!spatial}
%
\begin{equation}
\sum_{\Bk} I_{\Bk} e^{i \Bk \cdot \Bl} = \Gamma_{12} = \calF( I_{\Bk} ).
\end{equation}
%
\begin{equation}
\begin{aligned}
I
&= \sum_{\Bk} \Abs{ \Psi_k(\Br_1, t) + e^{i \Bk \cdot \Bl} \Psi_k(\Br_1, t) }^2 \\
&= \sum \left( 2 I_{\Bk} + 2 \Real \left( \Psi_k^\conj e^{i \Bk \cdot \Bl} \Psi_k \right) \right) \\
&= 2 \sum I_\Bk + 2 \Real \left( \sum_{\Bk} I_{\Bk} e^{i \Bk \cdot \Bl} \right).
\end{aligned}
\end{equation}
%
%\begin{equation}
%\Gamma_{12}
%=
%e^{ i \Bk_{\mathrm{av}} \cdot \Delta \Br }
%\inv{ \lambda^2 \overbar{R_1} \overbar{R_2} }
%\iint d^2 r_s e^{-i \Bk_s \cdot \Delta \Br } I( \Br_s )
%\end{equation}
%
\begin{subequations}
\begin{equation}
\Gamma_{12}
=
\inv{\lambda^2
\overbar{R_1}
\overbar{R_2}
}
e^{i k \Delta \Br \cdot \rcap_{\mathrm{av}}}
\int d^2 r_s
g(\Br_s) I(\Br_s)
e^{-i k \Delta \Br \cdot \Br_s/r_{\mathrm{av}}}.
\end{equation}
\begin{equation}
\Delta \Br = \Br_1 - \Br_2.
\end{equation}
\begin{equation}
\Br_{\mathrm{av}} = \Br_2 - \Delta \Br/2.
\end{equation}
\begin{equation}
k \left( R_1 - R_2 \right) \approx k \left( \Br_{\mathrm{av}} - \Br_s \right) \cdot \frac{ \Delta \Br}{r_{\mathrm{av}}}.
\end{equation}
\begin{equation}
\gamma(\Br_1, \Br_2) = \frac{\Gamma_{12}}{\evalbar{\Gamma_{12}}{\Delta r = 0}}.
\end{equation}
\end{subequations}
%\begin{equation}
%\Bk_s = k \frac{\Br_s}{r_{\mathrm{av}}}
%\end{equation}
%\begin{equation}
%\Bk_{\mathrm{av}} = k \rcap_{\mathrm{av}}
%\end{equation}
%
\begin{equation}
\calV
%= \frac
%{
%I_{\mathrm{max}}
%- I_{\mathrm{max}}
%}
%{
%I_{\mathrm{max}}
%+ I_{\mathrm{max}}
%}
= \Abs{\gamma_{12}}
=
2 \Abs{\frac{J_1(\pi \theta_s d/\lambda)}{\pi \theta_s d/\lambda}}.
\end{equation}
%
\begin{equation}
l_c = \frac{\lambda}{\Delta \theta_s}.
\end{equation}
%
\section{Multiple interference.}
\index{multiple interference}

\subsection{Fabry-Perot.}
\index{Fabry-Perot}
%
\begin{subequations}
\begin{equation}
\delta = 2 k L \cos\theta.
\end{equation}
\begin{equation}
r = e^{i \delta_r} \sqrt{R}.
\end{equation}
\begin{equation}
t = e^{i \delta_t} \sqrt{T}.
\end{equation}
\begin{equation}
\Delta = 2 \delta_r + \delta = 2 \pi m.
\end{equation}
\begin{equation}
\Psi_{\mathrm{transmission}}
=
\Psi_0 t^2 \inv{ 1 - R e^{i \Delta}}.
\end{equation}
\begin{equation}
I_{\mathrm{trans}}
=
\frac{I_{\mathrm{max}}}{1 + F \sin^2 (\Delta/2)}.
\end{equation}
\begin{equation}
I_{\mathrm{max}} = \frac{I_0 T^2}{(1 -R)^2}.
\end{equation}
\begin{equation}
F = \frac{4 R}{(1 - R)^2}.
\end{equation}
\begin{equation}
\calF = \pi \frac{\sqrt{R}}{1 - R} = \frac{\pi}{2} \sqrt{F} \sim \frac{\pi}{T}.
\end{equation}
\begin{equation}
\frac{\omega_1 - \omega_2}{\overbar{\omega}} = \inv{\calF m}.
\end{equation}
\begin{equation}
\overbar{\omega} = \frac{\pi c}{L} (m + j)
= \omega_0 + j \text{F S R}.
\end{equation}
\end{subequations}
%
\begin{subequations}
\begin{equation}
\delta = \omega - \omega_m.
\end{equation}
\begin{equation}
\frac{I}{I_0} = \inv{ 1 + \frac{\delta^2}{\Gamma^2}}.
\end{equation}
\begin{equation}
\Gamma = \frac{ \pi c }{2 L \calF} = \frac{\text{FSR}}{ 2 \calF }.
\end{equation}
\end{subequations}
%
\begin{equation}
\gamma
= \inv{2} a \frac{\omega}{c} \sin\theta
= \pi \frac{\omega}{\omega_0} \sin\theta.
\end{equation}
%
\begin{equation}
\omega_0 = \frac{2 \pi c}{a}.
\end{equation}
%
\subsection{Diffraction grating interferometry.}
\index{diffraction grating}
\index{interferometry}
%
\begin{subequations}
\begin{equation}
I
= I_0 \left( \frac{\sin\beta}{\beta} \right)^2 \left(
\frac
{\sin N\gamma}
{N \sin \gamma}
\right)^2.
\end{equation}
\begin{equation}
\beta = \inv{2} b k_y = \inv{2} b k \sin\theta.
\end{equation}
\begin{equation}
\gamma = \inv{2} k_y a = \inv{2} k a \sin\theta.
\end{equation}
\begin{equation}
\omega_0 = \frac{2 \pi c}{a}.
\end{equation}
\begin{equation}
\gamma = \pi \frac{\omega}{\omega_0} \sin\theta.
\end{equation}
\begin{equation}
\gamma = m \pi = \pi \frac{\omega}{\omega_0} \sin\theta.
\end{equation}
\begin{equation}
l = N m + 1.
\end{equation}
\begin{equation}
\Delta \gamma = \frac{m \pi}{\omega_1} \Delta \omega = \frac{\pi}{N}.
\end{equation}
\begin{equation}
\frac{\Delta \omega}{\overbar{\omega}} = \inv{N m}.
\end{equation}
\end{subequations}
%
\section{Lasers.}
\index{Laser}
%
\begin{subequations}
\begin{equation}
\frac{N_e}{N_g} = \frac{P_e}{P_g} = e^{-(E_e - E_g)/k T}.
\end{equation}
\begin{equation}
u_\omega = \Hbar \omega \expectation{ n_\omega } \calD(\omega).
\end{equation}
\begin{equation}
\expectation{ n_\omega } = \inv{ e^{\Hbar \omega / k T} - 1}.
\end{equation}
\begin{equation}
\calD(\omega) = \frac{\omega^2}{\pi^2 c^3}.
\end{equation}
\begin{equation}
\frac{d}{dt} N_e = -A N_e + B_{\mathrm{abs}} u N_g - B_{\mathrm{\mathrm{se}}} u N_e = 0.
\end{equation}
\begin{equation}
\frac{d}{dt} N_g = - \frac{d}{dt} N_e.
\end{equation}
\begin{equation}
u_\omega
=
\frac{\Hbar \omega^3}{\pi^2 c^3} \inv{ e^{ \Hbar \omega/ k T } -1 }.
\end{equation}
\begin{equation}
\frac{A}{B_{\mathrm{se}}} =\frac{\Hbar \omega^3}{\pi^2 c^3}.
\end{equation}
\begin{equation}
B_{\mathrm{abs}}/B_{\mathrm{se}} = 1.
\end{equation}
\begin{equation}
\frac{ B u}{A}  = \expectation{ n_\omega }.
\end{equation}
\begin{equation}
\expectation{n}^2 - (C - 1) n_s \expectation{n} - C n_s = 0.
\end{equation}
\begin{equation}
C \equiv
\frac{ N R \Gamma_{\mathrm{st}} }{\Gamma_{\mathrm{sp}} \Gamma_{\mathrm{cav}} }.
\end{equation}
\begin{equation}
n_s \equiv \frac{\Gamma_{\mathrm{sp}}}{ \Gamma_{\mathrm{st}}}.
\end{equation}
\begin{equation}
\expectation{n}
\approx
\left\{
\begin{array}{l l}
( C - 1 ) n_s & \quad \mbox{\(C > 1\)} \\
\frac{C}{1 - C} & \quad \mbox{\(C < 1\)} \\
\end{array}
\right.
\end{equation}
\begin{equation}
N_2 = \frac{N R}{ \Gamma_{\mathrm{sp}} + \Gamma_{\mathrm{st}} \expectation{n} } = \frac{C}{1 + \expectation{n}/n_s}.
\end{equation}
\end{subequations}
%
\section{Gaussian beams.}
\index{Gaussian beam}
%
\begin{subequations}
\begin{equation}
\spacegrad^2 \BE(\Br) + k^2(\Br) \BE(\Br) = 0.
\end{equation}
\begin{equation}
k^2(\Br) = k_0^2 - k_0 k_2 r^2.
\end{equation}
\begin{equation}
\BE = \BE_0 u(r, \theta, z) e^{i k_0 z}.
\end{equation}
\begin{equation}
\spacegrad^2
=
\spacegrad_{\txtT}^2
+ \inv{r^2} \PDSq{\theta}{}
+ \PDSq{z}{}.
\end{equation}
\begin{equation}
\spacegrad_{\txtT}^2
 = \PDSq{r}{} + \inv{r} \PD{r}{} = \PDSq{x}{} + \PDSq{y}{}.
\end{equation}
\begin{equation}
\cancel{\PDSq{z}{u}}
+ 2 i k_0 \PD{z}{u} + \spacegrad_{\txtT}^2 u - k_0 k_2 r^2 u = 0.
\end{equation}
\begin{equation}
\frac{\omega_{\mathrm{eff}}}{c} \equiv \sqrt{\frac{k_2}{k_0}}.
\end{equation}
\begin{equation}
m_{\mathrm{eff}} \equiv \frac{\Hbar k_0}{c}.
\end{equation}
\begin{equation}
t_{\mathrm{eff}} \equiv \frac{z}{c}.
\end{equation}
\begin{equation}
-\frac{\Hbar^2}{2 m_{\mathrm{eff}}} \spacegrad_{\txtT}^2 u + \inv{2} m_{\mathrm{eff}} \left(\omega_{\mathrm{eff}} \right)^2 r^2 u = i \Hbar \PD{t_{\mathrm{eff}}}{u}.
\end{equation}
\begin{equation}
u(r, z) = \frac{z_0}{i q} \exp\left( i \frac{ k_0 r^2}{2 q(z)} \right).
\end{equation}
\begin{equation}
q = z - i z_0.
\end{equation}
\begin{equation}
z_0 = \frac{\pi w_0^2}{\lambda} = \frac{\pi n w_0^2}{\lambda_0}.
\end{equation}
\begin{equation}
u
= \frac{w_0}{w(z)} \exp\left(
-\frac{r^2}{w^2(z)}
+i \frac{ k_0 r^2 }{2 R(z)}
- i \phi(z)
\right).
\end{equation}
\begin{equation}
\phi(z) = \Atan \left( \frac{z}{z_0} \right).
\end{equation}
\begin{equation}
w^2(z) = w_0^2 \left( 1  + \frac{z^2}{z_0^2} \right).
\end{equation}
\begin{equation}
\inv{R(z)} = \frac{z}{z^2 + z_0^2}.
\end{equation}
\begin{equation}
\inv{q(z)} = \inv{R(z)} + i \frac{\lambda_0}{n \pi w^2(z)}.
\end{equation}
\begin{equation}
\Theta_{\mathrm{div}}
=
\frac{w_0}{z_0}
=
\frac{1}{\pi w_0} \lambda.
\end{equation}
\end{subequations}
%
\begin{subequations}
\begin{equation}
-i \phi(z) + i k_0 z - i \omega t = i C.
\end{equation}
\begin{equation}
\ddt{z} = V_{\mathrm{ph}} = \frac{\omega}{k_{\mathrm{eff}}}.
\end{equation}
\begin{equation}
k_{\mathrm{eff}} = k_0 - \frac{z_0}{z^2 + z_0^2}.
\end{equation}
\end{subequations}
%
\begin{subequations}
\begin{equation}
\begin{aligned}
u_{lm} (x, y, z) 
&\sim \frac{w_0}{w(z)}
\exp
\left(
-\frac{r^2}{w^2(z)} + \frac{i k_0 r^2}{R(z)} - i (m + l + 1) \phi(z)
\right) 
\times
\\
&\quad
H_l \left(
\frac{\sqrt{2} x}{w(z)}
\right)
H_m \left(
\frac{\sqrt{2} y}{w(z)}
\right)
\end{aligned}
\end{equation}
\begin{equation}
\begin{aligned}
H_0(x) &= 1 \\
H_1(x) &= 2 x \\
H_2(x) &= 4 x^2 - 1
\end{aligned}
\end{equation}
\begin{equation}
k_{\mathrm{eff}} = k_0 - ( m + l + 1) \frac{z_0}{z^2 + z_0^2}.
\end{equation}
\begin{equation}
q' = \frac{ A q + B }{C q + D}.
\end{equation}
\begin{equation}
0 < g_1 g_2 < 1.
\end{equation}
\begin{equation}
g_{1,2} = 1 - \frac{L}{R_{1,2}}.
\end{equation}
\end{subequations}
%
\section{Fourier transforms.}
\index{Fourier transform}
%
\begin{equation}
f(x) = g * h = \int_{-\infty}^\infty dx' g(x') h(x - x').
\end{equation}
%
\begin{equation}
F(k) = G(k) H(k).
\end{equation}
%
%\EndNoBibArticle
