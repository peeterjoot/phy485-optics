%
% Copyright � 2012 Peeter Joot.  All Rights Reserved.
% Licenced as described in the file LICENSE under the root directory of this GIT repository.
%
%\input{../blogpost.tex}
%\renewcommand{\basename}{modernOpticsLecture13}
%\renewcommand{\dirname}{notes/phy485/}
%\newcommand{\keywords}{Optics, PHY485H1F}
%\input{../peeter_prologue_print2.tex}
%\beginArtNoToc
%\generatetitle{PHY485H1F Modern Optics.  Lecture 13: Fabry-Perot interferometry.  Taught by Prof.\ Joseph Thywissen}
%\label{chap:modernOpticsLecture13}

%\fxwarning{review}{work through this lecture in detail.}
We've got Etalons in real world situations such as light off a CD
\cref{fig:modernOpticsLecture13:modernOpticsLecture13Fig1}.
\imageFigure{../figures/phy485-optics/modernOpticsLecture13Fig1}{Laser on CD.}{fig:modernOpticsLecture13:modernOpticsLecture13Fig1}{0.2}
%
We'd previously considered wavefront splitting \cref{fig:modernOpticsLecture13:modernOpticsLecture13Fig2}, but now wish to consider amplitude splitting \cref{fig:modernOpticsLecture13:modernOpticsLecture13Fig2b}.
%
\imageFigure{../figures/phy485-optics/modernOpticsLecture13Fig2}{Wavefront splitting.}{fig:modernOpticsLecture13:modernOpticsLecture13Fig2}{0.2}
\imageFigure{../figures/phy485-optics/modernOpticsLecture13Fig2b}{Amplitude splitting.}{fig:modernOpticsLecture13:modernOpticsLecture13Fig2b}{0.2}
%
Last time we found
%
\begin{subequations}
\begin{dmath}\label{eqn:modernOpticsLecture13:20}
I_t = \frac{I_{\mathrm{max}}}{1 + F \sin^2 \Delta/2}.
\end{dmath}
\begin{dmath}\label{eqn:modernOpticsLecture13:40}
F = \frac{4 R}{(1 - R)^2}.
\end{dmath}
\begin{dmath}\label{eqn:modernOpticsLecture13:60}
\Delta = 2 \delta_r + \delta.
\end{dmath}
\begin{dmath}\label{eqn:modernOpticsLecture13:80}
\delta = 2 L k \cos\theta.
\end{dmath}
\end{subequations}
%
We've got sharp peaks at \(\Delta = 2 \pi m\)

How good is an Etalon at resolving frequency?

Suppose we've shined in two beams of the same frequency, and then slowly start changing the frequency of the other beam, until we get to the point where we've got both peaks centered at \(2 \pi m / \omega_k\) as in \cref{fig:modernOpticsLecture13:modernOpticsLecture13Fig3}.
%
\imageFigure{../figures/phy485-optics/modernOpticsLecture13Fig3}{Intensity from multiple Etalons.}{fig:modernOpticsLecture13:modernOpticsLecture13Fig3}{0.2}
%
re-label with
%
\begin{dmath}\label{eqn:modernOpticsLecture13:100}
2 k_1 L = \frac{ 2 \omega_1 L }{c}.
\end{dmath}
%
Relabeling \cref{fig:modernOpticsLecture13:modernOpticsLecture13Fig4}.
%
\imageFigure{../figures/phy485-optics/modernOpticsLecture13Fig4}{Intensity from multiple Etalons, relabeled.}{fig:modernOpticsLecture13:modernOpticsLecture13Fig4}{0.2}
%
We'll consider this ``resolved'' when the second peak is centered at the point when our first peak has lost half of its intensity as in \cref{fig:modernOpticsLecture13:modernOpticsLecture13Fig5}.
%
\imageFigure{../figures/phy485-optics/modernOpticsLecture13Fig5}{Two peaks resolved.}{fig:modernOpticsLecture13:modernOpticsLecture13Fig5}{0.2}
%
In mathese, this resolution is
%
\begin{dmath}\label{eqn:modernOpticsLecture13:120}
\frac{I}{I_{\mathrm{max}}}
=
\inv{2}.
\end{dmath}
%
at the peak for \(\omega_2\).  That is
%
\begin{dmath}\label{eqn:modernOpticsLecture13:140}
1 + F\sin^2\frac{\Delta_1}{2} = 2.
\end{dmath}
%
\begin{dmath}\label{eqn:modernOpticsLecture13:280}
\sin\frac{\Delta_1}{2} = \inv{\sqrt{F}}.
\end{dmath}
%
\begin{dmath}\label{eqn:modernOpticsLecture13:160}
x = \frac{2}{\sqrt{F}}.
\end{dmath}
%
\begin{dmath}\label{eqn:modernOpticsLecture13:180}
\Delta_1 = 2 \pi m + x.
\end{dmath}
%
Define, the \underlineAndIndex{Finesse}, as
%
\begin{dmath}\label{eqn:modernOpticsLecture13:200}
\calF = \pi \frac{\sqrt{R}}{1 - R} = \frac{\pi}{2} \sqrt{F} \sim \frac{\pi}{T}.
\end{dmath}
%
\begin{dmath}\label{eqn:modernOpticsLecture13:220}
\omega_1 - \omega_2 = \frac{2 c}{L \sqrt{F}}
=\frac{\pi c}{L \calF}.
\end{dmath}
%
\begin{dmath}\label{eqn:modernOpticsLecture13:240}
\frac{\omega_1 - \omega_2}{\overbar{\omega}} = \inv{\calF m}.
\end{dmath}
%
Roughly speaking \(\calF\) is an instruction to ``buy good mirrors'', whereas \(m\) means ``use a long cavity''

How many reflections?
%
\begin{dmath}\label{eqn:modernOpticsLecture13:260}
N \sim \inv{T} \sim \calF.
\end{dmath}
%
\paragraph{\(N\)-wave interference}
\index{interference!multiple wave}

Cavity length is important.  Suppose we had \cref{fig:modernOpticsLecture13:modernOpticsLecture13Fig6}
\index{cavity length}
%
\imageFigure{../figures/phy485-optics/modernOpticsLecture13Fig6}{Many Etalons.}{fig:modernOpticsLecture13:modernOpticsLecture13Fig6}{0.2}
%
which gives
%
\begin{dmath}\label{eqn:modernOpticsLecture13:300}
\Delta = \text{offset} + 2 \frac{\overbar{\omega}}{c} L = 2 \pi (m + j).
\end{dmath}
%
Neglecting the offset so that
%
\begin{dmath}\label{eqn:modernOpticsLecture13:320}
2 \frac{\overbar{\omega}}{c} L = 2 \pi (m + j).
\end{dmath}
%
or
%
\begin{dmath}\label{eqn:modernOpticsLecture13:340}
\overbar{\omega} = \frac{\pi c}{L} (m + j)
= \omega_0 + j \text{F S R}.
\end{dmath}
%
where \(\text{F S R}\) is the \underlineAndIndex{Free Spectral Range}.  Re-plotting in \cref{fig:modernOpticsLecture13:modernOpticsLecture13Fig7}.
%
\imageFigure{../figures/phy485-optics/modernOpticsLecture13Fig7}{Illustrating Free Spectral Resolution.}{fig:modernOpticsLecture13:modernOpticsLecture13Fig7}{0.2}
%
This is called a \underlineAndIndex{Fabry-Perot Spectrometer}.  These guys, who were first able to achieve a good spectrometer of this sort, achieved \(\calF \sim 30-100\), using \(\lambda/100\) flatness, where a typical mirror has \(\lambda/10\) flatness!

\paragraph{Reading}: \S 4 of \citep{fowles1989introduction}.

\paragraph{Additional discussion from last class.}
%
\begin{dmath}\label{eqn:modernOpticsLecture13:360}
\Delta = \frac{2 L}{c} \omega + 2 \delta_r = 2 \pi m.
\end{dmath}
%
Here \(\Delta\) is the round trip phase.  Our resonances are
%
\begin{dmath}\label{eqn:modernOpticsLecture13:380}
\omega = \frac{c}{2L} 2 \pi m - \frac{c}{2 L} 2 \delta_r
= \text{FSR} m + \text{offset}.
\end{dmath}
%
If \(\text{FSR} = \pi c/L\).

%\EndArticle
