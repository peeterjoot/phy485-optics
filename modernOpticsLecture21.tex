%
% Copyright � 2012 Peeter Joot.  All Rights Reserved.
% Licenced as described in the file LICENSE under the root directory of this GIT repository.
%
%\input{../blogpost.tex}
%\renewcommand{\basename}{modernOpticsLecture21}
%\renewcommand{\dirname}{notes/phy485/}
%\newcommand{\keywords}{Optics, PHY485H1F}
%\input{../peeter_prologue_print2.tex}
%
%\usepackage{tikz}
%\usepackage[draft]{fixme}
%\fxusetheme{color}
%
%\beginArtNoToc
%\generatetitle{PHY485H1F Modern Optics.  Lecture 21: Spectral line width (coherence time) of laser.  Taught by Prof.\ Joseph Thywissen}
%%\chapter{Spectral line width (coherence time) of laser}
\index{spectral line width}
\index{coherence time}
\index{laser}
%\label{chap:modernOpticsLecture21}
%
%\section{Disclaimer}
%
%Peeter's lecture notes from class.  May not be entirely coherent.

\section{Spectral line width (coherence time) of laser}

READING: \citep{grynberg2010introduction} Appendix C.

\fxwarning{review lecture 21}{work through this lecture in detail.}

The spectral linewidth (or coherence time) of a laser is called the \underlineAndIndex{Schawlow-Townes limit}.

Laser depicted in \cref{fig:modernOpticsLecture21:modernOpticsLecture21Fig1}.
\imageFigure{../figures/phy485-optics/modernOpticsLecture21Fig1}{Laser cavity}{fig:modernOpticsLecture21:modernOpticsLecture21Fig1}{0.3}
%
\begin{dmath}\label{eqn:modernOpticsLecture21:20}
\text{rate of emission into laser mode} = \Gamma_{\mathrm{st}}(
\mathLabelBox
{
1
}{spontaneous}
+
\mathLabelBox[labelstyle={below of=m\themathLableNode,below of=m\themathLableNode}]
{
\expectation{n}
}{stimulated}
) N_2
\end{dmath}

Reminder:
%
\begin{dmath}\label{eqn:modernOpticsLecture21:40}
\tau_c = \mbox{coherence time}
\end{dmath}
%
\begin{dmath}\label{eqn:modernOpticsLecture21:60}
\Gamma_{\mathrm{st}} = \inv{\tau_c} = \mbox{line width}
\end{dmath}
%
\begin{dmath}\label{eqn:modernOpticsLecture21:80}
c \tau_c = l_c = \mbox{longitudinal coherence length}
\end{dmath}

Phase of a laser, if monochromatic as in \cref{fig:modernOpticsLecture21:modernOpticsLecture21Fig2}.
%
\imageFigure{../figures/phy485-optics/modernOpticsLecture21Fig2}{Field associated with ground state}{fig:modernOpticsLecture21:modernOpticsLecture21Fig2}{0.3}

goes like
%
\begin{dmath}\label{eqn:modernOpticsLecture21:100}
\Psi \sim e^{-i \omega_0 t}
\end{dmath}

If not monochromatic \cref{fig:modernOpticsLecture21:modernOpticsLecture21Fig3}, define
%
\imageFigure{../figures/phy485-optics/modernOpticsLecture21Fig3}{Random walk in phase}{fig:modernOpticsLecture21:modernOpticsLecture21Fig3}{0.3}
%
\begin{dmath}\label{eqn:modernOpticsLecture21:120}
\bcE_{\txtL} = E_{\txtL} e^{i \phi(t)}
\end{dmath}

(underscore L stands for laser).
%
\begin{dmath}\label{eqn:modernOpticsLecture21:140}
\Psi = \bcE_{\txtL} e^{-i \omega t} \times \mbox{spatial mode}
\end{dmath}

Random relative phase \(\theta\).  Two components of \(\bcE_{\mathrm{sp}}\)

\begin{enumerate}
\item Parallel to \(\bcE_{\txtL}\) changes amplitude, taken out by steady state operation of laser.
\item perpendicular: changes \(\phi\) by \(\delta \phi\).
\end{enumerate}

Considering the average process we have random phase changes, with no effective change in magnitude, or
%
\begin{dmath}\label{eqn:modernOpticsLecture21:160}
\bcE_{\txtL}' \sim \bcE_{\txtL} e^{i \delta \phi}
\end{dmath}

This is a random walk \cref{fig:modernOpticsLecture21:modernOpticsLecture21Fig4} in phase!
%
\imageFigure{../figures/phy485-optics/modernOpticsLecture21Fig4}{Angle is a random variable}{fig:modernOpticsLecture21:modernOpticsLecture21Fig4}{0.3}

Average step:
%
\begin{dmath}\label{eqn:modernOpticsLecture21:180}
\expectation{\delta \phi} = \frac{E_{\mathrm{sp}}}{E_{\txtL}}
\mathLabelBox{
\expectation{\cos\phi}
}{\(= 0\)}
\end{dmath}

However, the RMS deviation
%
\begin{dmath}\label{eqn:modernOpticsLecture21:200}
\sigma_\phi
= \sqrt{ \expectation{(\delta \phi)^2} - \expectation{\delta \phi}^2 }
= \sqrt{
\frac{E_{\mathrm{sp}}^2}{E_{\txtL}^2}
\mathLabelBox{\expectation{\cos^ \theta} }{\(1/2\)}
- 0
}
= \frac{E_{\mathrm{sp}}}{\sqrt{2} E_{\txtL}}
\end{dmath}

After \(N\) events
%
\begin{dmath}\label{eqn:modernOpticsLecture21:220}
\sigma_\phi
= \frac{E_{\mathrm{sp}}}{\sqrt{2} E_{\txtL}} \sqrt{N}
\end{dmath}

The number of spontaneous events is given by the emission rate of the laser mode, and is
%
\begin{dmath}\label{eqn:modernOpticsLecture21:240}
\text{number of spontaneous events} = T \Gamma_{\mathrm{st}} N_2
\end{dmath}

Also
%
\begin{dmath}\label{eqn:modernOpticsLecture21:260}
\expectation{n} = \frac{E_{\txtL}^2}{E_{\mathrm{sp}}^2}
\end{dmath}

After time \(T = \tau_c\), \(\sigma_\phi = 1\), or
%
\begin{dmath}\label{eqn:modernOpticsLecture21:280}
\frac{E_{\mathrm{sp}}}{\sqrt{2} E_{\txtL}} \sqrt{ \tau_c \Gamma_{\mathrm{st}} N_2} = 1
\end{dmath}

Referring to \cref{fig:modernOpticsLecture21:modernOpticsLecture21Fig5} we have
%
\imageFigure{../figures/phy485-optics/modernOpticsLecture21Fig5}{Change in field due to spontaneous emission}{fig:modernOpticsLecture21:modernOpticsLecture21Fig5}{0.3}
%
\begin{dmath}\label{eqn:modernOpticsLecture21:300}
\frac{E_{\mathrm{sp}}}{E_{\txtL}} = \inv{\sqrt{\expectation{n}}}
\end{dmath}

or
%
\begin{dmath}\label{eqn:modernOpticsLecture21:320}
\frac{1}{\sqrt{2} \sqrt{\expectation{n}}} \sqrt{ \tau_c \Gamma_{\mathrm{st}} N_2} = 1,
\end{dmath}

so
%
\begin{dmath}\label{eqn:modernOpticsLecture21:340}
\tau_c = \left(
\frac{N_2 \Gamma_{\mathrm{st}}}{2 \expectation{n}}
\right)^{-1}
\end{dmath}
%
\begin{dmath}\label{eqn:modernOpticsLecture21:360}
\Gamma_{\mathrm{st}} = \inv{\tau_c} = \frac{N_2 \Gamma_{\mathrm{st}}}{2 \expectation{n}}
\end{dmath}

Energy balance
\begin{dmath}\label{eqn:modernOpticsLecture21:380}
N_2 =
\frac{\Gamma_{\mathrm{cav}}}{
 \Gamma_{\mathrm{st}}
}
\end{dmath}

so
\begin{subequations}
\begin{dmath}\label{eqn:modernOpticsLecture21:400}
\tau_c = \left(
\frac{\Gamma_{\mathrm{cav}}}{2 \expectation{n}}
\right)^{-1}
\end{dmath}
\begin{dmath}\label{eqn:modernOpticsLecture21:420}
\Gamma_{\mathrm{st}} = \inv{\tau_c} = \frac{\Gamma_{\mathrm{cav}}}{2 \expectation{n}}
\end{dmath}
\end{subequations}

With power output
%
\begin{dmath}\label{eqn:modernOpticsLecture21:440}
P = \Hbar \omega \expectation{n} \Gamma_{\mathrm{cav}}
\end{dmath}
%
\begin{dmath}\label{eqn:modernOpticsLecture21:460}
\Gamma_st = \frac{\Gamma_{\mathrm{cav}} \Hbar \omega}{P}
\end{dmath}
%
\begin{dmath}\label{eqn:modernOpticsLecture21:480}
\Gamma_{\mathrm{cav}} = (\text{Trans}) \frac{c}{2L}
\end{dmath}

where \(L\) is the cavity length.

\makeexample{Some numbers}{example:modernOpticsLecture21:1}{
\begin{equation}\label{eqn:modernOpticsLecture21:500}
\begin{aligned}
P &\sim 1 \mathrm{mW} \\
L &\sim 1 \txtm \\
T &\sim 2 \% \\
\lambda &\sim 500 \mathrm{nm} \\
\expectation{n} &= 8 \times 10^8 \\
\Gamma_st &= 4 \times 10^{-3} \txts^-1
\end{aligned}
\end{equation}

or about \(1 m \mathrm{Hz}\).
}

\makeexample{Some numbers for Diode laser}{example:modernOpticsLecture21:2}{
\begin{equation}\label{eqn:modernOpticsLecture21:520}
\begin{aligned}
L &\sim 300 \mu m \\
T &\sim 10\%
\end{aligned}
\end{equation}
}

%\EndArticle
%\EndNoBibArticle
