%
% Copyright � 2012 Peeter Joot.  All Rights Reserved.
% Licenced as described in the file LICENSE under the root directory of this GIT repository.
%
%\input{../blogpost.tex}
%\renewcommand{\basename}{modernOpticsLecture1}
%\renewcommand{\dirname}{notes/phy485/}
%\newcommand{\keywords}{Optics, PHY485H1F, ABCD matrix, lens-makers formula, transfer matrix}
%\input{../peeter_prologue_print2.tex}
%
%\beginArtNoToc
\index{geometric optics}

%\generatetitle{PHY485H1F Modern Optics.  Lecture 1: Matrix methods in Geometric Optics.  Taught by Prof.\ Joseph Thywissen}
\label{chap:modernOpticsLecture1}

%\section{Disclaimer}
%
%Peeter's lecture notes from class.  May not be entirely coherent.

\section{Missing content.}

I was late.  Glancing at somebody else's notes very quickly, it looked like I missed

\begin{itemize}
\item A derivation of Snell's law from Fermat's principle.
\item The focal point formula
\begin{equation}\label{eqn:modernOpticsLecture1:411}
\inv{f} = \inv{s} + \inv{s'}.
\end{equation}
\item The paraxial approximation was defined (small angles only, looking only near the center of a lens).
\end{itemize}

I was going to scrounge for notes to scan from somebody else so I can fill in missing content, but instead think I'll fill in with some self assigned problems.

Suggested reading for this lecture is \S 5 from \citep{hecht1998hecht}.

\section{Matrix methods.}
\index{matrix methods}

Referring to \cref{fig:modernOpticsLecture1:modernOpticsLecture1Fig1}
%
\imageFigure{../figures/phy485-optics/modernOpticsLecture1Fig1}{Matrix method.}{fig:modernOpticsLecture1:modernOpticsLecture1Fig1}{0.2}
%
we can define a transfer out matrix, or \(A B C D\) matrix taking pairs of coordinates describing rays
%
\begin{dmath}\label{eqn:modernOpticsLecture1:10}
\begin{bmatrix}
y \\
\alpha
\end{bmatrix}
\end{dmath}
%
so that the transition of the ray through the interface is described as
\begin{dmath}\label{eqn:modernOpticsLecture1:30}
\begin{bmatrix}
y_f \\
\alpha_f
\end{bmatrix}
=
\begin{bmatrix}
A & B \\
C & D
\end{bmatrix}
\begin{bmatrix}
y_i \\
\alpha_i
\end{bmatrix}
\end{dmath}
%
\subsection{Free propagation.}
\index{free propagation}
Referring to \cref{fig:modernOpticsLecture1:modernOpticsLecture1Fig2} we see that
%
\imageFigure{../figures/phy485-optics/modernOpticsLecture1Fig2}{Free propagation.}{fig:modernOpticsLecture1:modernOpticsLecture1Fig2}{0.2}
%
\begin{dmath}\label{eqn:modernOpticsLecture1:35}
\tan \alpha = \frac{\Delta y}{L} \approx \alpha,
\end{dmath}
%
so that
%
\begin{dmath}\label{eqn:modernOpticsLecture1:45}
y' = y + \Delta y \approx y + L \alpha,
\end{dmath}
%
so that our matrix describing free propagation is
%
\begin{dmath}\label{eqn:modernOpticsLecture1:50}
\begin{bmatrix}
y_f \\
\alpha_f
\end{bmatrix}
=
\begin{bmatrix}
1 & L \\
0 & 1
\end{bmatrix}
\begin{bmatrix}
y_i \\
\alpha_i
\end{bmatrix}
\end{dmath}
%
\subsection{Refraction off of a flat lens.}
\index{refraction}
\index{flat lens}

Referring to \cref{fig:modernOpticsLecture1:modernOpticsLecture1Fig3}
%
\imageFigure{../figures/phy485-optics/modernOpticsLecture1Fig3}{Refraction at flat surface.}{fig:modernOpticsLecture1:modernOpticsLecture1Fig3}{0.2}
%
where
%
\begin{dmath}\label{eqn:modernOpticsLecture1:70}
n \sin\alpha = n' \sin\alpha'.
\end{dmath}
%
We employ the paraxial approximation
%
\begin{dmath}\label{eqn:modernOpticsLecture1:90}
n \alpha \sim n' \alpha',
\end{dmath}
%
or
%
\begin{dmath}\label{eqn:modernOpticsLecture1:110}
\alpha' = \frac{n'}{n} \alpha,
\end{dmath}
%
allowing for the description of refraction off of a flat lens by
%
\begin{dmath}\label{eqn:modernOpticsLecture1:130}
\begin{bmatrix}
y_f \\
\alpha_f
\end{bmatrix}
=
\begin{bmatrix}
1 & 0 \\
0 & \frac{n'}{n}
\end{bmatrix}
\begin{bmatrix}
y_i \\
\alpha_i
\end{bmatrix}
\end{dmath}
%
\subsection{Refraction of a curved surface.}
\index{refraction!curved surface}

\subsubsection{Convex refraction.}
\index{refraction!convex}

Referring to \cref{fig:modernOpticsLecture1:modernOpticsLecture1Fig4}
%
\imageFigure{../figures/phy485-optics/modernOpticsLecture1Fig4}{Refraction of convex surface.}{fig:modernOpticsLecture1:modernOpticsLecture1Fig4}{0.2}
%
we see that
%
\begin{dmath}\label{eqn:modernOpticsLecture1:150}
\phi \approx \frac{y}{R},
\end{dmath}
%
and can also employ Snell's law in the approximation
%
\begin{dmath}\label{eqn:modernOpticsLecture1:370}
n \theta = n' \theta'.
\end{dmath}
%
From the figure we see that
%
\begin{subequations}
\begin{equation}\label{eqn:modernOpticsLecture1:410}
\theta = \alpha + \phi.
\end{equation}
\begin{equation}\label{eqn:modernOpticsLecture1:430}
\theta' = \alpha + \phi'.
\end{equation}
\end{subequations}
%
or
\begin{dmath}\label{eqn:modernOpticsLecture1:390}
n(\alpha + \phi) = n' (\alpha' + \phi).
\end{dmath}
%
Using \eqnref{eqn:modernOpticsLecture1:150} this is
\begin{dmath}\label{eqn:modernOpticsLecture1:390b}
n\left( \alpha + \frac{y}{R} \right) = n' \left( \alpha' + \frac{y}{R} \right),
\end{dmath}
%
which we can rearrange to find
%
\begin{equation}\label{eqn:modernOpticsLecture1:450}
\alpha' = \frac{y}{R}\left( \frac{n}{n'} - 1 \right) + \frac{n}{n'} \alpha.
\end{equation}
%
This can now be put into matrix form, yielding
%
\begin{dmath}\label{eqn:modernOpticsLecture1:170}
\begin{bmatrix}
y_f \\
\alpha_f
\end{bmatrix}
=
\begin{bmatrix}
1 & 0 \\
\inv{R}\left( \frac{n}{n'} -1 \right) & \frac{n}{n'}
\end{bmatrix}
\begin{bmatrix}
y_i \\
\alpha_i
\end{bmatrix}
\end{dmath}
%
Observe that for \(R \rightarrow \infty\) we have the same result as a flat surface.

\subsubsection{Concave refraction.}
\index{refraction!concave}

Now, let's consider an input ray against a concave surface.  We have just to flip around some of the labels as in \cref{fig:modernOpticsLecture1:modernOpticsLecture1Fig4b}.
%
\imageFigure{../figures/phy485-optics/modernOpticsLecture1Fig4b}{Refraction against concave surface.}{fig:modernOpticsLecture1:modernOpticsLecture1Fig4b}{0.2}
%
From the figure the only difference is that the coordinate of the focus is now at \((-R, 0)\) whereas it was \((R, 0)\) before.  So if we write \(R' = -R\), with \(R\) being positive we find for a ray incident on a concave lens
%
\begin{dmath}\label{eqn:modernOpticsLecture1:170b}
\begin{bmatrix}
y_f \\
\alpha_f
\end{bmatrix}
=
\begin{bmatrix}
1 & 0 \\
\inv{R'}\left( \frac{n}{n'} -1 \right) & \frac{n}{n'}
\end{bmatrix}
\begin{bmatrix}
y_i \\
\alpha_i
\end{bmatrix}
\end{dmath}
%
Alternatively, and this is what we do, is that we allow for \(R\) to be signed, in which case this formula is true for both \(R > 0\) (convex) and \(R < 0\) (concave).  We employ the following sign convention \cref{fig:modernOpticsLecture1:modernOpticsLecture1Fig5}
%
\imageFigure{../figures/phy485-optics/modernOpticsLecture1Fig5}{Concave curvature.}{fig:modernOpticsLecture1:modernOpticsLecture1Fig5}{0.2}
%
where \(-R\) is used for concave, and \(+R\) is used for convex.

\index{curvature!sign convention}

The sign conventions are illustrated in \cref{fig:modernOpticsLecture1:modernOpticsLecture1Fig6}
%
\imageFigure{../figures/phy485-optics/modernOpticsLecture1Fig6}{Sign convention.}{fig:modernOpticsLecture1:modernOpticsLecture1Fig6}{0.2}
%
and \cref{fig:modernOpticsLecture1:modernOpticsLecture1Fig7}.
\imageFigure{../figures/phy485-optics/modernOpticsLecture1Fig7}{Convex curvature.}{fig:modernOpticsLecture1:modernOpticsLecture1Fig7}{0.2}
%
\subsection{\(ABCD\) matrix for a lens.}
\index{ABCD matrix}

Consider the lens illustrated in \cref{fig:modernOpticsLecture1:modernOpticsLecture1Fig8}
%
\imageFigure{../figures/phy485-optics/modernOpticsLecture1Fig8}{matrix for lens.}{fig:modernOpticsLecture1:modernOpticsLecture1Fig8}{0.2}
%
where once again we use the paraxial approximation, assuming we are considering only close enough to the middle that we can neglect any variation in thickness.  In this approximation we have a geometry of the form \cref{fig:modernOpticsLecture1:modernOpticsLecture1Fig9}
%
\imageFigure{../figures/phy485-optics/modernOpticsLecture1Fig9}{Lens paraxial approximation.}{fig:modernOpticsLecture1:modernOpticsLecture1Fig9}{0.2}
%
where \(y \ll R_1, R_2\).

Our complete transfer matrix is then given by
%
\begin{dmath}\label{eqn:modernOpticsLecture1:190}
\begin{bmatrix}
y_f \\
\alpha_f
\end{bmatrix}
=
M_3 M_2 M_1
\begin{bmatrix}
y_i \\
\alpha_i
\end{bmatrix}
= M
\begin{bmatrix}
y_i \\
\alpha_i
\end{bmatrix}
.
\end{dmath}
%
We've described \(M_1, M_2, M_3\) individually above, and they are
%
\begin{subequations}
\begin{equation}\label{eqn:modernOpticsLecture1:470}
M_1 =
\begin{bmatrix}
1 & 0 \\
\inv{R_1}\left( \frac{n}{n'} -1 \right) & \frac{n}{n'}
\end{bmatrix}
\end{equation}
\begin{equation}\label{eqn:modernOpticsLecture1:490}
M_2 =
\begin{bmatrix}
1 & t \\
0 & 1
\end{bmatrix}
\end{equation}
\begin{equation}\label{eqn:modernOpticsLecture1:510}
M_3 =
\begin{bmatrix}
1 & 0 \\
\inv{R_2}\left( \frac{n'}{n} -1 \right) & \frac{n'}{n}
\end{bmatrix}
\end{equation}
\end{subequations}
%
The matrix \(M\) is the mess
%
\begin{dmath}\label{eqn:modernOpticsLecture1:530}
M
=
\begin{bmatrix}
1 & 0 \\
\inv{R_2}\left( \frac{n'}{n} -1 \right) & \frac{n'}{n}
\end{bmatrix}
\begin{bmatrix}
1 & t \\
0 & 1
\end{bmatrix}
\begin{bmatrix}
1 & 0 \\
\inv{R_1}\left( \frac{n}{n'} -1 \right) & \frac{n}{n'}
\end{bmatrix}
\end{dmath}
%
With only the ratio \(n'/n\) showing up, let's make the substitution
%
\begin{dmath}\label{eqn:modernOpticsLecture1:550}
\frac{n'}{n} \rightarrow n,
\end{dmath}
%
effectively working with \(n = 1\) outside of the lens.  This gives us
%
\begin{dmath}\label{eqn:modernOpticsLecture1:570}
M
=
\begin{bmatrix}
1 & 0 \\
\inv{R_2}\left( n -1 \right) & n
\end{bmatrix}
\begin{bmatrix}
1 & t \\
0 & 1
\end{bmatrix}
\begin{bmatrix}
1 & 0 \\
\inv{R_1}\left( \inv{n} -1 \right) & \inv{n}
\end{bmatrix}
=
\begin{bmatrix}
1 & t \\
\inv{R_2}\left( n -1 \right) &
\frac{t}{R_2}\left( n -1 \right) + n
\end{bmatrix}
\begin{bmatrix}
1 & 0 \\
\inv{R_1}\left( \inv{n} -1 \right) & \inv{n}
\end{bmatrix}
=
\begin{bmatrix}
1 + \frac{t}{R_1}\left( \inv{n} -1 \right) & \frac{t}{n} \\
\inv{R_2}\left( n -1 \right)
- \inv{R_1}\left( n -1 \right)
+
\frac{t}{R_1 R_2}\left( n -1 \right) \left( \inv{n} -1 \right) &
\frac{t}{R_2}\left( -\inv{n} + 1 \right) + 1
\end{bmatrix}
=
\begin{bmatrix}
1 + \frac{t}{R_1}\left( \inv{n} -1 \right) & t \inv{n} \\
-\left( n - 1 \right)
\left(
\inv{R_1} -\inv{R_2}
+
\frac{t}{n R_1 R_2}\left( n - 1 \right)
\right)
&
1 - \frac{t}{R_2}\left( \inv{n} - 1 \right)
\end{bmatrix}
\end{dmath}
%
Writing the \(C\) term as \(-1/f\) we have what's called the \underlineAndIndex{Lens makers formula} (for a thick lens in this case), and putting back in \(n\) and \(n'\) we have
%
\begin{dmath}\label{eqn:modernOpticsLecture1:590}
\inv{f} =
\left( \frac{n'}{n} - 1 \right)
\left(
\inv{R_1} -\inv{R_2}
+
\frac{n t}{n' R_1 R_2}\left( \frac{n'}{n} - 1 \right)
\right).
\end{dmath}
%
Observe that for a thin lens where \(t \rightarrow 0\) we have the approximation
%
\begin{dmath}\label{eqn:modernOpticsLecture1:210}
M =
\begin{bmatrix}
1 & 0 \\
- 1/f & 1
\end{bmatrix}.
\end{dmath}
%
For our focus in \eqnref{eqn:modernOpticsLecture1:590} we get
%
\begin{dmath}\label{eqn:modernOpticsLecture1:230}
\inv{f} = \frac{n' - n}{n} \left( \inv{R_1} - \inv{R_2} \right).
\end{dmath}
%
This is called the (thin lens) Lens makers formula.  This, and \(n \approx 1.5\) is enough to construct many lens designs.  Our sign conventions for \(f\) are illustrated by \cref{fig:modernOpticsLecture1:modernOpticsLecture1Fig10}.
\imageFigure{../figures/phy485-optics/modernOpticsLecture1Fig10}{Lens.}{fig:modernOpticsLecture1:modernOpticsLecture1Fig10}{0.2}
%
where \(f > 0\) is convex and \(f < 0\) is concave.

\subsection{Properties of the transfer matrix.}
\index{transfer matrix}

\begin{enumerate}
\item The determinant of the transfer matrix is described by index of refraction of just the initial and final media, and \(\Det M = n_0/n_f\).
%
%\fxwarning{Demonstrate, matrix optics}{demonstrate.}
Examples
\begin{enumerate}
\item Thin lens
\begin{equation}\label{eqn:modernOpticsLecture1:250}
M =
\begin{bmatrix}
1 & 0 \\
- 1/f & 1
\end{bmatrix}
\end{equation}
\item Free propagation
\begin{equation}\label{eqn:modernOpticsLecture1:270}
M =
\begin{bmatrix}
1 & L \\
0 & 1
\end{bmatrix}
\end{equation}
\end{enumerate}

In particular, if imaging something in air, where we have \(n_0 = n_f\) we have \(\Abs{M} = 1\).
\item How about \(\Abs{M} = 0\).  There are a couple of cases.  One is \(D = 0\) where
%
\begin{equation}\label{eqn:modernOpticsLecture1:290}
\alpha_f = C y_i + \cancel{ D \alpha_i}.
\end{equation}
%
output \(\rightarrow\) input is focus, as illustrated in \cref{fig:modernOpticsLecture1:modernOpticsLecture1Fig11}.
\imageFigure{../figures/phy485-optics/modernOpticsLecture1Fig11}{\(D = 0\).}{fig:modernOpticsLecture1:modernOpticsLecture1Fig11}{0.2}
%
%\fxwarning{Review matrix optics figures}{Understand this.}
%
\item We also have zero determinant when \(A = 0\), in which case we have
\begin{equation}\label{eqn:modernOpticsLecture1:310}
y_f = \cancel{ A y_i } + B \alpha_i.
\end{equation}
The output location is only a function of the input angle as illustrated in \cref{fig:modernOpticsLecture1:modernOpticsLecture1Fig12}.
\imageFigure{../figures/phy485-optics/modernOpticsLecture1Fig12}{\(A = 0\).}{fig:modernOpticsLecture1:modernOpticsLecture1Fig12}{0.2}
\item How about if \(B = 0\).  Now we have
\begin{equation}\label{eqn:modernOpticsLecture1:330}
\alpha_f = \cancel{B y_i} + C \alpha_i,
\end{equation}
so that we see that the output is an image of the input, but scaled (a magnifier or reducer).  This is illustrated in \cref{fig:modernOpticsLecture1:modernOpticsLecture1Fig13}.
\imageFigure{../figures/phy485-optics/modernOpticsLecture1Fig13}{\(B = 0\).}{fig:modernOpticsLecture1:modernOpticsLecture1Fig13}{0.2}
\item And finally if \(C = 0\) we have
\begin{equation}\label{eqn:modernOpticsLecture1:350}
\alpha_f = \cancel{C y_i} + D \alpha_i,
\end{equation}
and we find out system is telescopic, magnifying the angle, as illustrated in \cref{fig:modernOpticsLecture1:modernOpticsLecture1Fig14}.
\imageFigure{../figures/phy485-optics/modernOpticsLecture1Fig14}{\(C = 0\).}{fig:modernOpticsLecture1:modernOpticsLecture1Fig14}{0.2}
\end{enumerate}
% had F15, but no description?
%\cref{fig:modernOpticsLecture1:modernOpticsLecture1Fig15}.
%\imageFigure{../figures/phy485-optics/modernOpticsLecture1Fig15}{?}{fig:modernOpticsLecture1:modernOpticsLecture1Fig15}{0.2}
\section{Problems.}
%
% Copyright � 2012 Peeter Joot.  All Rights Reserved.
% Licenced as described in the file LICENSE under the root directory of this GIT repository.
%
% pick one:
%\input{../assignment.tex}
%\input{../blogpost.tex}
%\renewcommand{\basename}{snellsFermat}
%\renewcommand{\dirname}{notes/phy485/}
%\newcommand{\dateintitle}{}
%\newcommand{\keywords}{}
%
%\input{../peeter_prologue_print2.tex}
%
%\beginArtNoToc
%
%\generatetitle{Derivation of Snell's law using Fermat's theorem}
\label{chap:snellsFermat}

\makeproblem{Derive Snell's law.}{snellsFermat:pr:1}{
Fermat's theorem, that light takes the path of least time, can be used to derive Snell's law without resorting to Maxwell's equations.

Note that a proof of Fermat's theorem using the Ray equation can be found in \S 3.3.2 \citep{born1980principles}.
}
\makeanswer{snellsFermat:pr:1}{

We refer to \cref{fig:snellsFermat:snellsFermatFig1}, and seek to express the optical path length.
%
\imageFigure{../figures/phy485-optics/snellsFermatFig1}{Snell's law light paths.}{fig:snellsFermat:snellsFermatFig1}{0.3}
%
Since \(n(s) = c/v(s)\), the time spent along any portion of the path is proportional to \(n(s) ds\).  For the two leg linear route that is
%
\begin{dmath}\label{eqn:snellsFermat:10}
OPL = n r + n' r'.
\end{dmath}
%
Since
%
\begin{subequations}
\begin{dmath}\label{eqn:snellsFermat:30}
r = \sqrt{ h^2 + (L - x)^2 }.
\end{dmath}
\begin{dmath}\label{eqn:snellsFermat:50}
r' = \sqrt{ {h'}^2 + x^2 }.
\end{dmath}
\end{subequations}
%
We want to find \(x\) such that
%
\begin{dmath}\label{eqn:snellsFermat:70}
0
= \frac{d(OPL)}{dx}
=
\ddx{} \left(
n \sqrt{ h^2 + (L - x)^2 }
+ n' \sqrt{ {h'}^2 + x^2 }
\right)
=
n \inv{2 r} 2 (L - x)(-1)
+ n' \inv{2 r'} 2 x
=
- n \sin\theta + n' \sin\theta'.
\end{dmath}
%
This gives us

\boxedEquation{eqn:snellsFermat:90}{
n \sin\theta = n' \sin\theta',
}

as desired.
} % makeanswer
\shipoutAnswer

%\vcsinfo
%\EndNoBibArticle

   %\shipoutAnswer

%
% Copyright � 2012 Peeter Joot.  All Rights Reserved.
% Licenced as described in the file LICENSE under the root directory of this GIT repository.
%
% pick one:
%\input{../assignment.tex}
%\input{../blogpost.tex}
%\renewcommand{\basename}{FIXMEbasename}
%\renewcommand{\dirname}{notes/FIXMEwheretodirname/}
%%\newcommand{\dateintitle}{}
%%\newcommand{\keywords}{}
%\input{../peeter_prologue_print2.tex}
%\beginArtNoToc

\makeproblem{Image distance for an ideal lens.}{modernOpticsProblemSet1Attempt1:pr:1}{
For the lens illustrated in \cref{fig:modernOpticsProblemSet1Attempt1:modernOpticsProblemSet1Attempt1Fig1a}, use geometrical arguments to derive the image location formula
\imageFigure{../figures/phy485-optics/modernOpticsProblemSet1Fig1a}{Thin paraxial lens with image in input and output conjugate planes}{fig:modernOpticsProblemSet1Attempt1:modernOpticsProblemSet1Attempt1Fig1a}{0.3}
\begin{equation}\label{eqn:modernOpticsProblemSet1Attempt1:1510}
\inv{s} + \inv{s'} = \inv{f}.
\end{equation}
}

\makeanswer{modernOpticsProblemSet1Attempt1:pr:1}{
%FIXME: rework.
%My interpretation of this problem is that are given the transfer matrix from class
%
%\begin{dmath}\label{eqn:modernOpticsProblemSet1Attempt1:1030}
%M =
%\begin{bmatrix}
%1 & 0 \\
%-1/f & 1
%\end{bmatrix},
%\end{dmath}
%
%describing the input/output response of a thin lens in the paraxial approximation.
%
%
%At the point \((0,a)\) with a horizontal input ray, our transfer matrix gives us
%
%\begin{dmath}\label{eqn:modernOpticsProblemSet1Attempt1:1050}
%\begin{bmatrix}
%a \\
%-\beta
%\end{bmatrix}
%=
%\begin{bmatrix}
%1 & 0 \\
%-1/f & 1
%\end{bmatrix}
%\begin{bmatrix}
%a \\
%0
%\end{bmatrix}.
%\end{dmath}
%
%The angular portion is
%
%\begin{dmath}\label{eqn:modernOpticsProblemSet1Attempt1:1070}
%-\beta = -\frac{a}{f} + 0,
%\end{dmath}
%
%or
%
%\begin{dmath}\label{eqn:modernOpticsProblemSet1Attempt1:1090}
%\beta = \frac{a}{f}.
%\end{dmath}
%
%Similarly, when the output is horizontal at the point \((0, -b)\) we have
%
%\begin{dmath}\label{eqn:modernOpticsProblemSet1Attempt1:1110}
%\begin{bmatrix}
%-b \\
%0
%\end{bmatrix}
%=
%\begin{bmatrix}
%1 & 0 \\
%-1/f & 1
%\end{bmatrix}
%\begin{bmatrix}
%-b \\
%-\theta
%\end{bmatrix}.
%\end{dmath}
%
%The angular component of this product is
%\begin{dmath}\label{eqn:modernOpticsProblemSet1Attempt1:1130}
%0 = \frac{b}{f} - \theta
%\end{dmath}
%
%or
%\begin{dmath}\label{eqn:modernOpticsProblemSet1Attempt1:1150}
%\theta = \frac{b}{f}.
%\end{dmath}
%
%Because this describes a paraxial system, these angles approximate the tangents, so we can recast \eqnref{eqn:modernOpticsProblemSet1Attempt1:1090} and \eqnref{eqn:modernOpticsProblemSet1Attempt1:1150} as
%
%\begin{subequations}
%\begin{dmath}\label{eqn:modernOpticsProblemSet1Attempt1:1170}
%\tan \beta = \frac{a}{f}
%\end{dmath}
%\begin{dmath}\label{eqn:modernOpticsProblemSet1Attempt1:1190}
%\tan \theta = \frac{b}{f}.
%\end{dmath}
%\end{subequations}
%
%Looking back to the figure, we see that we have \(b/n = \tan \theta\), so by \eqnref{eqn:modernOpticsProblemSet1Attempt1:1190} \(n = f\), the focal length.  Similarly \(a/o = \tan \beta\), so by \eqnref{eqn:modernOpticsProblemSet1Attempt1:1170}, we have \(o = f\), also the focal length.  The geometry of the figure associated with image formation is now fully determined by the transfer matrix, and we are free to extract some additional relations.
%In particular
Note that we have
%
\begin{subequations}
\label{eqn:modernOpticsProblemSet1Attempt1:1210a}
\begin{equation}\label{eqn:modernOpticsProblemSet1Attempt1:1210}
\tan \theta = \frac{a}{x} = \frac{b}{f} = \frac{a + b}{s}
\end{equation}
\begin{equation}\label{eqn:modernOpticsProblemSet1Attempt1:1230}
\tan \beta = \frac{b}{x'} = \frac{a}{f} = \frac{a + b}{s'}
\end{equation}
\end{subequations}

We have respectively
%
\begin{subequations}
\label{eqn:modernOpticsProblemSet1Attempt1:1250a}
\begin{dmath}\label{eqn:modernOpticsProblemSet1Attempt1:1250}
\frac{b}{a} = \frac{f}{x}
\end{dmath}
\begin{dmath}\label{eqn:modernOpticsProblemSet1Attempt1:1270}
\frac{b}{a} = \frac{x'}{f}
\end{dmath}
\end{subequations}

Now we can eliminate the \(x\) and \(x'\) variables using \(x = s - f\) and \(x' = s' - f\)
%
\begin{equation}\label{eqn:modernOpticsProblemSet1Attempt1:1290}
\frac{b}{a} = \frac{f}{s - f} = \frac{s' -f}{f}.
\end{equation}

Rearranging we have
%
\begin{equation}\label{eqn:modernOpticsProblemSet1Attempt1:1310}
f^2 = (s' - f)(s - f) = s s' - f s - s' f + f^2,
\end{equation}

or
%
\begin{equation}\label{eqn:modernOpticsProblemSet1Attempt1:1330}
s s' = f s + f s'.
\end{equation}

Dividing through by \(f s s'\) we have
%
\begin{equation}\label{eqn:modernOpticsProblemSet1Attempt1:1350}
\inv{f} = \inv{s'} + \inv{s},
\end{equation}

as expected.
} % makeanswer

\shipoutAnswer
\makeproblem{Newton's lens focus formula.}{modernOpticsProblemSet1Attempt1:pr:2}{
Demonstrate the equivalence of Newton's lens focus formula to the inverse distance result shown above.
Also referring to the figure where the distances \(x\) and \(x'\) were labeled, show that this is equivalent to
\begin{equation}\label{eqn:modernOpticsProblemSet1Attempt1:1530}
x x' = f^2.
\end{equation}
}

\makeanswer{modernOpticsProblemSet1Attempt1:pr:2}{
%FIXME: did I have another figure?
In the class notes, the magnification of the thin lens system described by \eqnref{eqn:modernOpticsProblemSet1Attempt1:1350} was given by
\begin{equation}\label{eqn:modernOpticsProblemSet1Attempt1:1370}
m = -\frac{s'}{s}.
\end{equation}
Other than the negation, this is a logical definition, the ratio of the output size with respect to the input size.  I'm guessing that this is defined as negated because the image is inverted.  From \eqnref{eqn:modernOpticsProblemSet1Attempt1:1270} we have
\begin{equation}\label{eqn:modernOpticsProblemSet1Attempt1:1390}
\frac{b}{a} = \frac{x'}{f}.
\end{equation}

Dividing the last two equalities in \eqnref{eqn:modernOpticsProblemSet1Attempt1:1210a} we have
\begin{equation}\label{eqn:modernOpticsProblemSet1Attempt1:1410}
\frac{b}{a} = \frac{s'}{s}.
\end{equation}

We can conclude that the magnification, expressed in \(x'\) and \(f\) is
%
\begin{equation}\label{eqn:modernOpticsProblemSet1Attempt1:1430}
m = -\frac{s'}{s} = \frac{x'}{f},
\end{equation}

and that \cref{fig:modernOpticsProblemSet1Attempt1:modernOpticsProblemSet1Attempt1Fig1a} with the distances as labeled describes the same system.  The remainder of the task is therefore algebraic.  We have
%
\begin{dmath}\label{eqn:modernOpticsProblemSet1Attempt1:1450}
0 = -\inv{f} + \inv{s} + \inv{s'} = -\inv{f} + \inv{x + f} + \inv{x' + f}.
\end{dmath}

Multiplying through by \(f(x + f)(x' + f)\) we have
%
\begin{dmath}\label{eqn:modernOpticsProblemSet1Attempt1:1470}
0
= -(x + f)(x' + f) + f(x' + f) + f(x + f)
= -x x' -\cancel{f x'} - \cancel{f x} - \cancel{f^2} + \cancel{f x'} + \cancel{f^2} + \cancel{f x} + f^2,
\end{dmath}

or
%
\begin{dmath}\label{eqn:modernOpticsProblemSet1Attempt1:1490}
x x' = f^2.
\end{dmath}
} % makeanswer

\shipoutAnswer
%\EndNoBibArticle

%   \shipoutAnswer

%
% Copyright © 2012 Peeter Joot.  All Rights Reserved.
% Licenced as described in the file LICENSE under the root directory of this GIT repository.
%
% pick one:
%\input{../assignment.tex}
%\input{../blogpost.tex}
%\renewcommand{\basename}{convexNonParaxial}
%\renewcommand{\dirname}{notes/phy485/}
%%\newcommand{\dateintitle}{}
%%\newcommand{\keywords}{}
%
%\input{../peeter_prologue_print2.tex}
%
%\beginArtNoToc
%
%\generatetitle{FIXME put title here}
\label{chap:convexNonParaxial}

Let's try to solve \cref{modernOpticsLecture1:pr:10}.

\makeproblem{Solve the geometry of a convex lens exactly.}{modernOpticsLecture1:pr:10}{ Solve this lens geometry exactly and show how we obtain the ABCD matrix result in the limit.}

\makeanswer{modernOpticsLecture1:pr:10}{
Our geometry is illustrated in \cref{fig:convexNonParaxial:convexNonParaxialFig1}.
\imageFigure{../figures/phy485-optics/convexNonParaxialFig1}{Convex lens refraction.}{fig:convexNonParaxial:convexNonParaxialFig1}{0.3}
%
We have using the law of sines
\begin{dmath}\label{eqn:convexNonParaxial:20}
\frac{\sin\theta_1}{R}
=
\frac{\sin(\pi - \theta)}{s + R}
=
\frac{\sin\theta}{s + R},
\end{dmath}
%
and also have
\begin{dmath}\label{eqn:convexNonParaxial:40}
\frac{\sin\theta_2}{R} = \frac{\sin\phi}{s' - R}
\end{dmath}
%
Dividing these we have a Snell's law ratio
%
\begin{equation}\label{eqn:convexNonParaxial:100}
\frac{n'}{n}
=
\frac{\sin\theta}{\sin\phi}
=
\frac{
(s + R) \frac{\sin\theta_1}{R} }
{
(s' - R)\frac{\sin\theta_2}{R}
},
\end{equation}
%
or
\boxedEquation{eqn:convexNonParaxial:80}{
\frac{n'}{n}
=
\frac{
(s + R)
}
{
(s' - R)
}
\frac{
\sin\theta_1
}
{
\sin\theta_2
}.
}

This is the exact result.  Let's verify that this matches our paraxial ABCD matrix result.

Introducing signed angles
\begin{subequations}
\begin{dmath}\label{eqn:convexNonParaxial:120}
\alpha = \theta_1
\end{dmath}
\begin{dmath}\label{eqn:convexNonParaxial:140}
\alpha' = -\theta_2.
\end{dmath}
\end{subequations}
%
The paraxial approximation gives
%
\begin{subequations}
\begin{equation}\label{eqn:convexNonParaxial:160}
s' = \frac{y}{\sin\theta_2} \sim -\frac{y}{\alpha'}
%\sin\theta_2 = \frac{y}{s'}
\end{equation}
\begin{equation}\label{eqn:convexNonParaxial:180}
s = \frac{y}{\sin\theta_1} \sim \frac{y}{\alpha}
%\sin\theta_1 = \frac{y}{s}.
\end{equation}
\end{subequations}
%
\begin{dmath}\label{eqn:convexNonParaxial:200}
\frac{n'}{n}
=
-\frac{
(s + R)
}
{
(s' - R)
}
\frac{
\alpha
}
{
\alpha'
}
=
-\frac{
\left( \frac{y}{\alpha} + R \right)
}
{
\left(-\frac{y}{\alpha'} - R \right)
}
\frac{
\alpha
}
{
\alpha'
}
=
\frac{
\left( y + R \alpha \right)
}
{
\left( y + R \alpha' \right)
}.
\end{dmath}
%
We want to solve for \(\alpha'\), and find
%
\begin{dmath}\label{eqn:convexNonParaxial:220}
\alpha'
= -\frac{y}{R} + \frac{n}{n'} \left( \alpha + \frac{y}{R} \right)
= \frac{y}{R} \left( \frac{n}{n'} - 1 \right) + \frac{n}{n'} \alpha
\end{dmath}
%
With \(y = y'\) we have in matrix form

\boxedEquation{eqn:convexNonParaxial:240}{
\begin{bmatrix}
y' \\
\alpha '
\end{bmatrix}
=
\begin{bmatrix}
1 & 0 \\
\frac{1}{R} \left( \frac{n}{n'} - 1 \right) & \frac{n}{n'}
\end{bmatrix}
\begin{bmatrix}
y \\
\alpha
\end{bmatrix},
}

which is the ABCD result as desired.
} % makeanswer

%\EndNoBibArticle

   %\shipoutAnswer

%
% Copyright � 2012 Peeter Joot.  All Rights Reserved.
% Licenced as described in the file LICENSE under the root directory of this GIT repository.
%
% pick one:
%\input{../assignment.tex}
%\input{../blogpost.tex}
%\renewcommand{\basename}{concaveReflection}
%\renewcommand{\dirname}{notes/FIXMEwheretodirname/}
%%\newcommand{\dateintitle}{}
%%\newcommand{\keywords}{}
%\input{../peeter_prologue_print2.tex}
%\beginArtNoToc
%\generatetitle{FIXME put title here}

\makeproblem{Solve the geometry of a concave spherical mirror.}{concaveReflection:pr:1}{
After solving, apply the paraxial approximation to find the ABCD matrix result. }

\makeanswer{concaveReflection:pr:1}{
Our system is illustrated in \cref{fig:concaveReflection:concaveReflectionFig1}.
%
\imageFigure{../figures/phy485-optics/concaveReflectionFig1}{Concave spherical reflector}{fig:concaveReflection:concaveReflectionFig1}{0.3}
%
From the figure, employing the law of sines, we have
%
\begin{equation}\label{eqn:concaveReflection:10}
t \sin\theta = r \sin\alpha = (r - s') \sin\theta_2,
\end{equation}
%
or
\begin{dmath}\label{eqn:concaveReflection:30}
\frac{r - s'}{\sqrt{s^2 + y^2}} \sin\theta_2
= \sin( \gamma + \theta_1)
= \sin \gamma \cos \theta_1
+ \cos \gamma \sin \theta_1,
\end{dmath}
%
but since
\begin{equation}\label{eqn:concaveReflection:50}
\gamma = \Atan \frac{y}{s},
\end{equation}
%
and
%
\begin{subequations}
\begin{equation}\label{eqn:concaveReflection:70}
\cos\Atan x = \inv{\sqrt{1 + x^2}}
\end{equation}
\begin{equation}\label{eqn:concaveReflection:90}
\sin\Atan x = \frac{x}{\sqrt{1 + x^2}},
\end{equation}
\end{subequations}
%
we have
\begin{dmath}\label{eqn:concaveReflection:110}
\frac{r - s'}{\sqrt{s^2 + y^2}} \sin\theta_2
=
\frac{y/s}{\sqrt{1 + (y/s)^2}} \cos \theta_1
+\frac{1}{\sqrt{1 + (y/s)^2}} \sin \theta_1
=
\frac{y}{\sqrt{s^2 + y^2}} \cos \theta_1
+\frac{s}{\sqrt{s^2 + y^2}} \sin \theta_1,
\end{dmath}
%
or
\boxedEquation{eqn:concaveReflection:130}{
(r - s') \sin\theta_2 = y \cos \theta_1 + s \sin \theta_1.
}

This is the exact result desired.  Application of the paraxial approximation gives us
%
\begin{equation}\label{eqn:concaveReflection:150}
(r - s') \theta_2 \sim y + s \theta_1.
\end{equation}
%
With
%
\begin{subequations}
\begin{equation}\label{eqn:concaveReflection:170}
\theta_2 \sim \frac{y'}{s'}
\end{equation}
\begin{equation}\label{eqn:concaveReflection:190}
\theta_1 \sim \frac{\Delta y}{r + s}
\end{equation}
\end{subequations}
%
we have
\begin{equation}\label{eqn:concaveReflection:210}
s \theta_1 \sim -r \theta_1 + \Delta y
\end{equation}
\begin{equation}\label{eqn:concaveReflection:230}
s' \theta_2 \sim y'
\end{equation}
%
\begin{equation}\label{eqn:concaveReflection:250}
r \theta_2 - y' \sim y - r \theta_1 + \Delta y,
\end{equation}
%
or
\begin{equation}\label{eqn:concaveReflection:270}
\theta_2 \sim \frac{2}{r} y' - \theta_1
\end{equation}
%
We need to fix the sign conventions for the ABCD matrices, so write
%
\begin{subequations}
\begin{equation}\label{eqn:concaveReflection:290}
\alpha' = -\theta_2
\end{equation}
\begin{equation}\label{eqn:concaveReflection:310}
R = -r
\end{equation}
\begin{equation}\label{eqn:concaveReflection:330}
\alpha = \theta_1
\end{equation}
\end{subequations}
%
A final substitution into \eqnref{eqn:concaveReflection:270} gives us
\begin{equation}\label{eqn:concaveReflection:270b}
\alpha' \sim \frac{2}{R} y' + \alpha
\end{equation}
%
or in matrix form
\begin{equation}\label{eqn:concaveReflection:350}
\begin{bmatrix}
y' \\
\alpha'
\end{bmatrix}
=
\begin{bmatrix}
1 & 0 \\
\frac{2}{R} & 1
\end{bmatrix}
\begin{bmatrix}
y \\
\alpha
\end{bmatrix}.
\end{equation}
%
This is the ABCD matrix we were given in class.
} % makeanswer

%\EndNoBibArticle

   %\shipoutAnswer

%
% Copyright © 2012 Peeter Joot.  All Rights Reserved.
% Licenced as described in the file LICENSE under the root directory of this GIT repository.
%

\makeoproblem{ABCD Matrices.}{modernOptics:problemSet1:1}{2012 Ps1, P1}{
Using the ABCD matrices from the lecture, prove these well-known rules of geometric optics. In each case, {\bf make an illustration}, tracing some important rays that illustrate the rule.
\begin{enumerate}
\item[(a)] {\em An image is formed when \(1/f = 1/s_o + 1/s_i\).} Solve this problem using the result we found in class: when B=0 for a system matrix, the input and output are conjugate planes.
\item[(b)] {\em An image with magnification \(-x'/f\) is formed when \(x x' = f^2\).} Repeat part (a), but in ``Newton's form'': replace \(s_o\) with \(f + x\), and replace \(s_i\) with \(f + x'\).
\item[(c)] {\em The position distribution at the focus of a lens is the angular position of the incident beam.} (In other words, a lens does a kind of Fourier transform, as you may know already.) Find where the input plane has to be located for \(y_o = f \alpha_i\).
\item[(d)] {\em Two identical lenses spaced by \(2f\) image an object at \(f\) with unity magnification.}
\item[(e)] Two identical lenses spaced by \(2f\) are {\em telecentric}, meaning that an object at \(f+x\) from the first lens has a magnification independent of \(x\), in contrast to a simple lens.
\item[(f)] A lens and a flat mirror spaced by distance \(f\) create a {\em cat's eye}. What are its properties? Consider, in particular, an emitter located \(f\) in front of the Cat's eye and located at \(y_i = 0\).
\end{enumerate}
} % makeoproblem

\makeanswer{modernOptics:problemSet1:1}{

\begin{enumerate}
\item[(a)]

Our system and the associated transfer matrices labels are illustrated in \cref{fig:modernOpticsProblemSet1:modernOpticsProblemSet1Fig1aTake2}.
%
\imageFigure{../figures/phy485-optics/modernOpticsProblemSet1Fig1aTake2}{Input and output conjugate planes for paraxial thin lens}{fig:modernOpticsProblemSet1:modernOpticsProblemSet1Fig1aTake2}{0.3}
%
We form the system transfer matrix by applying first a free propagation matrix, then a thin lens paraxial matrix, and one more free propagation matrix
%
\begin{dmath}\label{eqn:modernOpticsProblemSet1P1:1790}
M
= M_3 M_2 M_1
=
\begin{bmatrix}
1 & s' \\
0 1
\end{bmatrix}
\begin{bmatrix}
1 & 0 \\
-1/f & 1
\end{bmatrix}
\begin{bmatrix}
1 & s \\
0 1
\end{bmatrix}
=
\begin{bmatrix}
1 - s'/f & s' \\
-1/f & 1
\end{bmatrix}
\begin{bmatrix}
1 & s \\
0 1
\end{bmatrix}
=
\begin{bmatrix}
1 - s'/f & s + s' - s s'/f \\
-1/f & -s/f + 1
\end{bmatrix}.
\end{dmath}
%
Consider ray \((B)\) from the figure, where we have
%
\begin{dmath}\label{eqn:modernOpticsProblemSet1P1:1810}
\begin{bmatrix}
0 \\
\alpha
\end{bmatrix}
\rightarrow
\alpha
\begin{bmatrix}
s + s' - s s'/f \\
-s/f + 1
\end{bmatrix}
=
\begin{bmatrix}
0 \\
\alpha'
\end{bmatrix}.
\end{dmath}
%
With
%
\begin{equation}\label{eqn:modernOpticsProblemSet1P1:1830}
y = y' = \alpha ( s + s' - s s'/f ) = 0,
\end{equation}
%
for all \(\alpha\).  We must have
%
\begin{equation}\label{eqn:modernOpticsProblemSet1P1:1850}
s + s' = \frac{s s'}{f}.
\end{equation}
%
Dividing through by \(s s'\) we have

\boxedEquation{eqn:modernOpticsProblemSet1P1:1870}{
\inv{s'} + \inv{s} = \inv{f},
}

as expected.

\item[(b)]

Let's consider the system as the compound action of three transfer matrices as illustrated in \cref{fig:modernOpticsProblemSet1:modernOpticsProblemSet1Fig2b1}, this time labeling the figure in terms of the variables for this problem.
%
\imageFigure{../figures/phy485-optics/modernOpticsProblemSet1Fig2b1}{Newton's form, an image with magnification}{fig:modernOpticsProblemSet1:modernOpticsProblemSet1Fig2b1}{0.3}
%
Compounding the transfer matrices we have
%
\begin{dmath}\label{eqn:modernOpticsProblemSet1P1:1510}
M
= M_3 M_2 M_1
=
\begin{bmatrix}
1 & x' + f \\
0 & 1
\end{bmatrix}
\begin{bmatrix}
1 & 0 \\
-1/f & 1
\end{bmatrix}
\begin{bmatrix}
1 & x + f \\
0 & 1
\end{bmatrix}
=
\begin{bmatrix}
-x'/f & x' + f \\
-1/f & 1
\end{bmatrix}
\begin{bmatrix}
1 & x + f \\
0 & 1
\end{bmatrix}
=
-\inv{f}
\begin{bmatrix}
x' & x x' - f^2 \\
1 & x
\end{bmatrix}.
\end{dmath}
%
Consider the ray \(A\) where the effect is
%
\begin{dmath}\label{eqn:modernOpticsProblemSet1P1:1530}
\begin{bmatrix}
y \\
0
\end{bmatrix}
\rightarrow
-\inv{f}
\begin{bmatrix}
y x' \\
y
\end{bmatrix}.
\end{dmath}
%
We see that \(y' = -y x'/f\) or

\boxedEquation{eqn:modernOpticsProblemSet1P1:1550}{
m = - \frac{x'}{f} = \frac{y'}{y}.
}

The quantity defined as the magnification is in fact the ratio of the output to the input size as intuitively expected.  Now consider a ray \(C\) originating at \(y = 0\) at the image source, and landing at \(y = 0\) on the conjugate output plane.  For this ray we have
%
\begin{equation}\label{eqn:modernOpticsProblemSet1P1:1570}
\begin{bmatrix}
0 \\
\alpha
\end{bmatrix}
\rightarrow
-\inv{f}
\begin{bmatrix}
x x' -f^2 \\
x
\end{bmatrix} \theta
=
\begin{bmatrix}
0 \\
\alpha'
\end{bmatrix}.
\end{equation}
%
Since this holds for all input angles originating at \(y = 0\) from the input plane, we must have
\boxedEquation{eqn:modernOpticsProblemSet1P1:1590}{
x x' = f^2,
}

as desired.

\item[(c)]

Here we refer to \cref{fig:modernOpticsProblemSet1:modernOpticsProblemSet1Fig1c}, this time considering no ray that passes the focus past the lens.  Our system transfer matrix, given the reduced free propagation distance past the lens is
\imageFigure{../figures/phy485-optics/modernOpticsProblemSet1Fig1c}{Position distribution at the focus of a lens}{fig:modernOpticsProblemSet1:modernOpticsProblemSet1Fig1c}{0.3}
%
\begin{dmath}\label{eqn:modernOpticsProblemSet1P1:1610}
M
= M_3 M_2 M_1
=
\begin{bmatrix}
1 & f \\
0 & 1
\end{bmatrix}
\begin{bmatrix}
1 & 0 \\
-1/f & 1
\end{bmatrix}
\begin{bmatrix}
1 & x + f \\
0 & 1
\end{bmatrix}
=
\begin{bmatrix}
0 & f \\
-1/f & 1
\end{bmatrix}
\begin{bmatrix}
1 & x + f \\
0 & 1
\end{bmatrix}
=
\begin{bmatrix}
0 & f \\
-1/f & -x/f
\end{bmatrix}.
\end{dmath}
%
A ray is transformed according to
\begin{dmath}\label{eqn:modernOpticsProblemSet1P1:1630}
\begin{bmatrix}
y \\
\theta
\end{bmatrix}
\rightarrow
\begin{bmatrix}
0 & f \\
-1/f & -x/f
\end{bmatrix}
\begin{bmatrix}
y \\
\theta
\end{bmatrix}
=
\begin{bmatrix}
f \theta \\
-\inv{f} ( y - x \theta )
\end{bmatrix}.
\end{dmath}
%
In particular
\boxedEquation{eqn:modernOpticsProblemSet1P1:1650}{
y' = f \theta,
}

demonstrating the claim that at the focus, the position is an angular distribution of the incident beam.  This is clearly independent of \(x\) so the input plane position is irrelevant.

\item[(d)]

Consider \cref{fig:modernOpticsProblemSet1:modernOpticsProblemSet1Fig1d}.
%
\imageFigure{../figures/phy485-optics/modernOpticsProblemSet1Fig1d}{Two identical lenses separated by twice focus}{fig:modernOpticsProblemSet1:modernOpticsProblemSet1Fig1d}{0.2}
%
The transfer matrix \(M = M_5 M_4 M_3 M_2 M_1\) for the system is
%
\begin{dmath}\label{eqn:modernOpticsProblemSet1P1:1670}
M
= M_5 M_4 M_3 M_2 M_1
=
\begin{bmatrix}
1 & x + f \\
0 & 1
\end{bmatrix}
\begin{bmatrix}
1 & 0 \\
-1/f & 1
\end{bmatrix}
\begin{bmatrix}
1 & 2 f \\
0 & 1
\end{bmatrix}
\begin{bmatrix}
1 & 0 \\
-1/f & 1
\end{bmatrix}
\begin{bmatrix}
1 & x + f \\
0 & 1
\end{bmatrix}
=
\begin{bmatrix}
-x/f & x + f \\
-1/f & 1
\end{bmatrix}
\begin{bmatrix}
1 & 2 f \\
0 & 1
\end{bmatrix}
\begin{bmatrix}
1 & x + f \\
-1/f & -x/f
\end{bmatrix}
=
\begin{bmatrix}
-x/f & -x + f \\
-1/f & -1
\end{bmatrix}
\begin{bmatrix}
1 & x + f \\
-1/f & -x/f
\end{bmatrix}
=
\begin{bmatrix}
-1 & -2 x \\
0 & -1
\end{bmatrix}.
\end{dmath}
%
Consider any ray from the source going towards the lens along the horizontal.  We have
%
\begin{dmath}\label{eqn:modernOpticsProblemSet1P1:1690}
\begin{bmatrix}
y \\
0
\end{bmatrix}
\rightarrow
\begin{bmatrix}
-y \\
0
\end{bmatrix},
\end{dmath}
%
The ratio of the output to the input height to be

\boxedEquation{eqn:modernOpticsProblemSet1P1:1710}{
\frac{y'}{y} = -1,
}

which is the unit magnitude magnification as desired.

\item[(e)]

This is actually demonstrated above.

\item[(f)]

Here we consider \cref{fig:modernOpticsProblemSet1:modernOpticsProblemSet1Fig1f}.
%
\imageFigure{../figures/phy485-optics/modernOpticsProblemSet1Fig1f}{Cat's eye.  Lens with mirror behind at focus}{fig:modernOpticsProblemSet1:modernOpticsProblemSet1Fig1f}{0.3}
%
Our system transfer matrix is
%
\begin{dmath}\label{eqn:modernOpticsProblemSet1P1:1730}
M =
M_7
M_6
M_5
M_4
M_3
M_2
M_1
=
\begin{bmatrix}
1 & s' \\
0 & 1
\end{bmatrix}
\begin{bmatrix}
1 & 0 \\
-1/f & 1
\end{bmatrix}
\begin{bmatrix}
1 & f \\
0 & 1
\end{bmatrix}
\begin{bmatrix}
1 & 0 \\
0 & 1
\end{bmatrix}
\begin{bmatrix}
1 & f \\
0 & 1
\end{bmatrix}
\begin{bmatrix}
1 & 0 \\
-1/f & 1
\end{bmatrix}
\begin{bmatrix}
1 & s \\
0 & 1
\end{bmatrix}
=
\begin{bmatrix}
1 - s'/f & s' \\
-1/f & 1
\end{bmatrix}
\begin{bmatrix}
1 & 2 f \\
0 & 1
\end{bmatrix}
\begin{bmatrix}
1 & s \\
-1/f & -s/f + 1
\end{bmatrix}
=
\begin{bmatrix}
1 - s'/f & 2f - s' \\
-1/f & -1
\end{bmatrix}
\begin{bmatrix}
1 & s \\
-1/f & -s/f + 1
\end{bmatrix}
=
\begin{bmatrix}
-1 & 2 f - s' -s \\
0 & -1
\end{bmatrix}
\end{dmath}
%
We see that the angle of the output light is unchanged except for sign, so we have no scattering in the paraxial limit.  Observe that if the emitter is positioned at \(s = f\) we have
%
\begin{dmath}\label{eqn:modernOpticsProblemSet1P1:1750}
M =
\begin{bmatrix}
-1 & f - s' \\
0 & -1
\end{bmatrix}
\end{dmath}
%
so
%
\begin{dmath}\label{eqn:modernOpticsProblemSet1P1:1770}
y' = -y + (f -s') \alpha.
\end{dmath}
%
The image is magnified (negatively) for any position \(\Abs{s'} > f\) without any angular distortion.  In fact, if the observation is also made at the focus, then the image magnification is unity.  Notice that at the focus we have both a sign change in the position and the angle coordinate, meaning that the output image is exactly the same as in the input image.  In retrospect, this is exactly the same system mathematically as the \(2f\) spaced lenses of parts (d) and (e), and we could have done the matrix products just once for all those parts of the problem!

\end{enumerate}
} % makeanswer


   %\shipoutAnswer

%\vcsinfo
%\EndArticle
