%
% Copyright � 2012 Peeter Joot.  All Rights Reserved.
% Licenced as described in the file LICENSE under the root directory of this GIT repository.
%
%\input{../blogpost.tex}
%\renewcommand{\basename}{modernOpticsLecture22}
%\renewcommand{\dirname}{notes/phy485/}
%\newcommand{\keywords}{Optics, PHY485H1F}
%\input{../peeter_prologue_print2.tex}
%
%\usepackage{tikz}
%\usepackage[draft]{fixme}
%\fxusetheme{color}
%
%\beginArtNoToc
%\generatetitle{PHY485H1F Modern Optics.  Lecture 22: Number of photons per free space mode.  Taught by Prof.\ Joseph Thywissen}
%%\chapter{Number of photons per free space mode}
\index{photon density}
%\label{chap:modernOpticsLecture22}
%
%%\section{Disclaimer}
%
%Peeter's lecture notes from class.  May not be entirely coherent.

\section{Number of photons per free space mode.}
%
%\fxwarning{review lecture 22}{work through this lecture in detail.}
%
Laser is fundamentally characterized by a large number of photons per free space mode.  Possible because photons are bosons!  Laser light has temporal and spatial coherence.

\paragraph{Review:} \textunderline{in a cavity}

\begin{itemize}
\item Laser light: \(\expectation{n} \sim 10^7 - 10^8\) above threshold \cref{fig:modernOpticsLecture22:modernOpticsLecture22Fig1}.
\item Thermal light: \(\expectation{n} = \inv{ e^{\Hbar \omega/\kB T} - 1 }\).
\end{itemize}

\makeexample{Some numbers}{example:modernOpticsLecture22:1}{
\begin{equation}\label{eqn:modernOpticsLecture22:20}
\expectation{n} \sim 1
\end{equation}
\begin{equation}\label{eqn:modernOpticsLecture22:40}
\Hbar \omega \sim \kB T \ln 2
\end{equation}
\begin{equation}\label{eqn:modernOpticsLecture22:60}
300 \txtK: \lambda > 70 \mu \txtm
\end{equation}
\begin{equation}\label{eqn:modernOpticsLecture22:80}
5000 \txtK: \lambda > 4 \mu \txtm
\end{equation}
}
%
\imageFigure{../figures/phy485-optics/modernOpticsLecture22Fig1}{Threshold.}{fig:modernOpticsLecture22:modernOpticsLecture22Fig1}{0.3}
%
\paragraph{3D Free space mode.}

Consider ``elementary pencil'' of light as in \cref{fig:modernOpticsLecture22:modernOpticsLecture22Fig2}, chosen to be diffraction limited.
%
\imageFigure{../figures/phy485-optics/modernOpticsLecture22Fig2}{Pencil of light.}{fig:modernOpticsLecture22:modernOpticsLecture22Fig2}{0.3}
%
Here

\begin{align}\label{eqn:modernOpticsLecture22:100}
(\Delta x \Delta k_x)_{\mathrm{min}} &\sim 1 \\
(\Delta y \Delta k_y)_{\mathrm{min}} &\sim 1
\end{align}

where \(1\) here means a constant of order \(1\).

Along 2: consider pulse train as in \cref{fig:modernOpticsLecture22:modernOpticsLecture22Fig3}.
%
\imageFigure{../figures/phy485-optics/modernOpticsLecture22Fig3}{Pulse train.}{fig:modernOpticsLecture22:modernOpticsLecture22Fig3}{0.2}
%
where \(\Delta z = c \tau_0\)
%
\begin{dmath}\label{eqn:modernOpticsLecture22:180}
\Psi(t) =
\left\{
\begin{array}{l l}
e^{-i \omega_0 t} & \quad \mbox{\(t \in [-\tau_0/2, \tau_0/2]\)} \\
0 & \quad \mbox{otherwise}
\end{array}
\right.
\end{dmath}
%
What is \(\Delta k_z\)?  Making a paraxial approximation
%
\begin{dmath}\label{eqn:modernOpticsLecture22:120}
\Delta k_z \approx \Delta k = \frac{\Delta \omega}{c}
\end{dmath}
%
We can compute the Fourier transform as depicted in \cref{fig:modernOpticsLecture22:modernOpticsLecture22Fig4}.
%
\imageFigure{../figures/phy485-optics/modernOpticsLecture22Fig4}{Fourier transform of pulse train.}{fig:modernOpticsLecture22:modernOpticsLecture22Fig4}{0.2}
%
\begin{dmath}\label{eqn:modernOpticsLecture22:140}
g(\omega)
= \int e^{i \omega t} \Psi(t) dt
= z \frac{\sin \left( (\omega - \omega_0) \tau_0/2 \right) }{\omega - \omega_0}
\end{dmath}
%
So
%
\begin{dmath}\label{eqn:modernOpticsLecture22:160}
\Delta z \Delta k_z
=
\left( c \tau_0
\right)
\left(
\frac{2\pi}{c \tau_0}
\right)
= 2 \pi
\end{dmath}
%
Now consider source that has width \(\Gamma\).  Imagine emission of pulses of length \(\tau_0 = 2 \pi/\Gamma\) \cref{fig:modernOpticsLecture22:modernOpticsLecture22Fig5}.
%
\imageFigure{../figures/phy485-optics/modernOpticsLecture22Fig5}{Overlapping pulses.}{fig:modernOpticsLecture22:modernOpticsLecture22Fig5}{0.2}
%
\paragraph{Laser light}:
\index{Laser light}
%
\begin{dmath}\label{eqn:modernOpticsLecture22:200}
\frac{\text{number of photons}}{\text{free space mode}}
=
\frac{P/\Hbar \omega}{\Gamma_{\mathrm{laser}}}
\sim \frac{\Gamma_{\mathrm{cav}} \expectation{n}}{\Gamma_{\mathrm{cav}}/\expectation{n}}
\sim \expectation{n}^2
\end{dmath}
%
Justifying this last operation \cref{fig:modernOpticsLecture22:modernOpticsLecture22Fig6}.
%
\imageFigure{../figures/phy485-optics/modernOpticsLecture22Fig6}{Why this division?.}{fig:modernOpticsLecture22:modernOpticsLecture22Fig6}{0.3}
%
%\EndArticle
%\EndNoBibArticle
