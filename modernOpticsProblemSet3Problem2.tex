%
% Copyright � 2013 Peeter Joot.  All Rights Reserved.
% Licenced as described in the file LICENSE under the root directory of this GIT repository.
%
\makeoproblem{Inside the Fabry Perot.}
{modernOptics:problemSet3:2}
{2012 Ps3, P2}
{
What does the energy density \(u(x) \equiv \langle | \Psi(x,t) |^2 \rangle\) look like {\em inside} the Fabry Perot Etalon? Assume a monochromatic traveling wave
\(\Psi(x,t) = \sqrt{u_0} \exp{(i k x - i \omega t)}\) is normally incident on a cavity. The mirrors are two surfaces with transmission
\(t = \sqrt{T} \exp{(i \delta_t)}\), reflectivity
\(r = \sqrt{R} \exp{(i \delta_r)}\), and
\(T=1-R\). The length of the cavity is
\(L\). As in class, the round-trip phase shift will be called
\(\Delta\).
\makesubproblem{
{\bf Show that}
\(u(x)\) in the cavity can be expressed in the form

\begin{dmath}\label{eqn:modernOptics:problemSet3:2:20}
u = u_0 A \, \left(1 + R + 2 \sqrt{R} \cos [\phi(x)] \right),
\end{dmath}

and {\bf give} \(A\) and
\(\phi(x)\) in terms of
\(R\),
\(\Delta\), and
\(k\). This should explain why we say the
\(m\)th order resonance is when there are
\(m\) standing-wave anti-nodes in the cavity.
}{modernOptics:problemSet3:2a}

\makesubproblem{Peak energy density}{modernOptics:problemSet3:2b}

Find an expression the {\bf peak energy density}
\(u_{\mathrm{max}}\) in the cavity, and {\bf plot}
\(u_{\mathrm{max}}/ u_0\) versus
\(\Delta\), for
\(R=0.8\). What is the {\bf resonance condition}? {\bf By how much} can
\(u_{\mathrm{max}}\) exceed
\(u_0\)? Where is this extra energy density coming from if energy is conserved?

\makesubproblem{Standing waves?}{modernOptics:problemSet3:2c}

Is there {\bf always a standing wave} in the cavity, even off resonance? Write an expression for the {\bf visibility}, (max-min)/(max+min), of the energy density
\(u(x)\). One way to measure this visibility would be to look at the optical forces on an atom in the cavity.

\makesubproblem{Compare to class}{modernOptics:problemSet3:2d}

Finally, make sure your expression also makes sense when compared to the
\(I_T\) we found in class. Since it's only the forward-going component that is transmitted through the last mirror, the intensity outside the cavity is
\(u_0 A T\). (You don't need to prove this.) {\bf Does this reproduce
\(I_T\)?}
} % makeoproblem

\makeanswer{modernOptics:problemSet3:2}{
\makeSubAnswer{Average intensity inside the cavity}{modernOptics:problemSet3:2a}

Referring to \cref{fig:modernOpticsProblemSet3Problem2:modernOpticsProblemSet3Problem2Fig1} (where the internal reflections are exaggerated), we want to look at the field after the wave gets to points
\(1\), reflects to
\(2\), reflects again to
\(3\) and so forth.

\imageFigure{../figures/phy485-optics/modernOpticsProblemSet3Problem2Fig1}{Internal Fabry-Perot field geometry}{fig:modernOpticsProblemSet3Problem2:modernOpticsProblemSet3Problem2Fig1}{0.3}

Our electric field at point
\(x\) is then the sum

\begin{equation}\label{eqn:modernOptics:problemSet3:2:40}
\begin{array}{l l l}
\Psi(x)
&= (\Psi_0 t) e^{i k x} & + (\Psi_0 t) r e^{i k (2 L -x)} \\
&+ (\Psi_0 t) r^2 e^{i k (2 L + x)} & + (\Psi_0 t) r^3 e^{i k (4 L -x)} \\
&+ (\Psi_0 t) r^4 e^{i k (4 L + x)} & + (\Psi_0 t) r^5 e^{i k (6 L -x)} \\
& \cdots &
\end{array}
\end{equation}

Factoring out the geometric series we have
\begin{dmath}\label{eqn:modernOptics:problemSet3:2:60}
\Psi(x)
=
(\Psi_0 t)
\left(
1 + r^2 e^{2 i k L} + r^4 e^{4 i k L} + \cdots
\right)
\left(
e^{i k x} + r e^{i k (2 L - x)}
\right)
=
(\Psi_0 t) \inv{1 - r^2 e^{ 2 i k L} }
\left(
e^{i k x} + r e^{i k (2 L - x)}
\right)
=
\frac{\Psi_0 \sqrt{T} e^{i \delta_t} }{1 - R e^{ 2 i k L + 2 i \delta_r } }
\left(
e^{i k x} + \sqrt{R} e^{i k (2 L - x) + \delta_r}
\right).
\end{dmath}

Introducing a round trip phase

\begin{dmath}\label{eqn:modernOptics:problemSet3:2:80}
\Delta \equiv 2 k L + 2 \delta_r,
\end{dmath}

we have

\begin{dmath}\label{eqn:modernOptics:problemSet3:2:100}
\Psi(x)
=
\frac{\Psi_0 \sqrt{T} e^{i \delta_t} }{1 - R e^{ i \Delta} }
\left(
e^{i k x} + \sqrt{R} e^{-i k x + i \Delta - i \delta_r}
\right).
\end{dmath}

This has squared magnitude
\boxedEquation{eqn:modernOptics:problemSet3:2:120}{
\Abs{\Psi(x)}^2
=
\frac{u_0 T }{1 + R^2 - 2 \cos(\Delta) }
\biglr{
1 + R + 2 \sqrt{R} \cos\biglr{ 2 k x - \Delta +
\mathLabelBox{
\delta_r
}{Wrong?}
}
}
}.
}
\footnote{The \( \delta_r\) above might be wrong, as it was circled by the grader with a question mark.  Recalculate.}

This has the desired structure of \eqnref{eqn:modernOptics:problemSet3:2:20} with
\begin{subequations}
\begin{dmath}\label{eqn:modernOptics:problemSet3:2:140}
A = \frac{T}{1 + R^2 - 2 \cos(\Delta) }
\end{dmath}
\begin{dmath}\label{eqn:modernOptics:problemSet3:2:160}
\phi(x) = 2 k x - \Delta + \delta_r
\end{dmath}
\end{subequations}

\makeSubAnswer{Peak energy density}{modernOptics:problemSet3:2b}

From \eqnref{eqn:modernOptics:problemSet3:2:120} we see that maximums and minimums occur respectively whenever

\begin{subequations}
\begin{dmath}\label{eqn:modernOptics:problemSet3:2:180}
2 k x - \Delta + \Delta_r = \pi (2 m)
\end{dmath}
\begin{dmath}\label{eqn:modernOptics:problemSet3:2:200}
2 k x - \Delta + \Delta_r = \pi (2 m  + 1)
\end{dmath}
\end{subequations}

At these points \(1 + R + 2 \sqrt{R} \cos\left( 2 k x - \Delta + \delta_r \right)\) takes the values

\begin{subequations}
\begin{dmath}\label{eqn:modernOptics:problemSet3:2:220}
1 + R + 2 \sqrt{R} = (1 + \sqrt{R})^2
\end{dmath}
\begin{dmath}\label{eqn:modernOptics:problemSet3:2:240}
1 + R - 2 \sqrt{R} = (1 - \sqrt{R})^2
\end{dmath}
\end{subequations}

The maximum and minimum energy densities are therefore

\begin{subequations}
\label{eqn:modernOptics:problemSet3:2:250}
\begin{dmath}\label{eqn:modernOptics:problemSet3:2:260}
u_{\mathrm{max}}
=
\frac{u_0 (1 - R) }{1 + R^2 - 2 \cos(\Delta) }
\left(
1 + \sqrt{R}
\right)^2
\end{dmath}
\begin{dmath}\label{eqn:modernOptics:problemSet3:2:280}
u_{\mathrm{min}}
=
\frac{u_0 (1 - R) }{1 + R^2 - 2 \cos(\Delta) }
\left(
1 - \sqrt{R}
\right)^2
\end{dmath}
\end{subequations}

This maximum is plotted in \cref{fig:modernOpticsProblemSet3Problem2:modernOpticsProblemSet3Problem2Fig2}.

\imageFigure{../figures/phy485-optics/modernOpticsProblemSet3Problem2Fig2}{peak energy density}{fig:modernOpticsProblemSet3Problem2:modernOpticsProblemSet3Problem2Fig2}{0.3}

\paragraph{Grading note: \(-1\)}
Marked ``??''.  Check against posted solution.

We see the resonance peaks when
\(1 + R^2 - 2 \cos\Delta = 0\).  Those points are

\begin{dmath}\label{eqn:modernOptics:problemSet3:2:320}
\Delta_{\mathrm{res}} = \pm \cos^{-1} \left( \frac{1 + R^2}{2} \right).
= 4 \pi \frac{L}{\lambda_{\mathrm{res}}} + 2 \delta_r
\end{dmath}

For a given interface phase shift
\(\delta_r\), and reflectivity
\(R\), and cavity width
\(L\) we see that we have a critical wavelength

\begin{dmath}\label{eqn:modernOptics:problemSet3:2:340}
\lambda_{\mathrm{res}}
=\frac{4 \pi L}{
\pm \cos^{-1} \left( \frac{1 + R^2}{2} \right) - 2 \delta_r
}
\end{dmath}

If large amounts of energy are supplied to the field due to this resonance, I think it would have to come from interactions with the interfaces, thermally cooling the atoms in the mirrors.  These thermal effects likely change
\(r\) as a side effect,
\textunderline{producing a feedback effect that would prevent the potential infinite spikes that we see in plot and associated expression of}
\(u_{\mathrm{max}}\).

\paragraph{Grading note: (\(-2\)}
Check against posted solution.

\makeSubAnswer{Standing waves and visibility}{modernOptics:problemSet3:2c}

From \eqnref{eqn:modernOptics:problemSet3:2:120} we see that our energy density has the form

\begin{dmath}\label{eqn:modernOptics:problemSet3:2:440}
u(x)
= \alpha + \beta \cos\left( \frac{4 \pi}{\lambda} (x - L) + \delta_r \right).
\end{dmath}

We'll have \textunderline{standing waves (possibly phase shifted) only for those input wavelengths that satisfy}

\begin{dmath}\label{eqn:modernOptics:problemSet3:2:460}
\frac{4 \pi L}{\lambda} = m \pi,
\end{dmath}

for integer \(m > 0\).

\paragraph{Grading note: $-2$}
Underline portion with question ``why?''.

From \eqnref{eqn:modernOptics:problemSet3:2:250} we see that our visibility is

\begin{dmath}\label{eqn:modernOptics:problemSet3:2:300}
\calV = \frac
{
1 + R + 2 \sqrt{R} - (1 + R - 2 \sqrt{R})
}
{
1 + R + 2 \sqrt{R} + 1 + R - 2 \sqrt{R}
}
=
\frac{2 \sqrt{R}}{1 + R}
\end{dmath}

\makeSubAnswer{Compare to previously calculated transmitted intensity}{modernOptics:problemSet3:2d}

If we split our wave function into forward \(\Psi_{+}(x)\) and reverse
\(\Psi_{-}(x)\) components we have from \eqnref{eqn:modernOptics:problemSet3:2:100}

\begin{subequations}
\begin{dmath}\label{eqn:modernOptics:problemSet3:2:360}
\Psi_{+}(x)
=
\frac{\Psi_0 \sqrt{T} e^{i \delta_t} }{1 - R e^{ i \Delta} }
e^{i k x}
\end{dmath}
\begin{dmath}\label{eqn:modernOptics:problemSet3:2:380}
\Psi_{-}(x)
=
\frac{\Psi_0 \sqrt{T} e^{i \delta_t} }{1 - R e^{ i \Delta} }
\sqrt{R} e^{-i k x + i \Delta - i \delta_r}
\end{dmath}
\end{subequations}

The externally transmitted portion of this wave is

\begin{dmath}\label{eqn:modernOptics:problemSet3:2:400}
\Psi_T
=
t \Psi_{+}(L)
=
\frac{\Psi_0 T e^{2 i \delta_t} }{1 - R e^{ i \Delta} }
e^{i k L},
\end{dmath}

which has squared magnitude

\begin{dmath}\label{eqn:modernOptics:problemSet3:2:420}
I_T
= \Abs{\Psi_T}^2
= \frac{u_0 T^2}{\Abs{1 - R e^{i \Delta}}^2}.
\end{dmath}

Except for the notation change \(u_0 \leftrightarrow I_0\), this, as expected, reproduces the result from class.
} % makeanswer
