%
% Copyright � 2012 Peeter Joot.  All Rights Reserved.
% Licenced as described in the file LICENSE under the root directory of this GIT repository.
%
% pick one:
%\input{../assignment.tex}
%\input{../blogpost.tex}
%\renewcommand{\basename}{optics2010ExamFraunhofer}
%\renewcommand{\dirname}{notes/phy485/}
%%\newcommand{\dateintitle}{}
%%\newcommand{\keywords}{}
%
%\input{../peeter_prologue_print2.tex}
%
%\beginArtNoToc
%
%\generatetitle{Fraunhofer diffraction pattern for four circular apertures}
%\chapter{Fraunhofer diffraction pattern for four circular apertures}
\index{Fraunhofer diffraction}
\index{circular aperture}
%\label{chap:\basename}
%\section{Motivation}
%\section{Guts}

\makeoproblem{Fraunhofer diff., 4 circular apertures.}
{pr:optics2010ExamFraunhofer:1}
{2010 final, q3}
{
Calculate the diffraction pattern for the geometry of \cref{fig:optics2010ExamFraunhofer:optics2010ExamFraunhoferFig2}.
\imageFigure{../figures/phy485-optics/optics2010ExamFraunhoferFig2}{Four circular apertures.}{fig:optics2010ExamFraunhofer:optics2010ExamFraunhoferFig2}{0.3}
} % makeoproblem

\makeanswer{pr:optics2010ExamFraunhofer:1}{
We are working with distances illustrated in \cref{fig:optics2010ExamFraunhofer:optics2010ExamFraunhoferFig1}.
\imageFigure{../figures/phy485-optics/optics2010ExamFraunhoferFig1}{Four apertures with observation point and distances.}{fig:optics2010ExamFraunhofer:optics2010ExamFraunhoferFig1}{0.3}
As usual we write
\begin{subequations}
\begin{dmath}\label{eqn:optics2010ExamFraunhofer:20}
\BR = \Br' - \Br_s.
\end{dmath}
\begin{dmath}\label{eqn:optics2010ExamFraunhofer:40}
R
=
r' \left( 1 + \frac{r_s^2}{{r'}^2} - 2 \frac{\Br_s \cdot \Br'}{{r'}^2} \right)^{1/2}
\approx
r' + \frac{r_s^2}{2 {r'}} - \Br_s \cdot \rcap'.
\end{dmath}
\end{subequations}
%
so that
%
\begin{dmath}\label{eqn:optics2010ExamFraunhofer:60}
\Psi(\Br') = \frac{\Psi_0}{e^{i k r'}}{i \lambda r'} \int_A e^{-i k \Br_s \cdot \rcap'}.
\end{dmath}
%
Let's write
%
\begin{subequations}
\begin{dmath}\label{eqn:optics2010ExamFraunhofer:80}
\rcap \cdot \xcap = \alpha.
\end{dmath}
\begin{dmath}\label{eqn:optics2010ExamFraunhofer:100}
\rcap \cdot \ycap = \beta,
\end{dmath}
\end{subequations}
%
and introduce an aperture function
%
\begin{dmath}\label{eqn:optics2010ExamFraunhofer:120}
g(x, y) =
\left\{
\begin{array}{l l}
1 & \quad \mbox{if \(x^2 + y^2 \le R^2\)} \\
0 & \quad \mbox{otherwise}
\end{array}
\right.
\end{dmath}
%
This allows us to write our diffraction integral as
\begin{dmath}\label{eqn:optics2010ExamFraunhofer:140}
\begin{aligned}
\Psi(\Br')
&=
\frac{\Psi_0}{e^{i k r'}}{i \lambda r'}
\int du dv
\Biglr{
e^{-i k ((u + b/2) \alpha + (v + b/2) \beta) }
+
e^{-i k ((u - b/2) \alpha + (v - b/2) \beta) } \\
&\qquad +
e^{-i k ((u + b/2) \alpha + (v - b/2) \beta) }
+
e^{-i k ((u - b/2) \alpha + (v + b/2) \beta) }
}
\\
&=
\frac{\Psi_0}{e^{i k r'}}{i \lambda r'}
\Biglr{
   e^{-i k (\alpha b/ 2 + \beta b/2) }
   +
   e^{-i k (-\alpha b/ 2 - \beta b/2) } \\
&\qquad   +
   e^{-i k (\alpha b/ 2 - \beta b/2) }
   +
   e^{-i k (-\alpha b/ 2 + \beta b/2) }
} \times \\
&\qquad \int du dv
e^{-i k (u \alpha + v \beta) } \\
&=
2 \frac{\Psi_0}{e^{i k r'}}{i \lambda r'}
\left(
e^{-i k \alpha b/ 2 } \cos( k \beta b/2 )
+
e^{i k \alpha b/ 2 } \cos( k \beta b/2 )
\right)
\int du dv
e^{-i k (u \alpha + v \beta) } \\
&=
4 \frac{\Psi_0}{e^{i k r'}}{i \lambda r'}
\cos( k \alpha b/ 2 ) \cos( k \beta b/2 )
\int_{\rho=0}^R \int_{\theta = 0}^{2 \pi}
\rho d\rho d\theta
e^{-i k \rho (\cos\theta \alpha + \sin \theta \beta) }.
\end{aligned}
\end{dmath}
%
This last integral isn't something that we can evaluate in just Bessel functions unless one of \(\alpha\) or \(\beta\) is zero.  For example, if \(\beta = 0\), so that the observation axis lies in the one of the perpendicular planes, then we have
\begin{dmath}\label{eqn:optics2010ExamFraunhofer:160}
\Psi \sim
\frac{e^{i k r'}}{r'}
\cos( k \alpha b/ 2 )
\int_{\rho=0}^R \int_{\theta = 0}^{2 \pi}
\rho d\rho d\theta
e^{-i k \rho \cos\theta \alpha }
=
\frac{e^{i k r'}}{r'}
\cos( k \alpha b/ 2 )
2 \pi R \frac{J_1( -k \alpha R)}{ -k \alpha }
=
\frac{e^{i k r'}}{r'}
\cos( k \alpha b/ 2 )
2 \pi R \frac{J_1( k \alpha R)}{ k \alpha }.
\end{dmath}
%
This looks fairly \(\sinc\) like \cref{fig:optics2010ExamFraunhofer:optics2010ExamFraunhoferFig3}.
%
\imageFigure{../figures/phy485-optics/optics2010ExamFraunhoferFig3}{Plot of \(J_1(x)/x\).}{fig:optics2010ExamFraunhofer:optics2010ExamFraunhoferFig3}{0.3}
%
We can also solve for the case when \(\alpha = \beta\), because we can write
%
\begin{dmath}\label{eqn:optics2010ExamFraunhofer:180}
\cos\theta \pm \sin\theta = \sqrt{2} \cos(\theta \mp \pi/4).
\end{dmath}
%
The phase shift doesn't make a difference when we are integrating over \([0, 2 \pi]\), so we are left with
%
\begin{dmath}\label{eqn:optics2010ExamFraunhofer:170}
\Psi \sim
\frac{e^{i k r'}}{r'}
\cos( k \alpha b/ 2 )
\int_{\rho=0}^R \int_{\theta = 0}^{2 \pi}
=
\frac{e^{i k r'}}{r'}
\cos( k \alpha b/ 2 )
2 \pi R \frac{J_1( \sqrt{2} k \alpha R)}{ \sqrt{2} k \alpha }.
\end{dmath}
%
For arbitrary \(\alpha\) and \(\beta\) there's no such obvious change of variables.  Mathematica calls the result a regularized hypergeometric function
%
\begin{dmath}\label{eqn:optics2010ExamFraunhofer:200}
\Psi \sim
\frac{e^{i k r'}}{r'}
\cos( k \alpha b/ 2 )
\pi R^2 \, _0\tilde{F}_1\left(;2;-\frac{1}{4} k^2 R^2 \left(\alpha^2+\beta^2\right)\right),
\end{dmath}
%
The hypergeometric function itself looks fairly sinc like, but not with the \(R^2\) multiplicative factor (plotted as a function of \(R\)).  This is plotted in \cref{fig:optics2010ExamFraunhofer:optics2010ExamFraunhoferFig4}, but curiously, this appears to be a divergent function?  On the exam, I expect that the expectation was just to look on axis, but it would probably also be useful to plug in some actually representitive numbers.
%
\imageFigure{../figures/phy485-optics/optics2010ExamFraunhoferFig4}{Plot of hypergeometric function.}{fig:optics2010ExamFraunhofer:optics2010ExamFraunhoferFig4}{0.3}
} % makeanswer

% this is to produce the sites.google url and version info and so forth (for blog posts)
%\vcsinfo
%\EndArticle
%\EndNoBibArticle
