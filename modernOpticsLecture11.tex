%
% Copyright � 2012 Peeter Joot.  All Rights Reserved.
% Licenced as described in the file LICENSE under the root directory of this GIT repository.
%
%\input{../blogpost.tex}
%\renewcommand{\basename}{modernOpticsLecture11}
%\renewcommand{\dirname}{notes/phy485/}
%\newcommand{\keywords}{Optics, PHY485H1F}
%\input{../peeter_prologue_print2.tex}
%\beginArtNoToc
%\generatetitle{PHY485H1F Modern Optics.  Lecture 11: Spatial Coherence.  Taught by Prof.\ Joseph Thywissen}
%%\chapter{Spatial Coherence}
%\label{chap:modernOpticsLecture11}
%
%\section{Disclaimer}
%
%Peeter's lecture notes from class.  May not be entirely coherent.
%
\section{Spatial Coherence (cont.)}

We want to look more at spatial distribution of sources, and the effects of that spread on the coherence.  In general we'd have to deal with both spectral width as well as spatial distribution, but here we choose to only dealing with the spatial distribution.  Consider a pair of point sources as in \cref{fig:modernOpticsLecture11:modernOpticsLecture11Fig1}.
%
\imageFigure{../figures/phy485-optics/modernOpticsLecture11Fig1}{Spatial inferometry with Lloyd's mirror.}{fig:modernOpticsLecture11:modernOpticsLecture11Fig1}{0.2}
%
\begin{dmath}\label{eqn:modernOpticsLecture11:10}
\Gamma_{12} = \expectation{
\Psi^\conj( t, \Br_1)
\Psi( t, \Br_2)
}.
\end{dmath}
%
At the photodetector, we have
%
\begin{dmath}\label{eqn:modernOpticsLecture11:30}
I = I_1 + I_2 + 2 \Real \Gamma_{12}.
\end{dmath}
%
This last bit \(2 \Real \Gamma_{12}\) predicts the fringe.

Last time, as plotted in \cref{fig:modernOpticsLecture11:modernOpticsLecture11Fig2}, we found
%
\begin{dmath}\label{eqn:modernOpticsLecture11:50}
\Abs{\gamma_{12}} = \Abs{ \cos \left( \frac{\omega(\tau_a - \tau_b)}{2} \right) }.
\end{dmath}
%
%\imageFigure{../figures/phy485-optics/modernOpticsLecture11Fig2}{\(\Abs{\gamma_{12}}\).}{fig:modernOpticsLecture11:modernOpticsLecture11Fig2}{0.2}
%
This absolute value of \(\gamma_{12}\) is telling us what the visibility of the fringes is.  Suppose we move around source \(a\) so that we are changing \(c \tau_a\).  This will give us something like \cref{fig:modernOpticsLecture11:modernOpticsLecture11Fig3} where the envelope is
%
\imageFigure{../figures/phy485-optics/modernOpticsLecture11Fig3}{Intensity.}{fig:modernOpticsLecture11:modernOpticsLecture11Fig3}{0.2}
%
\begin{dmath}\label{eqn:modernOpticsLecture11:70}
\cos \left( \frac{\omega(\tau_a - \tau_b)}{2} \right).
\end{dmath}
%
and the fast phase frequencies oscillate with the higher frequency
%
\begin{dmath}\label{eqn:modernOpticsLecture11:90}
\cos \left( \frac{\omega(\tau_a + \tau_b)}{2} \right).
\end{dmath}
%
To see this we need to calculate this difference in \(\tau\)'s.
%
\begin{dmath}\label{eqn:modernOpticsLecture11:110}
\tau_a - \tau_b = \frac{
r_{1a}
-r_{2a}
-r_{1b}
+r_{2b}
}{c}
= \text{see Prof's notes}
=
\frac{l s}{c z}
=
\frac{l \theta_s}{c }.
\end{dmath}
%
In the last step we've used a small angle approximation
%
\begin{dmath}\label{eqn:modernOpticsLecture11:130}
\theta_s \approx \frac{s}{z}.
\end{dmath}
%
This is one way to calculate this difference, and we see that in the limit of a single point source, this difference
%
\begin{dmath}\label{eqn:modernOpticsLecture11:150}
\tau_a - \tau_b \rightarrow 0,
\end{dmath}
and
\begin{dmath}\label{eqn:modernOpticsLecture11:170}
\Abs{\gamma_{12}} \rightarrow 1,
\end{dmath}
as we expect, so the visibility of the fringe disappears (illustrated with figure?)
%
%\fxwarning{review Leonid's comment}{Leonid: This explanation doesn't seem exactly right. The fringes' visibility should not disappears.  \(\Gamma_{12}\) goes to 0.5 (FWHM) and that is the definition of coherence length. Does it make sense?}
%
When
%
\begin{dmath}\label{eqn:modernOpticsLecture11:190}
\Delta \tau = \frac{\lambda}{2 c}.
\end{dmath}
%
when
%
\begin{dmath}\label{eqn:modernOpticsLecture11:210}
l \theta_s = \frac{\lambda}{2}.
\end{dmath}
%
We can do this in a more general way, using some math we already know if we think about these two sources separated by some vector \(\Br_s\), going to two points, again with vector separation \(\Delta \Br\), as in \cref{fig:modernOpticsLecture11:modernOpticsLecture11Fig4}.
%
\imageFigure{../figures/phy485-optics/modernOpticsLecture11Fig4}{Vector spatial coherence diagram.}{fig:modernOpticsLecture11:modernOpticsLecture11Fig4}{0.2}
%
We also introduce a vector average \(\Br_{\mathrm{av}}\) from point \(a\) to the midpoint of \(1\) and \(2\).  Note that we are preparing for a setup with an extended source where we'll be integrating over points \(a\) so we don't want this midpoint to start from the average of \(a\) and \(b\) as may be expected.

We find in \cref{lecture11VectorDifference:pr:1}
%
\begin{dmath}\label{eqn:modernOpticsLecture11:230}
k \left( R_1 - R_2 \right) \approx -k \left( \Br_{\mathrm{av}} - \Br_s \right) \cdot \frac{ \Delta \Br}{r_{\mathrm{av}}},
\end{dmath}
where \(\BR_1\), and \(\BR_2\) are the vectors from \(b\) to \(1\) and \(2\) respectively.  
%\fxwarning{review signs}{This difference is worked in the notes, but with a sign error compared to the figure that is corrected above.  Signs below should be reviewed.}
This includes an overall phase shift
%
\begin{dmath}\label{eqn:modernOpticsLecture11:250}
\Bk_{\mathrm{av}} \cdot \Delta \Br,
\end{dmath}
%
and a shift
%
\begin{dmath}\label{eqn:modernOpticsLecture11:270}
k \frac{\Br_s \cdot \Delta \Br}{r_{\mathrm{av}}}.
\end{dmath}
%
\section{What's so special about this pathlength difference?}
\index{pathlength difference}
%
\begin{dmath}\label{eqn:modernOpticsLecture11:290}
\Delta \tau = \tau_a - \tau_b = \frac{\lambda}{2c}.
\end{dmath}
%
Let's consider the fringes that we make, using an interfometer such as Lloyd's mirror, from source \(a\) made from these two points \cref{fig:modernOpticsLecture11:modernOpticsLecture11Fig5}.
%
\imageFigure{../figures/phy485-optics/modernOpticsLecture11Fig5}{Path length differences.}{fig:modernOpticsLecture11:modernOpticsLecture11Fig5}{0.2}
%
Could set things up so that the phase of the pairs of contributions are exactly opposite in phase as in \cref{fig:modernOpticsLecture11:modernOpticsLecture11Fig6}.
%
\imageFigure{../figures/phy485-optics/modernOpticsLecture11Fig6}{Opposing phase contributions eliminating fringes.}{fig:modernOpticsLecture11:modernOpticsLecture11Fig6}{0.2}
%
Our intensities could then add to produce no fringes as in \cref{fig:modernOpticsLecture11:modernOpticsLecture11Fig7}.
%
\imageFigure{../figures/phy485-optics/modernOpticsLecture11Fig7}{Fringe elimination.}{fig:modernOpticsLecture11:modernOpticsLecture11Fig7}{0.2}
%
\section{Continuum spatial distribution.}
\index{spatial distribution}

Our small angle source contributions \cref{fig:modernOpticsLecture11:modernOpticsLecture11Fig8} \cref{fig:modernOpticsLecture11:modernOpticsLecture11Fig10} provide us with one sinusoidal term
%
\imageFigure{../figures/phy485-optics/modernOpticsLecture11Fig8}{small \(\theta_s\).}{fig:modernOpticsLecture11:modernOpticsLecture11Fig8}{0.2}
\imageFigure{../figures/phy485-optics/modernOpticsLecture11Fig10}{One frequency contribution.}{fig:modernOpticsLecture11:modernOpticsLecture11Fig10}{0.2}
%
whereas our larger angular spreads \cref{fig:modernOpticsLecture11:modernOpticsLecture11Fig9} will give us more terms, none that will reduce the main peak.
%
\imageFigure{../figures/phy485-optics/modernOpticsLecture11Fig9}{larger \(\theta_s\).}{fig:modernOpticsLecture11:modernOpticsLecture11Fig9}{0.2}
%
In the sum \cref{fig:modernOpticsLecture11:modernOpticsLecture11Fig11} we may end up with something like \cref{fig:modernOpticsLecture11:modernOpticsLecture11Fig12}.
%
\imageFigure{../figures/phy485-optics/modernOpticsLecture11Fig11}{Many contributions.}{fig:modernOpticsLecture11:modernOpticsLecture11Fig11}{0.2}
\imageFigure{../figures/phy485-optics/modernOpticsLecture11Fig12}{Resulting superposition.}{fig:modernOpticsLecture11:modernOpticsLecture11Fig12}{0.2}
%
Our total intensity is
%
\begin{dmath}\label{eqn:modernOpticsLecture11:310}
I_{\mathrm{total}} = \sum_{\Bk}
\Abs{
\Psi_k(\Br_1, t) + e^{i \Bk \cdot \Bl} \Psi_k(\Br_1, t)
}^2
=
\sum \left(
2 I_{\Bk} + 2 \Real \left( \Psi_k^\conj e^{i \Bk \cdot \Bl} \Psi_k \right)
     \right)
= \text{incoherent sum} + 2 \Real \left(
\sum_{\Bk} I_{\Bk} e^{i \Bk \cdot \Bl}
\right).
\end{dmath}
%
Here
%
\begin{dmath}\label{eqn:modernOpticsLecture11:330}
\sum_{\Bk} I_{\Bk} e^{i \Bk \cdot \Bl} = \Gamma_{12} = \calF( I_{\Bk} ).
\end{dmath}
%
This is called the \underlineAndIndex{Van Cittert-Zernike Theorem}.  Like the time domain result, we have something that essentially says that the coherence is a Fourier transform of the distribution.

Our next task will be to extend this result to continuous spatial distributions as in \cref{fig:modernOpticsLecture11:modernOpticsLecture11Fig13}.
%
\imageFigure{../figures/phy485-optics/modernOpticsLecture11Fig13}{Spatial distribution.}{fig:modernOpticsLecture11:modernOpticsLecture11Fig13}{0.2}
%

%\EndArticle
%\EndNoBibArticle
