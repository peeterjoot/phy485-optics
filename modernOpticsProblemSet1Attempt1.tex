%
% Copyright � 2012 Peeter Joot.  All Rights Reserved.
% Licenced as described in the file LICENSE under the root directory of this GIT repository.
%
% pick one:
%\input{../assignment.tex}
%\input{../blogpost.tex}
%\renewcommand{\basename}{FIXMEbasename}
%\renewcommand{\dirname}{notes/FIXMEwheretodirname/}
%%\newcommand{\dateintitle}{}
%%\newcommand{\keywords}{}
%\input{../peeter_prologue_print2.tex}
%\beginArtNoToc

\makeproblem{Image distance for an ideal lens.}{modernOpticsProblemSet1Attempt1:pr:1}{
For the lens illustrated in \cref{fig:modernOpticsProblemSet1Attempt1:modernOpticsProblemSet1Attempt1Fig1a}, use geometrical arguments to derive the image location formula
\imageFigure{../figures/phy485-optics/modernOpticsProblemSet1Fig1a}{Thin paraxial lens with image in input and output conjugate planes}{fig:modernOpticsProblemSet1Attempt1:modernOpticsProblemSet1Attempt1Fig1a}{0.3}
\begin{equation}\label{eqn:modernOpticsProblemSet1Attempt1:1510}
\inv{s} + \inv{s'} = \inv{f}.
\end{equation}
}

\makeanswer{modernOpticsProblemSet1Attempt1:pr:1}{
%FIXME: rework.
%My interpretation of this problem is that are given the transfer matrix from class
%
%\begin{dmath}\label{eqn:modernOpticsProblemSet1Attempt1:1030}
%M =
%\begin{bmatrix}
%1 & 0 \\
%-1/f & 1
%\end{bmatrix},
%\end{dmath}
%
%describing the input/output response of a thin lens in the paraxial approximation.
%
%
%At the point \((0,a)\) with a horizontal input ray, our transfer matrix gives us
%
%\begin{dmath}\label{eqn:modernOpticsProblemSet1Attempt1:1050}
%\begin{bmatrix}
%a \\
%-\beta
%\end{bmatrix}
%=
%\begin{bmatrix}
%1 & 0 \\
%-1/f & 1
%\end{bmatrix}
%\begin{bmatrix}
%a \\
%0
%\end{bmatrix}.
%\end{dmath}
%
%The angular portion is
%
%\begin{dmath}\label{eqn:modernOpticsProblemSet1Attempt1:1070}
%-\beta = -\frac{a}{f} + 0,
%\end{dmath}
%
%or
%
%\begin{dmath}\label{eqn:modernOpticsProblemSet1Attempt1:1090}
%\beta = \frac{a}{f}.
%\end{dmath}
%
%Similarly, when the output is horizontal at the point \((0, -b)\) we have
%
%\begin{dmath}\label{eqn:modernOpticsProblemSet1Attempt1:1110}
%\begin{bmatrix}
%-b \\
%0
%\end{bmatrix}
%=
%\begin{bmatrix}
%1 & 0 \\
%-1/f & 1
%\end{bmatrix}
%\begin{bmatrix}
%-b \\
%-\theta
%\end{bmatrix}.
%\end{dmath}
%
%The angular component of this product is
%\begin{dmath}\label{eqn:modernOpticsProblemSet1Attempt1:1130}
%0 = \frac{b}{f} - \theta
%\end{dmath}
%
%or
%\begin{dmath}\label{eqn:modernOpticsProblemSet1Attempt1:1150}
%\theta = \frac{b}{f}.
%\end{dmath}
%
%Because this describes a paraxial system, these angles approximate the tangents, so we can recast \eqnref{eqn:modernOpticsProblemSet1Attempt1:1090} and \eqnref{eqn:modernOpticsProblemSet1Attempt1:1150} as
%
%\begin{subequations}
%\begin{dmath}\label{eqn:modernOpticsProblemSet1Attempt1:1170}
%\tan \beta = \frac{a}{f}
%\end{dmath}
%\begin{dmath}\label{eqn:modernOpticsProblemSet1Attempt1:1190}
%\tan \theta = \frac{b}{f}.
%\end{dmath}
%\end{subequations}
%
%Looking back to the figure, we see that we have \(b/n = \tan \theta\), so by \eqnref{eqn:modernOpticsProblemSet1Attempt1:1190} \(n = f\), the focal length.  Similarly \(a/o = \tan \beta\), so by \eqnref{eqn:modernOpticsProblemSet1Attempt1:1170}, we have \(o = f\), also the focal length.  The geometry of the figure associated with image formation is now fully determined by the transfer matrix, and we are free to extract some additional relations.
%In particular
Note that we have

\begin{subequations}
\label{eqn:modernOpticsProblemSet1Attempt1:1210a}
\begin{equation}\label{eqn:modernOpticsProblemSet1Attempt1:1210}
\tan \theta = \frac{a}{x} = \frac{b}{f} = \frac{a + b}{s}
\end{equation}
\begin{equation}\label{eqn:modernOpticsProblemSet1Attempt1:1230}
\tan \beta = \frac{b}{x'} = \frac{a}{f} = \frac{a + b}{s'}
\end{equation}
\end{subequations}

We have respectively

\begin{subequations}
\label{eqn:modernOpticsProblemSet1Attempt1:1250a}
\begin{dmath}\label{eqn:modernOpticsProblemSet1Attempt1:1250}
\frac{b}{a} = \frac{f}{x}
\end{dmath}
\begin{dmath}\label{eqn:modernOpticsProblemSet1Attempt1:1270}
\frac{b}{a} = \frac{x'}{f}
\end{dmath}
\end{subequations}

Now we can eliminate the \(x\) and \(x'\) variables using \(x = s - f\) and \(x' = s' - f\)

\begin{equation}\label{eqn:modernOpticsProblemSet1Attempt1:1290}
\frac{b}{a} = \frac{f}{s - f} = \frac{s' -f}{f}.
\end{equation}

Rearranging we have

\begin{equation}\label{eqn:modernOpticsProblemSet1Attempt1:1310}
f^2 = (s' - f)(s - f) = s s' - f s - s' f + f^2,
\end{equation}

or

\begin{equation}\label{eqn:modernOpticsProblemSet1Attempt1:1330}
s s' = f s + f s'.
\end{equation}

Dividing through by \(f s s'\) we have

\begin{equation}\label{eqn:modernOpticsProblemSet1Attempt1:1350}
\inv{f} = \inv{s'} + \inv{s},
\end{equation}

as expected.
} % makeanswer

\shipoutAnswer
\makeproblem{Newton's lens focus formula.}{modernOpticsProblemSet1Attempt1:pr:2}{
Demonstrate the equivalence of Newton's lens focus formula to the inverse distance result shown above.
Also referring to the figure where the distances \(x\) and \(x'\) were labeled, show that this is equivalent to
\begin{equation}\label{eqn:modernOpticsProblemSet1Attempt1:1530}
x x' = f^2.
\end{equation}
}

\makeanswer{modernOpticsProblemSet1Attempt1:pr:2}{
%FIXME: did I have another figure?
In the class notes, the magnification of the thin lens system described by \eqnref{eqn:modernOpticsProblemSet1Attempt1:1350} was given by
\begin{equation}\label{eqn:modernOpticsProblemSet1Attempt1:1370}
m = -\frac{s'}{s}.
\end{equation}
Other than the negation, this is a logical definition, the ratio of the output size with respect to the input size.  I'm guessing that this is defined as negated because the image is inverted.  From \eqnref{eqn:modernOpticsProblemSet1Attempt1:1270} we have
\begin{equation}\label{eqn:modernOpticsProblemSet1Attempt1:1390}
\frac{b}{a} = \frac{x'}{f}.
\end{equation}

Dividing the last two equalities in \eqnref{eqn:modernOpticsProblemSet1Attempt1:1210a} we have
\begin{equation}\label{eqn:modernOpticsProblemSet1Attempt1:1410}
\frac{b}{a} = \frac{s'}{s}.
\end{equation}

We can conclude that the magnification, expressed in \(x'\) and \(f\) is

\begin{equation}\label{eqn:modernOpticsProblemSet1Attempt1:1430}
m = -\frac{s'}{s} = \frac{x'}{f},
\end{equation}

and that \cref{fig:modernOpticsProblemSet1Attempt1:modernOpticsProblemSet1Attempt1Fig1a} with the distances as labeled describes the same system.  The remainder of the task is therefore algebraic.  We have

\begin{dmath}\label{eqn:modernOpticsProblemSet1Attempt1:1450}
0 = -\inv{f} + \inv{s} + \inv{s'} = -\inv{f} + \inv{x + f} + \inv{x' + f}.
\end{dmath}

Multiplying through by \(f(x + f)(x' + f)\) we have

\begin{dmath}\label{eqn:modernOpticsProblemSet1Attempt1:1470}
0
= -(x + f)(x' + f) + f(x' + f) + f(x + f)
= -x x' -\cancel{f x'} - \cancel{f x} - \cancel{f^2} + \cancel{f x'} + \cancel{f^2} + \cancel{f x} + f^2,
\end{dmath}

or

\begin{dmath}\label{eqn:modernOpticsProblemSet1Attempt1:1490}
x x' = f^2.
\end{dmath}
} % makeanswer

\shipoutAnswer
%\EndNoBibArticle
