%
% Copyright � 2012 Peeter Joot.  All Rights Reserved.
% Licenced as described in the file LICENSE under the root directory of this GIT repository.
%
%\input{../blogpost.tex}
%\renewcommand{\basename}{modernOpticsLecture15}
%\renewcommand{\dirname}{notes/phy485/}
%\newcommand{\keywords}{Optics, PHY485H1F}
%\input{../peeter_prologue_print2.tex}
%
%\usepackage[draft]{fixme}
%\fxusetheme{color}
%
%\beginArtNoToc
%\generatetitle{PHY485H1F Modern Optics.  Lecture 15: Multiple interference (cont.).  Taught by Prof.\ Joseph Thywissen}
%%\chapter{Multiple interference (cont.)}
%\label{chap:modernOpticsLecture15}
%
%\section{Disclaimer}
%
%Peeter's lecture notes from class.  May not be entirely coherent.

We are going to look at the (Fraunhofer) far field of a diffraction grating with \(N\) illuminated slits \cref{fig:modernOpticsLecture15:modernOpticsLecture15Fig1}.
%
\imageFigure{../figures/phy485-optics/modernOpticsLecture15Fig1}{Diffraction grating interferometry.}{fig:modernOpticsLecture15:modernOpticsLecture15Fig1}{0.2}
%
Our geometry is
%
\begin{dmath}\label{eqn:modernOpticsLecture15:640}
\BR + \Br' = \Br.
\end{dmath}
\begin{dmath}\label{eqn:modernOpticsLecture15:740}
\BR_s + \Br' = \Br_s.
\end{dmath}
%
for which our path length from \(\Br'\) to the observation point is
%
\begin{dmath}\label{eqn:modernOpticsLecture15:660}
\Abs{\BR}
= r \left( 1 + \frac{{r'}^2}{r^2} - 2 \frac{\Br \cdot \Br'}{r^2} \right)^{1/2}
\sim r + \frac{{r'}^2}{2 r^2} - \rcap \cdot \Br',
\end{dmath}
%
and to first order
%
\begin{dmath}\label{eqn:modernOpticsLecture15:680}
k\Abs{\BR}
\sim k r - k \rcap \cdot \Br'
= k r + \Bk \cdot \Br'
= k r + k y' \sin\theta.
\end{dmath}
%
Similarly
\begin{dmath}\label{eqn:modernOpticsLecture15:760}
k\Abs{\BR_s}
\sim k r_s - k \rcap \cdot \Br'
= k r_s - \Bk \cdot \Br'
= k r_s - k \sin\theta_s y'.
\end{dmath}
%
With
%
\begin{dmath}\label{eqn:modernOpticsLecture15:780}
k_y \equiv \frac{2 \pi}{\lambda} \sin\theta,
\end{dmath}
%
and
\begin{dmath}\label{eqn:modernOpticsLecture15:800}
k_{y,s} \equiv \frac{2 \pi}{\lambda} \sin\theta_s,
\end{dmath}
%
our diffraction integral, in one dimension takes the form
%
\begin{dmath}\label{eqn:modernOpticsLecture15:700}
\Psi(\Br) \sim \int e^{i (k_{y,s} - k_{y}) y' } dy'.
\end{dmath}
%
or
\begin{dmath}\label{eqn:modernOpticsLecture15:820}
\Psi(\Br) \sim \int e^{i \frac{2 \pi}{\lambda} (\sin\theta_s - \sin\theta) y' } dy'.
\end{dmath}
%
We'll work with \(\theta_s = 0\), normally incident plane waves, for which the diffraction integral reduces to just
%
\begin{dmath}\label{eqn:modernOpticsLecture15:720}
\Psi(\Br) \sim \int e^{-i k_y y' } dy'.
\end{dmath}
%
This is the Fourier transform of the aperture, say \(g(y)\) evaluated at \(k_y = (2 \pi/\lambda) \sin\theta\).  Note that \(k_y \ne \Bk \cdot \rcap'\), it is not the projection in the \(\ycap\) direction, which is \(k \sin\theta_s\).  The angle \(\theta\) here is the angle to the \textunderline{observation} point.

We will use the convolution theorem \ref{pr:convolution:1}, constructing the complete aperture as a convolution, with transmission \cref{fig:modernOpticsLecture15:modernOpticsLecture15Fig2}.
%
\imageFigure{../figures/phy485-optics/modernOpticsLecture15Fig2}{Convolution of box with comb.}{fig:modernOpticsLecture15:modernOpticsLecture15Fig2}{0.2}
%
\begin{dmath}\label{eqn:modernOpticsLecture15:20}
f(y) = \int dy' g(y') h(y - y') = g(y) \conj h(y).
\end{dmath}
%
Consider the convolution with the delta function comb
%
\begin{dmath}\label{eqn:modernOpticsLecture15:40}
h(y) = \sum_{n=0}^{N-1} \delta (y - n a).
\end{dmath}
%
So that the convolution is
%
\begin{dmath}\label{eqn:modernOpticsLecture15:60}
f(y)
= \int dy' g(y') \sum_{n = 0}^{N-1} \delta(y - y' - n a)
=
\sum_{n = 0}^{N-1}
\int dy' g(y')
\delta(y - y' - n a)
=
\sum_{n = 0}^{N-1}
g(y - n a).
\end{dmath}
%
Convolution theorem
%
\begin{dmath}\label{eqn:modernOpticsLecture15:80}
F( k_y ) = H( k_y ) G( k_y ).
\end{dmath}
%
As mentioned, here we write
%
\begin{dmath}\label{eqn:modernOpticsLecture15:100}
k_y = \frac{2 \pi}{\lambda} \sin\theta.
\end{dmath}
%
(to distinguish from our normal writing of \(k = 2 \pi /\lambda\))

For a single slit, in the Fraunhofer limit, we compute
%
\begin{dmath}\label{eqn:modernOpticsLecture15:120}
G(k_y)
=
\int_{-b/2}^{b/2} e^{-i k_y y} dy
= \evalrange{\frac{e^{-i k_y y}}{-i k_y}}{-b/2}{b/2}
= \frac{
e^{i k_y b/2}
-e^{-i k_y b/2}
}
{
2 (i k_y) /2
}
=
\frac{\sin(b k_y/2)}{b k_y/2}.
\end{dmath}
%
as illustrated in \cref{fig:lecture15figuresSincCrossings:lecture15figuresSincCrossingsFig7}.
% hand drawn:
%\cref{fig:modernOpticsLecture15:modernOpticsLecture15Fig3}.
%\imageFigure{../figures/phy485-optics/modernOpticsLecture15Fig3}{Zero of diffraction intensity.}{fig:modernOpticsLecture15:modernOpticsLecture15Fig3}{0.2}
% mathematica:
\imageFigure{../figures/phy485-optics/lecture15figuresSincCrossingsFig7}{Zero of diffraction wavefunction.}{fig:lecture15figuresSincCrossings:lecture15figuresSincCrossingsFig7}{0.2}
%
For the Fourier transform of the delta comb, we have
%
\begin{dmath}\label{eqn:modernOpticsLecture15:140}
H(k_y)
=
\int_{-\infty}^\infty
e^{i k_y y}
\sum_{n = 0}^{N-1}
\delta(y - n a)
=
\sum_{n = 0}^{N-1}
\int_{-\infty}^\infty
e^{i k_y y}
\delta(y - n a)
=
\sum_{n = 0}^{N-1}
e^{i k_y n a}
=
\sum_{n = 0}^{N-1}
\left( e^{i k_y a}  \right)^n.
\end{dmath}
%
Recall that we can sum a finite geometric series, by taking the difference of
%
\begin{subequations}
\begin{dmath}\label{eqn:modernOpticsLecture15:160}
a S_N = a + a^2 + \cdots a^{N}.
\end{dmath}
\begin{dmath}\label{eqn:modernOpticsLecture15:180}
S_N = 1 + a + \cdots a^{N-1}.
\end{dmath}
\end{subequations}
%
so that
%
\begin{dmath}\label{eqn:modernOpticsLecture15:200}
(a - 1)S_N = a^N - 1.
\end{dmath}
%
or
\begin{dmath}\label{eqn:modernOpticsLecture15:220}
S_N = \frac{a^N - 1}{a-1},
\end{dmath}
%
and we have for our Fourier transform we have
%
\begin{dmath}\label{eqn:modernOpticsLecture15:240}
H(k_y)
=
\frac{1 - e^{i k_y a N }}{1 - e^{i k_y a}}
=
\frac{e^{i k_y a N/2}}{e^{i k_y a/2}}
\frac{e^{-i k_y a N/2} - e^{i k_y a N/2 }}{e^{-i k_y a/2} - e^{i k_y a/2}}
=
e^{i \gamma (N-1)}
\frac{\sin (N\gamma)}{ \sin \gamma}.
\end{dmath}
%
where
\begin{equation}\label{eqn:modernOpticsLecture15:260}
\gamma = \inv{2} k_y a = \inv{2} k a \sin\theta.
\end{equation}
%
We want
%
\begin{dmath}\label{eqn:modernOpticsLecture15:280}
I
= \Abs{F(k_y)}^2
=
\Abs{H(k_y)}^2
\Abs{G(k_y)}^2
= I_0 \left( \frac{\sin\beta}{\beta} \right)^2 \left(
\frac
{\sin N\gamma}
{N \sin \gamma}
\right)^2.
\end{dmath}
%
where
%
\begin{subequations}
\begin{equation}\label{eqn:modernOpticsLecture15:300}
\beta = \inv{2} b k_y = \inv{2} b k \sin\theta.
\end{equation}
\begin{equation}\label{eqn:modernOpticsLecture15:320}
\gamma = \inv{2} k_y a = \inv{2} k a \sin\theta.
\end{equation}
\end{subequations}
%
\paragraph{Check: \(N = 2\)}
%
\begin{dmath}\label{eqn:modernOpticsLecture15:340}
\left( \frac{\sin 2 \gamma}{2 \sin\gamma} \right)^2 =
\left( \frac{2 \sin \gamma \cos\gamma}{2 \sin\gamma} \right)^2 = \cos^2\gamma.
\end{dmath}
%
(Good).

To get a bit of a feeling for what this looks like we can check out a plot, as in \cref{fig:lecture15figures5aperaturesIntensityPlot:lecture15figures5aperaturesIntensityPlotFig8}, plotting \(n = 5, a = 0.869, b = 0.172, \lambda = 1\) (generated from \nbref{lecture15figures.nb}.)
%
\imageFigure{../figures/phy485-optics/lecture15figures5aperaturesIntensityPlotFig8}{A sample intensity pattern for a multiple aperture diffraction grating.}{fig:lecture15figures5aperaturesIntensityPlot:lecture15figures5aperaturesIntensityPlotFig8}{0.2}
%
Our far field view was \cref{fig:modernOpticsLecture15:modernOpticsLecture15Fig4}.
%
\imageFigure{../figures/phy485-optics/modernOpticsLecture15Fig4}{Far field view.}{fig:modernOpticsLecture15:modernOpticsLecture15Fig4}{0.2}
%
\paragraph{Frequency resolution}
\index{frequency resolution}
%\fxwarning{review lecture 15 from this point}{work through lecture in detail from this point.}
How well can we resolve frequency?

We will write out \(\gamma\) in terms of angular frequency
%
\begin{dmath}\label{eqn:modernOpticsLecture15:360}
\gamma
= \inv{2} a \frac{\omega}{c} \sin\theta
= \pi \frac{\omega}{\omega_0} \sin\theta.
\end{dmath}
%
where
%
\begin{dmath}\label{eqn:modernOpticsLecture15:380}
\omega_0 = \frac{2 \pi c}{a},
\end{dmath}
%
where \(a\) is the period of the diffraction grating.  The peak is at \(\gamma = m \pi\), or
%
\begin{dmath}\label{eqn:modernOpticsLecture15:400}
\gamma_{\mathrm{peak}}
= m \pi
= \pi \frac{\omega}{\omega_0} \sin\theta,
\end{dmath}
%
or
\begin{dmath}\label{eqn:modernOpticsLecture15:420}
m = \frac{\omega}{\omega_0} \sin\theta,
\end{dmath}
%
Zeros are at \(N \gamma = l \pi\) (provided \(l \ne N m\)).  Closest zero is
%
\begin{dmath}\label{eqn:modernOpticsLecture15:440}
l = N m + 1.
\end{dmath}
%
\begin{dmath}\label{eqn:modernOpticsLecture15:460}
\Delta \gamma
= \gamma_{\mathrm{zero}} - \gamma_{\mathrm{peak}}
= \frac{l}{N} \pi - m \pi
= \frac{N m + 1}{N} \pi - m \pi.
\end{dmath}
%
or
%
\begin{dmath}\label{eqn:modernOpticsLecture15:480}
\Delta \gamma
= \frac{\pi}{N}
= \frac{\pi \sin\theta}{\omega_0} \Delta \omega.
\end{dmath}
%
\paragraph{Rework after question:}

Peaks for \(\omega_{1,2}\) are
%
\begin{dmath}\label{eqn:modernOpticsLecture15:500}
\gamma_{\mathrm{peak}, 1} = \pi \frac{\omega_1}{\omega_0} \sin\theta = m \pi.
\end{dmath}
\begin{dmath}\label{eqn:modernOpticsLecture15:520}
\gamma_{\mathrm{peak}, 2} = \pi \frac{\omega_2}{\omega_0} \sin\theta = m \pi.
\end{dmath}
%
so that
\begin{dmath}\label{eqn:modernOpticsLecture15:540}
\Delta \gamma
=
\Delta \omega
\frac{\pi}{\omega_0} \sin\theta.
\end{dmath}
%
The peak is resolved when
%
\begin{dmath}\label{eqn:modernOpticsLecture15:560}
\Delta \gamma = \frac{m \pi}{\omega_1} \Delta \omega = \frac{\pi}{N},
\end{dmath}
%
as shown in \cref{fig:modernOpticsLecture15:modernOpticsLecture15Fig5}.
%
\imageFigure{../figures/phy485-optics/modernOpticsLecture15Fig5}{Resolution.}{fig:modernOpticsLecture15:modernOpticsLecture15Fig5}{0.2}
%
Writing \(\omega_1 \sim \omega_2 \sim \overbar{\omega}\), this is
%
\boxedEquation{eqn:modernOpticsLecture15:580}{
\frac{\Delta \omega}{\overbar{\omega}} = \inv{N m}
}

Observe that this looks like
%
\begin{dmath}\label{eqn:modernOpticsLecture15:600}
\frac{\Delta \omega}{\overbar{\omega}} = \inv{m \calF},
\end{dmath}
%
because \(\calF \sim \text{number of bounces in the Etalon}\).

But \(m \approx 1,2,3\), not \(m \gg 1\).

%\EndNoBibArticle
